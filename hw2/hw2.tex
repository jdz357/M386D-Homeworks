\documentclass[letterpaper,twoside,11pt]{article}
\usepackage{a4wide,graphicx,fancyhdr,clrscode,tabularx,amsmath,amssymb,amsfonts,color,enumitem, bm, array, textcomp, subcaption, color, listings, chemformula, tcolorbox, setspace, xcolor}
\usepackage{amsthm}		%theorem style 
%\usepackage{mathptmx}      %SET MATH TYPE FONT TO TIMES NEW ROMAN
%These lines make the theorem NAME BOLD
\newtheoremstyle{mystyle}%                % Name
  {}%                                     % Space above
  {}%                                     % Space below
  {\itshape}%                                     % Body font
  {}%                                     % Indent amount
  {\bfseries}%                            % Theorem head font
  {.}%                                    % Punctuation after theorem head
  { }%                                    % Space after theorem head, ' ', or \newline
  {\thmname{#1}\thmnumber{ #2}\thmnote{ (#3)}}%                                     % Theorem head spec (can be left empty, meaning `normal')
\theoremstyle{mystyle}
\newtheorem{theorem}{Theorem}[section]
%end of making theorem name bold. 
\newtheorem*{thm}{Theorem}		%Theorem no number. 
\newtheorem{definition}{Definition}[section]
\newtheorem{corollary}{Corollary}[theorem]
\newtheorem{lemma}[theorem]{Lemma}
\newtheorem{prop}{Proposition}[section]
\newtheorem*{ex}{Example}
\newtheorem{notee}{Note}[section]
\newtheorem*{note}{Note}
\usepackage[super]{nth}
\usepackage[makeroom]{cancel}

%----------------------- Macros and Definitions --------------------------
\setlength{\fboxsep}{2.5\fboxsep}
%Sets boxlength size

\setlength\headheight{15pt}
\addtolength\topmargin{-25pt}
\addtolength\footskip{0pt}

\fancypagestyle{plain}{%
\fancyhf{}
\fancyhead[LO,RE]{\sffamily UT Austin}
\fancyhead[RO,LE]{\sffamily CSE 386D}
\fancyfoot[LO,RE]{\sffamily Oden Institute}
\fancyfoot[RO,LE]{\sffamily\bfseries\thepage} 
\renewcommand{\headrulewidth}{0pt}
\renewcommand{\footrulewidth}{0pt}
}

\pagestyle{fancy}
\fancyhf{}
\fancyhead[RO,LE]{\sffamily CSE 386D}
\fancyhead[LO,RE]{\sffamily UT Austin}
\fancyfoot[LO,RE]{\sffamily Oden Institute}
\fancyfoot[RO,LE]{\sffamily\bfseries\thepage}
\renewcommand{\headrulewidth}{1pt}
\renewcommand{\footrulewidth}{0pt}
\newcommand{\R}{{\mathbb R}}
\newcommand{\N}{{\mathbb N}}
\newcommand{\Z}{{\mathbb Z}}
\newcommand{\Q}{{\mathbb Q}}
\newcommand{\C}{{\mathbb C}}
\newcommand{\SL}{{\mathcal{L}}}
\usepackage{libertine}            %% For fancy font
\DeclareMathOperator*{\slim}{s-lim}
\newcommand{\cg}{\color{gray}}
\newcommand{\cbk}{\color{black}}
\newcommand{\cred}{\color{red}}
\newcommand{\cblu}{\color{blue}}
\newcommand{\inv}{^{-1}}




\begin{document}
%\fontfamily{ptm}\selectfont     %% TO SELECT THE FONT
\title{\vspace{-2\baselineskip} 
Homework 2
}
%\author{Jonathan Zhang \qquad EID: { jdz357} \qquad { jdz@utexas.edu}}
\author{Jonathan Zhang \qquad EID: { jdz357} }
\date{}
\maketitle


\subsection*{Problem 6.9:}
\textit{
Suppose that $f\in L^p\left( \R^d \right)$ for $p \in \left( 1, 2 \right)$. Show that there exist $f_1 \in L^1\left( \R^d \right)$ and $f_2 \in L^2\left( \R^d \right)$ such that $f = f_1 + f_2$. 
Define $\hat f = \hat f_1 + \hat f_2$. Show that this definition is well defined, that is, independent of the choices $f_1$ and $f_2$. }

\paragraph*{Solution}
Write $f$ as 
\[f = f\chi_{[x:|f(x)<1]} + f\chi_{[x:|f(x)\geqslant 1]}\]
and define 
\[f_2(x) = f\chi_{[x:|f(x)<1]},\qquad f_1(x) = f\chi_{[x:|f(x)\geqslant 1]}.\]
Then, 
\[\|f_1\|_1 = \int\limits_{\R^d} |f(x)|\chi_{[x:|f(x)\geqslant 1]}dx = \int\limits_{x:|f(x)|\geqslant 1}|f(x)| \leq\int\limits_{x:|f(x)|\geqslant 1}|f(x)|^p \leq \|f\|_p^p. \]
\[\|f_2\|_2^2 = \int\limits_{\R^d} |f(x)|^2\chi_{[x:|f(x)< 1]}dx = \int\limits_{x:|f(x)|< 1}|f(x)|^2 \leq\int\limits_{x:|f(x)|< 1}|f(x)|^p \leq \|f\|_p^p. \]

Next, suppose that $f$ admits two different decompositions, $f = f_1 + f_2 = g_1 + g_2$, where $f_i, g_i \in L^i\left( \R^d \right)$ for $i = 1, 2$. Then, 
\begin{align*}
\hat f_1 + \hat f_2 &= \left( 2\pi \right)^{-d/2}\left(\int\limits_{\R^d} f_1(x) e^{-ix\cdot \xi } dx+\int\limits_{\R^d} f_2(x) e^{-ix\cdot \xi } dx \right)\\
&=\left( 2\pi \right)^{-d/2} \int\limits_{\R^d} \left( f_1(x) + f_2(x)  \right)e^{-ix\cdot \xi }dx\\
&=\left( 2\pi \right)^{-d/2} \int\limits_{\R^d} \left( g_1(x) + g_2(x)  \right)e^{-ix\cdot \xi }dx\\
&=\hat g_1 + \hat g_2.
\end{align*}














\newpage
\subsection*{Problem 6.11:}
\textit{Let the ground field be $\mathbb C$ and define $T: L^2\left( \R^d  \right) \to L^2$ by }
\[\left( Tf \right)(x) = \int e^{-|x-y|^2/2} f(y) dy.\]
\textit{Use the Fourier transform to show that $T$ is a positive, injective but not surjective operator. }
\paragraph*{Solution} 
Define $\varphi(x) = e^{-|x|^2/2}$. Observe that $T$ is nothing more than convolution, i.e. 
\[\left( {Tf} \right)\left( x \right) = \left( {\varphi  * f} \right)\left( x \right) = \int\limits_{{\mathbb{R}^d}} {\varphi \left( {x - y} \right)f\left( y \right)dy}  = \int\limits_{{\mathbb{R}^d}} {{e^{ - {{\left| {x - y} \right|}^2}/2}}f\left( y \right)dy} .\]
If we consider now the Fourier transform,
\[\widehat {\left( {Tf} \right)} = \widehat {\left( {\varphi  * f} \right)} = {\left( {2\pi } \right)^{d/2}}\hat \varphi \hat f.\]
The Fourier transform preserves inner product, so 
\[\left( {Tf,f} \right) = \left( {\widehat {\left( {Tf} \right)},\hat f} \right) = \int\limits_{{\mathbb{R}^d}} {{{\left( {2\pi } \right)}^{d/2}}\hat \varphi \left( \xi  \right){{\left\{ {\hat f\left( \xi  \right)} \right\}}^2}d\xi }  \geqslant 0\]
Notice this is the case since $\varphi = \hat \varphi >0$. So we conclude that $T$ is positive. 

\cred show T is injective?
\cbk 
















\newpage 
\subsection*{Problem 6.15:}
\textit{Let $\varphi \in \mathcal S \left( \R^d \right)$, $\hat \varphi \left( 0 \right) = \left( 2\pi \right)^{-d/2}$, and $\varphi_\varepsilon \left( x \right) = \varepsilon^{-d} \varphi\left( x/\varepsilon \right)$. Prove that $\varphi_\varepsilon \to \delta_0$ and $\hat \varphi_\varepsilon \to \left( 2\pi  \right)^{-d/2}$ as $\varepsilon \to 0^+$. In what sense do these convergences take place? }

We claim that $\varphi_\varepsilon \to \delta_0$ in $\mathcal S'$. 




















\end{document}
 