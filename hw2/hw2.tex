\documentclass[letterpaper,twoside,11pt]{article}
\usepackage{a4wide,graphicx,fancyhdr,clrscode,tabularx,amsmath,amssymb,amsfonts,color,enumitem, bm, array, textcomp, subcaption, color, listings, chemformula, tcolorbox, setspace, xcolor}
\usepackage{amsthm}		%theorem style 
%\usepackage{mathptmx}      %SET MATH TYPE FONT TO TIMES NEW ROMAN
%These lines make the theorem NAME BOLD
\newtheoremstyle{mystyle}%                % Name
  {}%                                     % Space above
  {}%                                     % Space below
  {\itshape}%                                     % Body font
  {}%                                     % Indent amount
  {\bfseries}%                            % Theorem head font
  {.}%                                    % Punctuation after theorem head
  { }%                                    % Space after theorem head, ' ', or \newline
  {\thmname{#1}\thmnumber{ #2}\thmnote{ (#3)}}%                                     % Theorem head spec (can be left empty, meaning `normal')
\theoremstyle{mystyle}
\newtheorem{theorem}{Theorem}[section]
%end of making theorem name bold. 
\newtheorem*{thm}{Theorem}		%Theorem no number. 
\newtheorem{definition}{Definition}[section]
\newtheorem{corollary}{Corollary}[theorem]
\newtheorem{lemma}[theorem]{Lemma}
\newtheorem{prop}{Proposition}[section]
\newtheorem*{ex}{Example}
\newtheorem{notee}{Note}[section]
\newtheorem*{note}{Note}
\usepackage[super]{nth}
\usepackage[makeroom]{cancel}

%----------------------- Macros and Definitions --------------------------
\setlength{\fboxsep}{2.5\fboxsep}
%Sets boxlength size

\setlength\headheight{15pt}
\addtolength\topmargin{-25pt}
\addtolength\footskip{0pt}

\fancypagestyle{plain}{%
\fancyhf{}
\fancyhead[LO,RE]{\sffamily UT Austin}
\fancyhead[RO,LE]{\sffamily CSE 386D}
\fancyfoot[LO,RE]{\sffamily Oden Institute}
\fancyfoot[RO,LE]{\sffamily\bfseries\thepage} 
\renewcommand{\headrulewidth}{0pt}
\renewcommand{\footrulewidth}{0pt}
}

\pagestyle{fancy}
\fancyhf{}
\fancyhead[RO,LE]{\sffamily CSE 386D}
\fancyhead[LO,RE]{\sffamily UT Austin}
\fancyfoot[LO,RE]{\sffamily Oden Institute}
\fancyfoot[RO,LE]{\sffamily\bfseries\thepage}
\renewcommand{\headrulewidth}{1pt}
\renewcommand{\footrulewidth}{0pt}
\newcommand{\R}{{\mathbb R}}
\newcommand{\N}{{\mathbb N}}
\newcommand{\Z}{{\mathbb Z}}
\newcommand{\Q}{{\mathbb Q}}
\newcommand{\C}{{\mathbb C}}
\newcommand{\SL}{{\mathcal{L}}}
\usepackage{libertine}            %% For fancy font
\DeclareMathOperator*{\slim}{s-lim}
\newcommand{\cg}{\color{gray}}
\newcommand{\cbk}{\color{black}}
\newcommand{\cred}{\color{red}}
\newcommand{\cblu}{\color{blue}}
\newcommand{\inv}{^{-1}}




\begin{document}
%\fontfamily{ptm}\selectfont     %% TO SELECT THE FONT
\title{\vspace{-2\baselineskip} 
Homework 2
}
%\author{Jonathan Zhang \qquad EID: { jdz357} \qquad { jdz@utexas.edu}}
\author{Jonathan Zhang \qquad EID: { jdz357} }
\date{}
\maketitle


\subsection*{Problem 6.9:}
\textit{
Suppose that $f\in L^p\left( \R^d \right)$ for $p \in \left( 1, 2 \right)$. Show that there exist $f_1 \in L^1\left( \R^d \right)$ and $f_2 \in L^2\left( \R^d \right)$ such that $f = f_1 + f_2$. 
Define $\hat f = \hat f_1 + \hat f_2$. Show that this definition is well defined, that is, independent of the choices $f_1$ and $f_2$. }






\subsection*{Problem 6.11:}
\textit{Suppose that $f$ and $g$ are in $L^2\left( \R^d \right)$. The convolution $f \ast g \in L^\infty\left( \R^d \right)$, so it may not have a Fourier Transform. Prove that $f \ast g = \left( 2\pi  \right)^{d/2}\mathcal F^{-1} \left( \hat f \hat g \right)$ is well defined, using the natural Fourier inverse integration formula. }


\subsection*{Problem 6.15:}
\textit{Is it psosible for there to be a continuous function $f$ defined on $\R^d$ with the following properties?}
\begin{enumerate}
  \item \textit{There is no polynomial $P$ in $d$ variables such that $|f(x)| \leq P(x)$ for all $x\in \R^d$.} 
  \item \textit{The distribution $\phi \mapsto \int \phi f$ is tempered.}
\end{enumerate}









\subsection*{Problem 6.18:}
\textit{The gamma function is $\Gamma(s) = \displaystyle \int\limits_0^\infty t^{s-1}e^{-t}dt$. Let $\phi \in \mathcal S\left( \R^d \right)$ and $0 <\alpha <d$. }
\begin{enumerate}
  \item \textit{Show that $|\xi|^\alpha\in L_{loc}^1\left( \R^d \right)$ and $|\xi|^\alpha \hat \phi \in L^1\left( \R^d \right)$.}
  \item \textit{Let $c_\alpha = 2^{\alpha/2} \Gamma\left( \alpha/2 \right)$. Show that }
  \[\mathcal F^{-1} \left( |\xi|^{-\alpha} \hat \phi \right) (x) = \frac{c_{d-\alpha}}{\left( 2\pi \right)^{d/2}c_\alpha} \int\limits_{\R^d} |x-y|^{\alpha - d}\phi(y) dy  \] 
  Hint: first show that $c_\alpha |\xi|^{-\alpha} = \int\limits_0^\infty t^{\alpha/2-1} e^{-|xi|^2 t/2}dt$ and recall $e^{-|\xi|^2 t/2}=t^{-d/2}\hat{\left( e^{-|x|^2/2t} \right)}$. 
\end{enumerate}










\subsection*{Problem 6.20:}
\textit{Argue that $\mathcal D\left( \R^d \right)$ is dense in $\mathcal S$. Show also that $\mathcal S'$ is dense in $\mathcal D'$ and that distributions with compact support are dense in $\mathcal S'$. }







\end{document}
 