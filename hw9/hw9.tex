\documentclass[letterpaper,twoside,11pt]{article}
\usepackage{a4wide,graphicx,fancyhdr,clrscode,tabularx,amsmath,amssymb,amsfonts,color,enumitem, bm, array, textcomp, subcaption, color, listings, chemformula, tcolorbox, setspace, xcolor}
\usepackage{amsthm}		%theorem style 
%\usepackage{mathptmx}      %SET MATH TYPE FONT TO TIMES NEW ROMAN
%These lines make the theorem NAME BOLD
\newtheoremstyle{mystyle}%                % Name
  {}%                                     % Space above
  {}%                                     % Space below
  {\itshape}%                                     % Body font
  {}%                                     % Indent amount
  {\bfseries}%                            % Theorem head font
  {.}%                                    % Punctuation after theorem head
  { }%                                    % Space after theorem head, ' ', or \newline
  {\thmname{#1}\thmnumber{ #2}\thmnote{ (#3)}}%                                     % Theorem head spec (can be left empty, meaning `normal')
\theoremstyle{mystyle}
\newtheorem{theorem}{Theorem}[section]
%end of making theorem name bold. 
\newtheorem*{thm}{Theorem}		%Theorem no number. 
\newtheorem{definition}{Definition}[section]
\newtheorem{corollary}{Corollary}[theorem]
\newtheorem{lemma}[theorem]{Lemma}
\newtheorem{prop}{Proposition}[section]
\newtheorem*{ex}{Example}
\newtheorem{notee}{Note}[section]
\newtheorem*{exercise}{Exercise}
\newtheorem*{note}{Note}
\usepackage[super]{nth}
\usepackage[makeroom]{cancel}

%----------------------- Macros and Definitions --------------------------
\setlength{\fboxsep}{2.5\fboxsep}
%Sets boxlength size

\setlength\headheight{15pt}
\addtolength\topmargin{-25pt}
\addtolength\footskip{0pt}

\fancypagestyle{plain}{%
\fancyhf{}
\fancyhead[LO,RE]{\sffamily UT Austin}
\fancyhead[RO,LE]{\sffamily CSE 386D}
\fancyfoot[LO,RE]{\sffamily Oden Institute}
\fancyfoot[RO,LE]{\sffamily\bfseries\thepage} 
\renewcommand{\headrulewidth}{0pt}
\renewcommand{\footrulewidth}{0pt}
}

\pagestyle{fancy}
\fancyhf{}
\fancyhead[RO,LE]{\sffamily CSE 386D}
\fancyhead[LO,RE]{\sffamily UT Austin}
\fancyfoot[LO,RE]{\sffamily Oden Institute}
\fancyfoot[RO,LE]{\sffamily\bfseries\thepage}
\renewcommand{\headrulewidth}{1pt}
\renewcommand{\footrulewidth}{0pt}
\newcommand{\R}{{\mathbb R}}
\newcommand{\N}{{\mathbb N}}
\newcommand{\Z}{{\mathbb Z}}
\newcommand{\Q}{{\mathbb Q}}
\newcommand{\C}{{\mathbb C}}
\newcommand{\SL}{{\mathcal{L}}}
\usepackage{libertine}            %% For fancy font
\DeclareMathOperator*{\slim}{s-lim}
\newcommand{\cg}{\color{gray}}
\newcommand{\cbk}{\color{black}}
\newcommand{\cred}{\color{red}}
\newcommand{\cblu}{\color{blue}}
\newcommand{\inv}{^{-1}}
\newcommand{\sch}{\mathcal S} 




\begin{document}
%\fontfamily{ptm}\selectfont     %% TO SELECT THE FONT
\title{\vspace{-2\baselineskip} 
Homework 9
}
%\author{Jonathan Zhang \qquad EID: { jdz357} \qquad { jdz@utexas.edu}}
\author{Jonathan Zhang \qquad EID: { jdz357} }
\date{}
\maketitle


Unless otherwise stated, denote by $\left( \cdot, \cdot \right)$ as the $L^2$ inner product, and $\|\cdot\|$ as $\|\cdot \|_{L^2\left( \Omega \right)}$. 

\begin{exercise}[19]
	Let $\Omega \subset \R^4$ be a bounded domain with smooth boundary, $f \in L^2 (\Omega)$. Consider the BVP 
	\[-\nabla^2 u + u^3 = f\]
	with $u = 0$ on $\partial \Omega$.
  We attempt to solve iteratively from $u_0 = 0$ by computing for each $n = 1, 2,\dots$ the solutions to the linear BVP 
  \[-\nabla^2 u_n + u_{n-1}^2 u_n = f,\qquad u_n = 0 \text{ on }\partial \Omega.\]
  \begin{enumerate}
    \item Find appropraite variational problems for the linear BVP's and show that they are well defined in $H_0^1 \left( \R^4 \right)$ provided that $u_{n-1} \in H_0^1 \left( \R^4 \right)$. (Hint: Sobolev Imbedding Theorem.)
    \item Show that there is a unique solution $u_n \in H_0^1 \left( \R^4 \right)$ assuming that $u_{n-1} \in H_0^1 (\R^4)$. Moreover, find a bound for the norm of $u_n$. 
    \item Show that the nonlinear BVP has a weak solution. Extract a subsequence of $u_n$ that converges weakly to some $u$ and show that $u$ satisfies the weak form of the nonlinear BVP. 
  \end{enumerate}
\end{exercise}

\cblu
\begin{proof}
  \begin{enumerate}
    \item Assume that $u_{n-1} \in H_0^1 \left( \R^4 \right)$ is given. Multiply by a test function and integrate by parts the first term, 
    \begin{align*}
      -\left( \nabla^2 u_n, v \right) + \left( u_{n-1}^2 u_n ,v \right) &= \left( f, v \right)\\
      \left( \nabla u_n, \nabla v \right) - \langle \nabla u_n \cdot n, v\rangle + \left( u_{n-1}^2 u_n ,v \right) &= \left( f, v \right) 
    \end{align*}
    For $v \in H_0^1 \left( \Omega \right)$, we have 
    \[B(u_n, v) := \left( \nabla u_n, \nabla v \right) + \left( u_{n-1}^2 u_n ,v \right) = \left( f, v \right)=: F(v) .\]
    The VP reads 
    \[\left\{ {\begin{array}{*{20}{l}}
      u_n \in H_0^1(\Omega), \\[.2cm]
      \left( \nabla u_n, \nabla v \right) + \left( u_{n-1}^2 u_n v \right) = \left( f, v \right)\\[.2cm]
      \quad \text{for every } v\in H_0^1 \left( \Omega \right).
    \end{array}} \right.\]
    We need to check that the second integral is well-defined. 
    We recall that for the zero spaces, we simply extend by zero to the whole space with the equality $\|f\|_{H_0^1 (\Omega)} = \|f\|_{H_0^1(\R^4)}$. 

    Let $p = 2$, $j = 0$, $m = 1$, and $d = 4$. The Sobolev Embedding Theorem tells us that $H_0^1 (\Omega) \equiv W_0^{1, 2} \left( \Omega \right)\hookrightarrow W^{0,q}\left( \Omega \right)\equiv L^q \left( \Omega \right)$ for all finite $q \leq dp/(d-mp) = 4$. We will use exactly $q = 4$ for the estimate
    \begin{align*}
      |\left( u_{n-1}^2u_n, v \right)| &\leq \|u_{n-1}\|^2_{L^4(\Omega)} \|u_n \|_{L^4(\Omega)} \|v\|_{L^4(\Omega)}  &\text{H\"older's}\\
      &\leq C\|u_{n-1}\|^2_{H_0^1(\R^4)} \|u_n \|_{H_0^1(\Omega)} \|v\|_{H_0^1(\Omega)}  &\text{Sobolev Emb.}\\
      &= C\|u_{n-1}\|^2_{H_0^1(\R^4)} \|u_n \|_{H_0^1(\R^4)} \|v\|_{H_0^1(\R^4)}  &\text{Bd. Ext. Op.}
    \end{align*}
    So, the problem is well-defined in the space $H^1 (\R^4)$ in the sense that all terms are well-defined, and the solution obtained from solving the VP which lives presently in $H_0^1 \left( \Omega \right)$ can be easily extended to $H^1 \left( \R^4 \right)$. 

    \item We have a symmetric functional setting, so Lax-Milgram is the tool. The right hand side is trivially continuous by H\"older's Inequality given $f \in L^2 (\Omega)$. The left hand side is also continuous, 
    \begin{align*}
      B(u_n,v) &= \left( \nabla u_n, \nabla v \right) + \left( u_{n-1}^2 u_n ,v \right)\\
      &\leq \|\nabla u_n\| \|\nabla v\| + C\|u_{n-1}\|^2_{H_0^1(\R^4)} \|u_n \|_{H_0^1(\R^4)} \|v\|_{H_0^1(\R^4)} \\
      &\leq \left( 1+C\|u_{n-1}\|^2_{H_0^1(\R^4)} \right)\|u_n \|_{H_0^1(\Omega)} \|v\|_{H_0^1(\Omega)}.
    \end{align*}
    The bilinear form is also coercive, 
    \begin{align*}
      B\left( v, v \right) &= \|\nabla v\|^2 + \int\limits_\Omega u_{n-1}^2 v^2 \\ 
      &\geq \|\nabla v\|^2 \\
      &\geq C_P^{-2} \|v\|_{H^1_0(\Omega)}^2.
    \end{align*}
    By Lax-Milgram, the problem admits a unique solution $u_n \in H_0^1 \left( \Omega \right)$ which we can extend by zero to $u_n \in H_0^1 \left( \R^4 \right)$. Moreover, we have the estimate 
    \[\|u_n\|_{H^1_0 \left( \Omega \right)} = \|u_n\|_{H^1_0 \left( \R^4 \right)} \leq C_P^2 \|f\|.\] Note the bound is independent of $n$. 
    \item Notice that $u_n$ has just been shown to be a bounded sequence in the Hilbert space $H_0^1 \left( \Omega \right)$. Therefore, there is a subsequence $u_{n_k} \rightharpoonup u$ in $H_0^1(\Omega)$, furthermore, $u_{n_k} \to u$ in $L^2 (\Omega)$ and $\nabla u_{n_k} \rightharpoonup \nabla u$ in $L^2 (\Omega)^{4}$. 
    We argue that $u$ is indeed a weak solution to the nonlinear PDE. For every test function $v$, $\left( \cdot, \nabla v \right)$ is a linear functional on $L^2(\Omega)^4$, so $\left( \nabla u_{n_k}, \nabla v \right) \to \left( \nabla u, \nabla v \right)$. Additionally, $u_{n_{k-1}}^2 u_{n_k} \to u^3$, so $\left( u_{n_{k-1}}^2 u_{n_k} ,v \right) \to \left( u^3, v \right)$. This implies 
    \[B(u_{n_k}, v) \to \left( \nabla u, \nabla v \right) + \left( u^3, v \right) = \left( f, v \right)\]
    which is exactly the weak form of the original nonlinear problem. 
  \end{enumerate}
\end{proof}
\cbk 
















\begin{exercise}[21]
  $\Omega \subset \R^d$ bounded domain with Lipschitz $\partial \Omega$, define 
  \[H\left( \text{div}, \Omega \right) = \left\{ v\in (L^2(\Omega))^d : \nabla \cdot v \in L^2 (\Omega )\right\}.\]
\begin{enumerate}
  \item Show that $H\left( \text{div}, \Omega \right)$ is a Hilbert space with inner product 
  \[\left( u,v \right)_{H(\text{div}, \Omega)} = \left( u,v \right) + \left( \nabla \cdot u, \nabla \cdot v \right).\]
  \item The trace Theorem does not imply that $\partial_\nu v = v \cdot \nu$ exists on $\partial \Omega$. Nevertheless, show that $\partial_\nu : H\left( \text{div}, \Omega \right) \to H^{-1/2} \left( \partial \Omega \right) = \left( H^{1/2}(\partial \Omega) \right)^*$ is a well defined bounded linear oeprator in the sense that 
  \[\int\limits_{\partial \Omega} v \cdot \nu \phi d\sigma(x) = \int\limits_{\Omega} \nabla \cdot v \phi dx + \int\limits_{\Omega} v\cdot \nabla \phi dx.\]
  \item Prove the following inf-sup condition: there exists $\gamma > 0$ such that 
  \[\inf_{w\in L^2} \sup_{v\in H(\text{div},\Omega)} \frac{(w, \nabla \cdot v)}{\|w\| \|v\|_{H(\text{div}, \Omega)}}\geq \gamma > 0.\]
  Hint solve $\Delta \varphi = w$ in $H_0^1 \left( \Omega \right)$ and consider $v  = \nabla \varphi$. 

\end{enumerate}
\end{exercise}

\begin{exercise}[25] 
  Consider the finite element method. 
  \begin{enumerate}
    \item Modify the method to account for nonhomogeneous Neumann conditions. 
    \item Modify the method to account for nonhomogeneous Dirichlet conditions. 
  \end{enumerate}
\end{exercise}

\begin{exercise}[27]
  Suppose $u \in H^1\left( \Omega \right)$ where $\Omega\subset \R^d$ is bounded connected. Recall the $H^1$ seminorm is $|u|_{H^1} = \left\{ \sum_{|\alpha| = 1} \|D^\alpha u \|^2 \right\}^{1/2}$. 
  \begin{enumerate}
    \item Show that there is a constant $C_\Omega$ such that 
    \[\inf_{c\in \R} \|u-c\| \leq C_\Omega |u|_{H^1}.\]
    \item Let $\Omega = \left( 0, h \right)^d$ for $h > 0$. Show that there is a constant $C$ independent of $h$ and $u$ such that 
    \[\inf_{c\in \R} \|u-c\| \leq Ch |u|_{H^1}.\]
    Change var. to integrate over $(0,1)^d$ and then use previous. 
    \item Let $\Omega = \left( 0,1 \right)^d$ and let $P$ be the set of piecewise discontinuous constants over the grid of spacing $h = 1/N$ for some positive integer $N$. Show that there is a constant $C$ independent of $h$, $u$, such that 
    \[\inf_{p\in P} \|u-p\| \leq Ch |u|_{H^1}.\]
  \end{enumerate}
\end{exercise}

\begin{exercise}[29]
  Consider the problem 

  \begin{enumerate}
    \item Find the Green's Function. 
    \item Instead impose Neumann BC's, and find the Green's function. Recall we now require $-\partial^2/\partial x^2 G(x, y) = \delta_y(x) -1$.
  \end{enumerate}
\end{exercise}

\end{document}
 
