\documentclass[letterpaper,twoside,11pt]{article}
\usepackage{a4wide,graphicx,fancyhdr,clrscode,tabularx,amsmath,amssymb,amsfonts,color,enumitem, bm, array, textcomp, subcaption, color, listings, chemformula, tcolorbox, setspace, xcolor}
\usepackage{amsthm}		%theorem style 
%\usepackage{mathptmx}      %SET MATH TYPE FONT TO TIMES NEW ROMAN
%These lines make the theorem NAME BOLD
\newtheoremstyle{mystyle}%                % Name
  {}%                                     % Space above
  {}%                                     % Space below
  {\itshape}%                                     % Body font
  {}%                                     % Indent amount
  {\bfseries}%                            % Theorem head font
  {.}%                                    % Punctuation after theorem head
  { }%                                    % Space after theorem head, ' ', or \newline
  {\thmname{#1}\thmnumber{ #2}\thmnote{ (#3)}}%                                     % Theorem head spec (can be left empty, meaning `normal')
\theoremstyle{mystyle}
\newtheorem{theorem}{Theorem}[section]
%end of making theorem name bold. 
\newtheorem*{thm}{Theorem}		%Theorem no number. 
\newtheorem{definition}{Definition}[section]
\newtheorem{corollary}{Corollary}[theorem]
\newtheorem{lemma}[theorem]{Lemma}
\newtheorem{prop}{Proposition}[section]
\newtheorem*{ex}{Example}
\newtheorem{notee}{Note}[section]
\newtheorem*{note}{Note}
\usepackage[super]{nth}
\usepackage[makeroom]{cancel}

%----------------------- Macros and Definitions --------------------------
\setlength{\fboxsep}{2.5\fboxsep}
%Sets boxlength size

\setlength\headheight{15pt}
\addtolength\topmargin{-25pt}
\addtolength\footskip{0pt}

\fancypagestyle{plain}{%
\fancyhf{}
\fancyhead[LO,RE]{\sffamily UT Austin}
\fancyhead[RO,LE]{\sffamily CSE 386C}
\fancyfoot[LO,RE]{\sffamily Oden Institute}
\fancyfoot[RO,LE]{\sffamily\bfseries\thepage} 
\renewcommand{\headrulewidth}{0pt}
\renewcommand{\footrulewidth}{0pt}
}

\pagestyle{fancy}
\fancyhf{}
\fancyhead[RO,LE]{\sffamily CSE 386C}
\fancyhead[LO,RE]{\sffamily UT Austin}
\fancyfoot[LO,RE]{\sffamily Oden Institute}
\fancyfoot[RO,LE]{\sffamily\bfseries\thepage}
\renewcommand{\headrulewidth}{1pt}
\renewcommand{\footrulewidth}{0pt}
\newcommand{\R}{{\mathbb R}}
\newcommand{\N}{{\mathbb N}}
\newcommand{\Z}{{\mathbb Z}}
\newcommand{\Q}{{\mathbb Q}}
\newcommand{\C}{{\mathbb C}}
\newcommand{\SL}{{\mathcal{L}}}
\usepackage{libertine}            %% For fancy font
\DeclareMathOperator*{\slim}{s-lim}
\newcommand{\cg}{\color{gray}}
\newcommand{\cbk}{\color{black}}
\newcommand{\cred}{\color{red}}
\newcommand{\cblu}{\color{blue}}




\begin{document}
%\fontfamily{ptm}\selectfont     %% TO SELECT THE FONT
\title{\vspace{-2\baselineskip} 
A fun problem
}
%\author{Jonathan Zhang \qquad EID: { jdz357} \qquad { jdz@utexas.edu}}
\author{Jonathan Zhang  }
\date{}
\maketitle


Consider a square of side length $4$, with a half-circle and quarter circle inscribed in it as shown below: 

\begin{center}
  \includegraphics*[scale=0.25]{Untitled.png}
\end{center}
Determine the enclosed area $S$. 

\paragraph*{A quick solution with calculus:}
It is easy to see that the enclosed area can be calculated by integrating the difference of the curves of the two circles. 
We begin by calculating the coordinates of the intersection point. Let the origin be the lower left corner of the figure. Then the half circle is givn by 
\[(x-2)^2 + y^2 = 4\]
and the quarter circle by 
\[x^2 + (y-4)^2 = 16. \]
Elementary calculation reveals that the intersection point of the two curves is located at $\left(\frac{16}{5}, \frac{8}{5}\right)$. It is then only the matter of evaluating the correct integral, 
\[\int\limits_0^{16/5} \sqrt{4-(x-2)^2} - (4-\sqrt{16-x^2}) dx \approx 3.847. \]

\paragraph*{A more elementary solution:} With the help of some cleverly drawn lines, the problem can be solved exactly with nothing more than simple formulae. We recall some fundemantal facts. Given an isoscoles triangle as shown, its area is given by $A = \frac{1}{2}R^2 \sin(\vartheta)$. 
\begin{center}
  \includegraphics*[scale=0.25]{triangle.png}
\end{center}
Additionally, the area of a sector swept by the angle $\vartheta$ is $\frac{1}{2}\vartheta R^2$. The area of what is leftover therefore is $\frac{1}{2}R^2 \vartheta - \frac{1}{2}R^2 \sin\vartheta$.

Consider now the same figure with some additional lines and points: 
\begin{center}
  \includegraphics*[scale=0.25]{addedlines.png}
\end{center}
We see that $S = S_1 + S_2$, and that triangles $OCD$ and $ODE$ are isoscoles triangles for which we can apply this rule. Our task now is only to find the angles $\vartheta_1$ and $\vartheta_2$. 
We see that $\sin\vartheta_1 = (16/5)/4 = 4/5$. This means that 
\[S_2 = \frac{1}{2}4^2 \arcsin(4/5) - \frac{1}{2}4^2\frac{4}{5} = 8\arcsin(4/5) - \frac{32}{5}.\]
Additionally, we see that $\sin(\pi - \vartheta_2) = (8/5)/2 = 4/5$ so $\vartheta_2 = \pi - \arcsin(4/5)$. This means that 
\[S_1 = \frac{1}{2}2^2 (\pi - \arcsin(4/5)) - \frac{1}{2}2^2\sin(\pi - \arcsin(4/5)) = 2(\pi - \arcsin(4/5)) - \frac{8}{5}\]
Summing up, 

\[S = S_1 + S_2 = 2(\pi - \arcsin(4/5)) - \frac{8}{5} + 8\arcsin(4/5) - \frac{32}{5} = 2\pi + 6\arcsin(4/5) -8 \approx 3.847.\]



\end{document}
 