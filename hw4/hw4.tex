\documentclass[letterpaper,twoside,11pt]{article}
\usepackage{a4wide,graphicx,fancyhdr,clrscode,tabularx,amsmath,amssymb,amsfonts,color,enumitem, bm, array, textcomp, subcaption, color, listings, chemformula, tcolorbox, setspace, xcolor}
\usepackage{amsthm}		%theorem style 
%\usepackage{mathptmx}      %SET MATH TYPE FONT TO TIMES NEW ROMAN
%These lines make the theorem NAME BOLD
\newtheoremstyle{mystyle}%                % Name
  {}%                                     % Space above
  {}%                                     % Space below
  {\itshape}%                                     % Body font
  {}%                                     % Indent amount
  {\bfseries}%                            % Theorem head font
  {.}%                                    % Punctuation after theorem head
  { }%                                    % Space after theorem head, ' ', or \newline
  {\thmname{#1}\thmnumber{ #2}\thmnote{ (#3)}}%                                     % Theorem head spec (can be left empty, meaning `normal')
\theoremstyle{mystyle}
\newtheorem{theorem}{Theorem}[section]
%end of making theorem name bold. 
\newtheorem*{thm}{Theorem}		%Theorem no number. 
\newtheorem{definition}{Definition}[section]
\newtheorem{corollary}{Corollary}[theorem]
\newtheorem{lemma}[theorem]{Lemma}
\newtheorem{prop}{Proposition}[section]
\newtheorem*{ex}{Example}
\newtheorem{notee}{Note}[section]
\newtheorem*{note}{Note}
\usepackage[super]{nth}
\usepackage[makeroom]{cancel}

%----------------------- Macros and Definitions --------------------------
\setlength{\fboxsep}{2.5\fboxsep}
%Sets boxlength size

\setlength\headheight{15pt}
\addtolength\topmargin{-25pt}
\addtolength\footskip{0pt}

\fancypagestyle{plain}{%
\fancyhf{}
\fancyhead[LO,RE]{\sffamily UT Austin}
\fancyhead[RO,LE]{\sffamily CSE 386D}
\fancyfoot[LO,RE]{\sffamily Oden Institute}
\fancyfoot[RO,LE]{\sffamily\bfseries\thepage} 
\renewcommand{\headrulewidth}{0pt}
\renewcommand{\footrulewidth}{0pt}
}

\pagestyle{fancy}
\fancyhf{}
\fancyhead[RO,LE]{\sffamily CSE 386D}
\fancyhead[LO,RE]{\sffamily UT Austin}
\fancyfoot[LO,RE]{\sffamily Oden Institute}
\fancyfoot[RO,LE]{\sffamily\bfseries\thepage}
\renewcommand{\headrulewidth}{1pt}
\renewcommand{\footrulewidth}{0pt}
\newcommand{\R}{{\mathbb R}}
\newcommand{\N}{{\mathbb N}}
\newcommand{\Z}{{\mathbb Z}}
\newcommand{\Q}{{\mathbb Q}}
\newcommand{\C}{{\mathbb C}}
\newcommand{\SL}{{\mathcal{L}}}
%\usepackage{libertine}            %% For fancy font
\DeclareMathOperator*{\slim}{s-lim}
\newcommand{\cg}{\color{gray}}
\newcommand{\cbk}{\color{black}}
\newcommand{\cred}{\color{red}}
\newcommand{\cblu}{\color{blue}}
\newcommand{\inv}{^{-1}}
\newcommand{\sch}{\mathcal S} 




\begin{document}
%\fontfamily{ptm}\selectfont     %% TO SELECT THE FONT
\title{\vspace{-2\baselineskip} 
Homework 4
}
%\author{Jonathan Zhang \qquad EID: { jdz357} \qquad { jdz@utexas.edu}}
\author{Jonathan Zhang \qquad EID: { jdz357} }
\date{}
\maketitle


\subsection*{Problem 7.1:}
Prove that for $f \in H^1\left( \R^d \right)$, $\|f\|_{H^1\left( \R^d \right)}$ is equivalent to 
\[\left[\int\limits_{\R^d} \left( 1+|\xi|^2 \right)|\hat f\left( \xi \right)|^2d\xi \right]^{1/2}\]
Generalize to $H^k\left( \R^d \right)$?

\paragraph*{Solution}
Make use of Plancherel theorem. Let $f\in H^1\left( \R^d \right)$. Then $f$ and $Df$ are $L^2$. The norm is 
\[\|f\|_{H^1}^2 = \|f\|^2_{L^2} + \|Df\|^2_{L^2} = \|\hat f\|^2_{L^2} + \|\left( Df \right)^\land\|^2_{L^2} = \int |\hat f|^2 + |\left( Df \right)^\land|^2 = \int \left( 1+|\xi|^2 \right) |\hat f|^2.\]
The result also generalizes to $H^k$. In this case, propose a norm 
\[\|f\|_{H^k\left( \R^d \right)} = \left[\int\limits_{\R^d} \left( 1+|\xi|^2 \right)^k|\hat f\left( \xi \right)|^2d\xi \right]^{1/2}.\]
We need to show that there exist constants $C_1, C_2 >0$ that bound the two norms. But, this boils down to showing that there exists $C_1, C_2$ such that 
\[C_1 \left( 1+x^2 \right)^{k/2} \leq \sum_{r=0}^{k} x^r \leq C_2 \left( 1+x^2 \right)^{k/2}. \]
Equivalently, 
\[C_1 \left( 1+x^2 \right)^{k} \leq \left( {\sum_{r=0}^{k} x^r} \right)^2 \leq C_2 \left( 1+x^2 \right)^{k}. \]
Consider the function 
\[f (x) = \frac{\left( {\sum_{r=0}^{k} x^r} \right)^2}{\left( 1+x^2 \right)^{k}}\in C^0\left( [0,\infty ) \right).\]
Since $f(0) = 1$ and $\lim_{x\to \infty} f(x) = 1$, $f$ has a maximum on $[0,\infty)$ which gives $C_2$. Similarly, $g(x) = 1/f(x)$ has a maximum, giving $C_1$. 













\newpage 
\subsection*{Problem 7.2:}
Prove that if $f \in H_0^1\left( 0,1 \right)$, then there exists $C>0$ such that $\|f\|_{L^2(0,1)} \leq C \|f'\|_{L^2(0,1)}$. 
If instead $f \in \left\{ g \in H^1(0,1) : \int_0^1 g(x) dx = 0 \right\}$, provide a similar estimate. 

\paragraph*{Proof: }
Go by contradiction. Suppose to the contrary that there exists a sequence $f_n \in H_0^1$ with $\|f_n\|=1$ and $\|Df_n\|\to 0$. From every bounded sequence in a Hilbert space, we can extract a weakly convergent subsequence, $f_{n_k} \rightharpoonup f \in H_0^1$. The weak convergence in $H^1$ implies weak convergence of the derivative $Df_{n_k} \rightharpoonup Df \in L^2\left( \Omega \right)$. Since the norm is continuous, then we would conclude that $Df = 0$, that is, $f$ must be a constant. But in order to satisfy the boundary condition, then $f$ is indentically zero. 

On the other side, the Rellich-Kondrachov Theorem tells us that weak convergence in $H^1$ implies $f_{n_k} \to f \in L^2$, which implies convergence in the $L^2$ norm. But, that would mean that $\|f\| = \lim \|f_n\| = 1$, a contradiction to the above conclusion that $f = 0$. 













\newpage 
\subsection*{Problem 7.3:}
Prove that $\delta_0 \notin \left( H^1\left( \R^d \right) \right)^*$ for $d \geq 2$, but that $\delta_0 \in \left( H^1\left( \R \right) \right)^*$. You will need to define that $\delta_0$ applied to $f \in H^1\left( \R \right)$ means. 



\subsection*{Problem 7.4:}
Prove that $H^1(0,1)$ is continuously embedded in $C_B(0,1)$, the set of bounded and continuous functions on $(0,1)$. 

\paragraph*{Solution} Need to show that the inclusion map 
\[H^1\left( 0,1 \right) \ni f \to f \in C_B\left( 0,1 \right)\] 
is bounded, i.e. there exists $C>0$ such that $\|f\|_{\infty} \leq C \|f\|_{H^1}$.

\end{document}
 