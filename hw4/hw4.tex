\documentclass[letterpaper,twoside,11pt]{article}
\usepackage{a4wide,graphicx,fancyhdr,clrscode,tabularx,amsmath,amssymb,amsfonts,color,enumitem, bm, array, textcomp, subcaption, color, listings, chemformula, tcolorbox, setspace, xcolor}
\usepackage{amsthm}		%theorem style 
\usepackage{mathptmx}      %SET MATH TYPE FONT TO TIMES NEW ROMAN
%These lines make the theorem NAME BOLD
\newtheoremstyle{mystyle}%                % Name
  {}%                                     % Space above
  {}%                                     % Space below
  {\itshape}%                                     % Body font
  {}%                                     % Indent amount
  {\bfseries}%                            % Theorem head font
  {.}%                                    % Punctuation after theorem head
  { }%                                    % Space after theorem head, ' ', or \newline
  {\thmname{#1}\thmnumber{ #2}\thmnote{ (#3)}}%                                     % Theorem head spec (can be left empty, meaning `normal')
\theoremstyle{mystyle}
\newtheorem{theorem}{Theorem}[section]
%end of making theorem name bold. 
\newtheorem*{thm}{Theorem}		%Theorem no number. 
\newtheorem{definition}{Definition}[section]
\newtheorem{corollary}{Corollary}[theorem]
\newtheorem{lemma}[theorem]{Lemma}
\newtheorem{prop}{Proposition}[section]
\newtheorem*{ex}{Example}
\newtheorem{notee}{Note}[section]
\newtheorem*{note}{Note}
\usepackage[super]{nth}
\usepackage[makeroom]{cancel}

%----------------------- Macros and Definitions --------------------------
\setlength{\fboxsep}{2.5\fboxsep}
%Sets boxlength size

\setlength\headheight{15pt}
\addtolength\topmargin{-25pt}
\addtolength\footskip{0pt}

\fancypagestyle{plain}{%
\fancyhf{}
\fancyhead[LO,RE]{\sffamily UT Austin}
\fancyhead[RO,LE]{\sffamily CSE 386D}
\fancyfoot[LO,RE]{\sffamily Oden Institute}
\fancyfoot[RO,LE]{\sffamily\bfseries\thepage} 
\renewcommand{\headrulewidth}{0pt}
\renewcommand{\footrulewidth}{0pt}
}

\pagestyle{fancy}
\fancyhf{}
\fancyhead[RO,LE]{\sffamily CSE 386D}
\fancyhead[LO,RE]{\sffamily UT Austin}
\fancyfoot[LO,RE]{\sffamily Oden Institute}
\fancyfoot[RO,LE]{\sffamily\bfseries\thepage}
\renewcommand{\headrulewidth}{1pt}
\renewcommand{\footrulewidth}{0pt}
\newcommand{\R}{{\mathbb R}}
\newcommand{\N}{{\mathbb N}}
\newcommand{\Z}{{\mathbb Z}}
\newcommand{\Q}{{\mathbb Q}}
\newcommand{\C}{{\mathbb C}}
\newcommand{\SL}{{\mathcal{L}}}
%\usepackage{libertine}            %% For fancy font
\DeclareMathOperator*{\slim}{s-lim}
\newcommand{\cg}{\color{gray}}
\newcommand{\cbk}{\color{black}}
\newcommand{\cred}{\color{red}}
\newcommand{\cblu}{\color{blue}}
\newcommand{\inv}{^{-1}}
\newcommand{\sch}{\mathcal S} 




\begin{document}
%\fontfamily{ptm}\selectfont     %% TO SELECT THE FONT
\title{\vspace{-2\baselineskip} 
Homework 4
}
%\author{Jonathan Zhang \qquad EID: { jdz357} \qquad { jdz@utexas.edu}}
\author{Jonathan Zhang \qquad EID: { jdz357} }
\date{}
\maketitle


\subsection*{Problem 7.1:}
Prove that for $f \in H^1\left( \R^d \right)$, $\|f\|_{H^1\left( \R^d \right)}$ is equivalent to 
\[\left[\int\limits_{\R^d} \left( 1+|\xi|^2 \right)|\hat f\left( \xi \right)|^2d\xi \right]^{1/2}\]
Generalize to $H^k\left( \R^d \right)$?

\paragraph*{Solution}
Make use of Plancherel theorem. Let $f\in H^1\left( \R^d \right)$. Then $f$ and $Df$ are $L^2$. The norm is 
\[\|f\|_{H^1}^2 = \|f\|^2_{L^2} + \|Df\|^2_{L^2} = \|\hat f\|^2_{L^2} + \|\left( Df \right)^\land\|^2_{L^2} = \int |\hat f|^2 + |\left( Df \right)^\land|^2 = \int \left( 1+|\xi|^2 \right) |\hat f|^2.\]
The result also generalizes to $H^k$. In this case, propose a norm 
\[\|f\|_{H^k\left( \R^d \right)} = \left[\int\limits_{\R^d} \left( 1+|\xi|^2 \right)^k|\hat f\left( \xi \right)|^2d\xi \right]^{1/2}.\]
We need to show that there exist constants $C_1, C_2 >0$ that bound the two norms. But, this boils down to showing that there exists $C_1, C_2$ such that 
\[C_1 \left( 1+x^2 \right)^{k/2} \leq \sum_{r=0}^{k} x^r \leq C_2 \left( 1+x^2 \right)^{k/2}. \]
Equivalently, 
\[C_1 \left( 1+x^2 \right)^{k} \leq \left( {\sum_{r=0}^{k} x^r} \right)^2 \leq C_2 \left( 1+x^2 \right)^{k}. \]
Consider the function 
\[f (x) = \frac{\left( {\sum_{r=0}^{k} x^r} \right)^2}{\left( 1+x^2 \right)^{k}}\in C^0\left( [0,\infty ) \right).\]
Since $f(0) = 1$ and $\lim_{x\to \infty} f(x) = 1$, $f$ has a maximum on $[0,\infty)$ which gives $C_2$. Similarly, $g(x) = 1/f(x)$ has a maximum, giving $C_1$. 













\newpage 
\subsection*{Problem 7.2:}
Prove that if $f \in H_0^1\left( 0,1 \right)$, then there exists $C>0$ such that $\|f\|_{L^2(0,1)} \leq C \|f'\|_{L^2(0,1)}$. 
Similarly, if $f \in \left\{ g \in H^1(0,1) : \int_0^1 g(x) dx = 0 \right\}$?

\paragraph*{Proof: } If $f \in H_0^1\left( 0,1 \right)$, then write $f$ as $f\left( x \right) = \int\limits_0^x {f'\left( s \right)ds} .$
Then 
\[\left| {f\left( x \right)} \right| = \left| {\int\limits_0^x {f'\left( s \right)ds} } \right| \leqslant \int\limits_0^x {\left| {f'\left( s \right)} \right|ds} .\]
Taking supremum w.r.t. $x$, we have ${\left\| f \right\|_\infty } \leqslant \int\limits_0^1 {\left| {f'\left( s \right)} \right|ds} $. Squaring both sides, 
\[\left\| f \right\|_\infty ^2 \leqslant {\left\{ {\int\limits_0^1 {\left| {f'\left( s \right)} \right|ds} } \right\}^2} \leqslant \int\limits_0^1 {{{\left| {f'\left( s \right)} \right|}^2}ds}  = \left\| {f'} \right\|_{{L^2}\left( {0,1} \right)}^2.\]
Finally, 
\[\left\| f \right\|_2^2 = \int\limits_0^1 {{{\left| f \right|}^2}}  \leqslant \left\| f \right\|_\infty ^2\]
which proves the result for $C=1$. This merely a bound, not necessarily the tightest. 

For the case of zero mean, we go by contradiction. We suppose to the contrary that there exists a sequence of functions $f_k \in H^1\left( 0,1 \right)$ with zero mean such that 
\[{\left\| {{f_k}} \right\|_{{L^2}}} > k{\left\| {\nabla {f_k}} \right\|_{{L^2}}}.\] 
Without loss of generality, let us assume that $\|f_k\|_{L^2} = 1$. Therefore, \[\frac{1}{k} > {\left\| {\nabla {f_k}} \right\|_{{L^2}}} \to 0.\]
Now, from every bounded sequence in a Hilbert space, we can extract a weakly convergent subsequence, $f_{k_l} \rightharpoonup f$. Weak convergence in $H^1$ implies weak convergence of $\nabla f_k \rightharpoonup \nabla f$ in $L^2$. Passing to the limit, then $\nabla f = 0$, i.e. $f$ must be a constant. But since $f$ must have zero mean, then $f \equiv 0$. 
On the other side, the Rellich-Kondrachov Theorem tells us that weak convergence in $H^1$ implies strong convergence $f_{k_l} \to f$ in $L^2$, and therefore also in the norm. But convergence in the $L^2$ norm implies that $\|f\|_{L^2} = 1$, a contradiction. 




















\newpage 
\subsection*{Problem 7.3:}
Prove that $\delta_0 \notin \left( H^1\left( \R^d \right) \right)^*$ for $d \geq 2$, but that $\delta_0 \in \left( H^1\left( \R \right) \right)^*$. You will need to define that $\delta_0$ applied to $f \in H^1\left( \R \right)$ means. 
\paragraph*{Solution} 

For $f \in H^1\left( \R \right)$, define the action of $\delta_0$ as follows. First, identify a sequence $f_j \in \mathcal D$ converging to $f$. Then define the action of $\delta_0$ using its definition on Schwartz functions, i.e. $\left\langle {{\delta _0},f} \right\rangle  = \mathop {\lim }\limits_{j \to \infty } \left\langle {{\delta _0},{f_j}} \right\rangle $. Indeed, for every $f_j$, we have 
\[\left| {\left\langle {{\delta _0},{f_j}} \right\rangle } \right| = \left| {\left\langle {{\mathcal{F}^{ - 1}}\mathcal{F}{\delta _0},{f_j}} \right\rangle } \right| = \left| {\left\langle {\mathcal{F}{\delta _0},\mathcal{F}{f_j}} \right\rangle } \right| = \left| {\int\limits_{{\mathbb{R}^d}} {{{\left( {2\pi } \right)}^{ - d/2}}{{\hat f}_j}\left( \xi  \right)d\xi } } \right| = {\left( {2\pi } \right)^{ - d/2}}\left| {\int\limits_{{\mathbb{R}^d}} {{{\hat f}_j}\left( \xi  \right)d\xi } } \right|\]
It follows from Cauchy-Schwarz that 
\begin{align*}
  {\left( {2\pi } \right)^{ - 1/2}}\left| {\int\limits_{{\mathbb{R}}} {{{\hat f}_j}\left( \xi  \right)d\xi } } \right| &\leqslant {\left( {2\pi } \right)^{ - 1/2}}\int\limits_{{\mathbb{R}}} {\left| {{{\hat f}_j}\left( \xi  \right)} \right|\sqrt {\frac{{1 + {{\left| \xi  \right|}^2}}}{{1 + {{\left| \xi  \right|}^2}}}} d\xi }  \hfill \\
   &\leqslant {\left( {2\pi } \right)^{ - 1/2}}\left\{ {\int\limits_{{\mathbb{R}}} {\left( {1 + {{\left| \xi  \right|}^2}} \right){{\left| {{{\hat f}_j}\left( \xi  \right)} \right|}^2}d\xi } } \right\}^{1/2}{\left\{ {\int\limits_{{\mathbb{R}}} {\frac{1}{{1 + {{\left| \xi  \right|}^2}}}d\xi } } \right\}^{1/2}} .
\end{align*}
We showed earlier that the first integral is nothing more than $\|f_j\|_{H^1\left( \R \right)}$ which converges to $\|f\|_{H^1\left( \R \right)}$. By performing change of var. on the second integral, we see that 
\[\int\limits_\mathbb{R} {\frac{1}{{1 + {{\left| \xi  \right|}^2}}}d\xi } \sim d{\omega _d}\int\limits_\mathbb{R} {\frac{{{r^{d - 1}}}}{{1 + {r^2}}}dr} \]
which converges only when $1 - d + 2 > 1$, i.e. when $d <2$. 

To show that $\delta_0$ is not a member of the dual space for $d>1$, we compute its norm in the dual space. First, the dual of $H^1\left( \R^d \right)$ is $H^{-1}\left( \R^d \right)$. Then, we have the norm 

\[{\left\| {{\delta _0}} \right\|_{{H^{ - 1}}\left( {{\mathbb{R}^d}} \right)}} = {\left\{ {\int\limits_{{\mathbb{R}^d}} {{{\left( {1 + {{\left| \xi  \right|}^2}} \right)}^{ - 1}}{{\left| {{{\hat \delta }_0}} \right|}^2}} } \right\}^{1/2}} = {\left( {2\pi } \right)^{ - d/2}}{\left\{ {\int\limits_{{\mathbb{R}^d}} {\frac{1}{{1 + {{\left| \xi  \right|}^2}}}d\xi } } \right\}^{1/2}}.\]
Performing the same change of variables, we again see that the integral is similar to 
\[\int\limits_\mathbb{R^d} {\frac{1}{{1 + {{\left| \xi  \right|}^2}}}d\xi } \sim d{\omega _d}\int\limits_\mathbb{R} {\frac{{{r^{d - 1}}}}{{1 + {r^2}}}dr} \]
and also converges only when $d<2$. 



\newpage
\subsection*{Problem 7.4:}
Prove that $H^1(0,1)$ is continuously embedded in $C_B(0,1)$, the set of bounded and continuous functions on $(0,1)$. 

\paragraph*{Solution} Take a Cauchy sequence of smooth functions $f_j \in H^1\left( 0,1 \right) \cap C^\infty \left( [0,1] \right)$. Let now $ g := f_j - f_k$. Notice that by Cauchy-Schwarz
\[\left| {\int\limits_0^1 g } \right| \leqslant \sqrt {\int\limits_0^1 {{1^2}} } \sqrt {\int\limits_0^1 {{g^2}} }  = 1\left\| g \right\|_{{L^2}}^2 \leqslant \left\| g \right\|_{{H^1}}^2.\]
Furthermore, by the mean value theorem, $g$ attains its average, $m = \int\limits_0^1 g $ at some point $x_0$. Then 
\[\left| {g\left( x \right) - m} \right| \leqslant \sqrt {\left| {x - {x_0}} \right|} {\left\| g \right\|_{{H^1}}} \leqslant {\left\| g \right\|_{{H^1}}}.\]
Finally, triangle inequality tells us 
\[{\left\| g \right\|_\infty } \leqslant m + {\left\| g \right\|_{{H^1}}} \leqslant 2{\left\| g \right\|_{{H^1}}}.\]
So, if $f_j$ is Cauchy in $H^1\left( 0,1 \right) \cap C^\infty \left( [0,1] \right)$, then it is Cauchy in $C_B(0,1)$ and therefore converges. The inclusion map is therefore sequentially continuous and therefore continuous. 









\end{document}
 