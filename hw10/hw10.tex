\documentclass[letterpaper,twoside,11pt]{article}
\usepackage{a4wide,graphicx,fancyhdr,clrscode,tabularx,amsmath,amssymb,amsfonts,color,enumitem, bm, array, textcomp, subcaption, color, listings, chemformula, tcolorbox, setspace, xcolor}
\usepackage{amsthm}		%theorem style 
%\usepackage{mathptmx}      %SET MATH TYPE FONT TO TIMES NEW ROMAN
%These lines make the theorem NAME BOLD
\newtheoremstyle{mystyle}%                % Name
  {}%                                     % Space above
  {}%                                     % Space below
  {\itshape}%                                     % Body font
  {}%                                     % Indent amount
  {\bfseries}%                            % Theorem head font
  {.}%                                    % Punctuation after theorem head
  { }%                                    % Space after theorem head, ' ', or \newline
  {\thmname{#1}\thmnumber{ #2}\thmnote{ (#3)}}%                                     % Theorem head spec (can be left empty, meaning `normal')
\theoremstyle{mystyle}
\newtheorem{theorem}{Theorem}[section]
%end of making theorem name bold. 
\newtheorem*{thm}{Theorem}		%Theorem no number. 
\newtheorem{definition}{Definition}[section]
\newtheorem{corollary}{Corollary}[theorem]
\newtheorem{lemma}[theorem]{Lemma}
\newtheorem{prop}{Proposition}[section]
\newtheorem*{ex}{Example}
\newtheorem{notee}{Note}[section]
\newtheorem*{exercise}{Exercise}
\newtheorem*{note}{Note}
\usepackage[super]{nth}
\usepackage[makeroom]{cancel}

%----------------------- Macros and Definitions --------------------------
\setlength{\fboxsep}{2.5\fboxsep}
%Sets boxlength size

\setlength\headheight{15pt}
\addtolength\topmargin{-25pt}
\addtolength\footskip{0pt}

\fancypagestyle{plain}{%
\fancyhf{}
\fancyhead[LO,RE]{\sffamily UT Austin}
\fancyhead[RO,LE]{\sffamily CSE 386D}
\fancyfoot[LO,RE]{\sffamily Oden Institute}
\fancyfoot[RO,LE]{\sffamily\bfseries\thepage} 
\renewcommand{\headrulewidth}{0pt}
\renewcommand{\footrulewidth}{0pt}
}

\pagestyle{fancy}
\fancyhf{}
\fancyhead[RO,LE]{\sffamily CSE 386D}
\fancyhead[LO,RE]{\sffamily UT Austin}
\fancyfoot[LO,RE]{\sffamily Oden Institute}
\fancyfoot[RO,LE]{\sffamily\bfseries\thepage}
\renewcommand{\headrulewidth}{1pt}
\renewcommand{\footrulewidth}{0pt}
\newcommand{\R}{{\mathbb R}}
\newcommand{\N}{{\mathbb N}}
\newcommand{\Z}{{\mathbb Z}}
\newcommand{\Q}{{\mathbb Q}}
\newcommand{\C}{{\mathbb C}}
\newcommand{\SL}{{\mathcal{L}}}
%\usepackage{libertine}            %% For fancy font
\DeclareMathOperator*{\slim}{s-lim}
\newcommand{\cg}{\color{gray}}
\newcommand{\cbk}{\color{black}}
\newcommand{\cred}{\color{red}}
\newcommand{\cblu}{\color{blue}}
\newcommand{\inv}{^{-1}}
\newcommand{\ve}{\varepsilon}
\newcommand{\sch}{\mathcal S} 

\newcommand{\Hone}{H^1\left( \Omega \right)}
\newcommand{\Hdiv}{H\left( \text{div}, \Omega \right)}
\newcommand{\Hcur}{H\left( \text{curl}, \Omega \right)}
\newcommand{\Ltwo}{L^2 \left( \Omega \right)}




\begin{document}
\fontfamily{ptm}\selectfont     %% TO SELECT THE FONT
\title{\vspace{-2\baselineskip} 
Homework 10
}
%\author{Jonathan Zhang \qquad EID: { jdz357} \qquad { jdz@utexas.edu}}
\author{Jonathan Zhang \qquad EID: { jdz357} }
\date{}
\maketitle


\begin{exercise}[9.3]
  Let $X = C([0,1])$ be the space of bounded continuous functions on $[0,1]$ and, for $u \in X$, define $F(u)(x) = \displaystyle\int\limits_0^1 K(x, y) f(u(y))dy$ where $K:[0,1] \times [0,1] \to \R$ is continuous and $f$ is a $C^1$ mapping of $\R$ into $\R$. Find the Fr\'echet derivative $DF(u)$ of $F$ at $u\in X$. Is the map $u \mapsto DF(u)$ continuous?
\end{exercise}

\cblu 
$f\in C^1$ implies that $f$ has a Fr\'echet derivative, denoted $f'$. Moreover, for $u \in X$, $f\circ u \in X$ since composition with a continuous function remains continuous. Define another operator $G : X \to X$ as 
\[(Gv)(x) = \int\limits_0^1 K(x,y) v(y) dy.\]
Since $K$ is continuous, $G$ is well defined, and $G \in B(X,X)$. We know then that the $DG(v) = G$. Furthermore, notice that $(Fu)(x)$ is nothing more than $(G \circ f)(x)$. So, using the chain rule, 
\[DF\left( u \right) = D\left( {G\circ f} \right)\left( u \right) = DG\left( {f\left( u \right)} \right)f'\left( u \right) = G\left( {f\left( u \right)} \right)f'\left( u \right).\]
The mapping is continuous since $f'$ is continuous, multiplication is continuous, and $G$ is continuous. 
\cbk 



\begin{exercise}[9.5]
  Set up and apply contraction mapping principle to show that the problem 
  \[-u_{xx} + u - \varepsilon u^2 = f(x), \quad x\in \R\]
  has a smooth bounded solution if $\varepsilon>0$ is small enough, where $f(x) \in \sch (\R)$. 
\end{exercise}

\cblu 
Take the Foruier transform: 
\[ - {u_{xx}} + u = f + \varepsilon {u^2} = :g \longrightarrow \left( {1 + {\xi ^2}} \right)\hat u = \hat g\]
yields the solution 
\[\hat u = \frac{{\hat g}}{{1 + {\xi ^2}}} \longrightarrow u = g * \frac{1}{2}{e^{ - \left| x \right|}} = \left( {f + \varepsilon {u^2}} \right) * \frac{1}{2}{e^{ - \left| x \right|}} = \frac{1}{2}\int\limits_\mathbb{R} {\left( {f\left( y \right) + \varepsilon {u^2}\left( y \right)} \right){e^{ - \left| {x - y} \right|}}dy} .\]
It is natural then to regard the RHS as an operator acting on continuous function on some ball of radius $r$, say. 
Let $X = C\left( {{B_r}\left( 0 \right)} \right) \subset C\left( \mathbb{R} \right)$ equipped with max norm, and let 
\[\left( {Gu} \right)\left( x \right) = \frac{1}{2}\int\limits_\mathbb{R} {\left( {f\left( y \right) + \varepsilon {u^2}\left( y \right)} \right){e^{ - \left| {x - y} \right|}}dy} .\]
In order to apply contraction mapping, we need to show that $G$ sends $X$ back into itself, and that $G$ is a contraction. For $u \in X$, we have the estimate 
\begin{align*}
    {\left\| {Gu} \right\|_\infty } &= \mathop {\sup }\limits_x \left| {\frac{1}{2}\int\limits_\mathbb{R} {\left( {f\left( y \right) + \varepsilon {u^2}\left( y \right)} \right){e^{ - \left| {x - y} \right|}}dy} } \right| \hfill \\
     &\leqslant \left( {{{\left\| f \right\|}_\infty } + \varepsilon \left\| u \right\|_\infty ^2} \right)\left( {\frac{1}{2}\mathop {\sup }\limits_x \int\limits_\mathbb{R} {{e^{ - \left| {x - y} \right|}}dy} } \right) \hfill \\
     &\leqslant {\left\| f \right\|_\infty } + \varepsilon {r^2}.
\end{align*}
The condition implies 
\[{\left\| f \right\|_\infty } + \varepsilon {r^2} \leqslant r \Rightarrow \varepsilon  \leqslant \frac{{ - {{\left\| f \right\|}_\infty } + r}}{{{r^2}}}.\]
Notably, we need $r > \|f\|_{\infty}$. 
To show that $G$ is a contraction, 
\begin{align*}
    {\left\| {G{u_1} - G{u_2}} \right\|_\infty } &= {\left\| {\frac{\varepsilon }{2}\int\limits_\mathbb{R} {\left( {u_1^2\left( y \right) - u_2^2\left( y \right)} \right){e^{ - \left| {x - y} \right|}}dy} } \right\|_\infty } \hfill \\
     &\leqslant {\left\| {\frac{\varepsilon }{2}\int\limits_\mathbb{R} {\left( {{u_1}\left( y \right) + {u_2}\left( y \right)} \right){e^{ - \left| {x - y} \right|}}dy} } \right\|_\infty }{\left\| {{u_1} - {u_2}} \right\|_\infty } \hfill \\
     &\leqslant \varepsilon r\mathop {\sup }\limits_x \int\limits_\mathbb{R} {{e^{ - \left| {x - y} \right|}}dy} {\left\| {{u_1} - {u_2}} \right\|_\infty } \hfill \\
     &= 2\varepsilon r{\left\| {{u_1} - {u_2}} \right\|_\infty } 
\end{align*}
Now $G$ is a contraction if 
\[2\varepsilon r < 1 \Rightarrow \varepsilon  < \frac{1}{{2r}}.\]
So, for $\varepsilon < \min \left\{ (2r)\inv, r^{-2}({ - {{\left\| f \right\|}_\infty } + r}) \right\}$, the problem admits a smooth and bounded solution. 


\cbk 

\begin{exercise}[9.7]
  Suppose that $F$ is defined on a Banach space $X$, that $x_0 = F(x_0)$ is a fixed point of $F$, $DF(x_0)$ exists, and that 1 is not in the spectrum of $DF(x_0)$. Prove that $x_0$ is an isolated fixed point. 
\end{exercise}

\cblu 
  Suppose to the contrary that $x_0$ is not an isolated fixed point. That is, there exists another fixed point, say $x_1$, within an $\ve$ ball of $x_0$. We consider the difference 
  \[{x_1} - {x_0} = F\left( {{x_1}} \right) - F\left( {{x_0}} \right) = DF\left( {{x_0}} \right)\left( {{x_1} - {x_0}} \right) + R\left( {{x_0},{x_1} - {x_0}} \right)\]
  whihc we can write as 
  \[\left( {I - DF\left( {{x_0}} \right)} \right)\left( {{x_1} - {x_0}} \right) = R\left( {{x_0},{x_1} - {x_0}} \right).\]
  Since 1 is not in the spectrum of $DF(x_0)$, then the operator $I - DF(x_0)$ has a bounded inverse. Finally, 
  \[1 = \mathop {\lim }\limits_{{x_1} \to {x_0}} \frac{{\left\| {{x_1} - {x_0}} \right\|}}{{\left\| {{x_1} - {x_0}} \right\|}} = \mathop {\lim }\limits_{{x_1} \to {x_0}} \frac{{\left\| {{{\left[ {I - DF\left( {{x_0}} \right)} \right]}^{ - 1}}R\left( {{x_0},{x_1} - {x_0}} \right)} \right\|}}{{\left\| {{x_1} - {x_0}} \right\|}} \lesssim \mathop {\lim }\limits_{{x_1} \to {x_0}} \frac{{\left\| {R\left( {{x_0},{x_1} - {x_0}} \right)} \right\|}}{{\left\| {{x_1} - {x_0}} \right\|}} \to 0\]
  since $R(x_0, x_1-x_0)$ is $o\left( \ve \right)$. Contradiction. 

\cbk 

\begin{exercise}[9.8]
  Consider the ODE 
  \[u'(t) + u(t) = \cos(u(t))\]
  posed as an IVP for $t > 0$ with $u(0) = u_0$. 
  \begin{enumerate}
    \item Use contraction mapping theorem to show that there is exactly one solution $u$ corresponding to any given $u_0 \in \R$. 
    \item Prove that there is a number $\xi$ such that $\lim_{t\to \infty} u(t) = \xi$ for any solution $u$, independent of the value of $u_0$. 
  \end{enumerate}
\end{exercise}

\cblu 
\begin{enumerate}
  \item ODE to integral equation: 
  \[u'\left( t \right) = \cos \left( {u\left( t \right)} \right) - u\left( t \right) \longrightarrow u\left( t \right) = {u_0} + \int\limits_0^t {\left\{ {\cos \left( {u\left( t \right)} \right) - u\left( t \right)} \right\}dt} \]
  Natural to consider $X = C([0,T])$ for some $T > 0$ and the map $G$ defined by 
  \[\left( {Gu} \right)\left( t \right) = {u_0} + \int\limits_0^t {\left\{ {\cos \left( {u\left( t \right)} \right) - u\left( t \right)} \right\}dt} .\]
  We see that $u$ solves the ODE only if it is a solution to the fixed point problem $Gu = u$. We also see that $G : X \to X$. (If $u$ continuous, then fed through the continuous operations of cosine, integration, and addition produce another continuous function). Check that $G$ is a contraction: 
  \begin{align*}
      {\left\| {Gu - Gv} \right\|_\infty } &= {\left\| {\int\limits_0^t {\left\{ {\cos \left( {u\left( t \right)} \right) - u\left( t \right)} \right\}dt}  - \left\{ {\int\limits_0^t {\left\{ {\cos \left( {v\left( t \right)} \right) - v\left( t \right)} \right\}dt} } \right\}} \right\|_\infty } \hfill \\
       &= {\left\| {\int\limits_0^t {\left( {\cos u - \cos v} \right)dt}  + \int\limits_0^t {\left( {v - u} \right)dt} } \right\|_\infty } \hfill \\
       &\leqslant {\left\| {\int\limits_0^t {\left( {\cos u - \cos v} \right)dt} } \right\|_\infty } + {\left\| {\int\limits_0^t {\left( {v - u} \right)dt} } \right\|_\infty } \hfill \\
       &\leqslant \mathop {\sup }\limits_{0 \leqslant t \leqslant T} \int\limits_0^t {\left| {\cos u - \cos v} \right|dt}  + T{\left\| {u - v} \right\|_\infty } \hfill \\
       &\leqslant \mathop {\sup }\limits_{0 \leqslant t \leqslant T} \int\limits_0^t {\left| { - \sin \left( {w\left( t \right)} \right)\left( {u\left( t \right) - v\left( t \right)} \right)} \right|dt}  + T{\left\| {u - v} \right\|_\infty } \hfill \\
       &\leqslant 2T{\left\| {u - v} \right\|_\infty }
  \end{align*}
  For $2T<1 \Rightarrow T < 1/2$, $G$ is a contraction, and we have a unique solution. In fact, the solution exists for all $t>0$. Solve the initial equation up to time $T = 1/3$, say. Then repeat the process with a new initial condition given by the value of $u$ at $t = 1/3$. 
  \item Let $\xi \in \R$ be the fixed point of cosine, that is $\xi = \cos\xi$. Indeed, $\xi$ is unique. One can just plot the two functions and observe this, or argue with another contraction map argument on the interval $[-\pi/2, \pi/2]$, say. We now have the following situation: 
  \begin{enumerate}
    \item If $u_0 = \xi$, then $u'(0) = \cos(u(0)) - u_0 = \xi - \xi = 0$, so $u = \xi$ is a constant. 
    \item If $u_0 < \xi$, then $u'\left( 0 \right) = \cos \left( {{u_0}} \right) - {u_0} > 0$, so $u$ is increasing. Moreover, $u'(t)$ remains positive as long as $u(t) < \xi$. However, since $\sin x - 1 \leq 0$ for all $x$, $u'(t)$ is increasing at a decreasing rate, and so $u'(t) \searrow 0$. 
    \item If $u_0 > \xi$, then $u'\left( 0 \right) = \cos \left( {{u_0}} \right) - {u_0} < 0$, so $u$ is decreasing. Moreover, $u'(t)$ remains negative as long as $u(t) > \xi$. However, since $\sin x - 1 \leq 0$ for all $x$, $u'(t)$ is decreasing at a decreasing rate, and so $u'(t) \nearrow 0$. 
  \end{enumerate}
\end{enumerate}

\cbk 

\begin{exercise}[9.10]
  Consider the PDE 
  \[u_{t} - u_{xxt} - \varepsilon u^3 = f, \quad x\in \R, t>0 \]
  with $u(x,0) = g(x)$. Use the Fourier transform and a contraction mapping argument to show that there exists a solution for small enough $\varepsilon$, at least up to some time $T< \infty$. In what spaces should $f$ and $g$ lie? 
\end{exercise}

\cblu 
Fourier transform in $x$ since we are on the whole real line: 
\[{u_t} - {u_{xxt}} = f + \varepsilon {u^3} = :h \to {\hat u_t} + {\xi ^2}{\hat u_t} = \hat h \to {\hat u_t} = \frac{{\hat h}}{{1 + {\xi ^2}}}\] 
which, after taking inverse (similar to Ex 9.5), yields 
\[{u_t}\left( {x,t} \right) = \frac{1}{2}\int\limits_\mathbb{R} {\left( {f\left( {y,t} \right) + \varepsilon {u^3}\left( {y,t} \right)} \right){e^{ - \left| {x - y} \right|}}dy} .\]
Integrating in time, 
\[u\left( {x,t} \right) = g\left( x \right) + \frac{1}{2}\int\limits_0^t {\int\limits_\mathbb{R} {\left( {f\left( {y,\tau} \right) + \varepsilon {u^3}\left( {y,\tau} \right)} \right){e^{ - \left| {x - y} \right|}}dy} d\tau} .\]
Let $X = C(B_r(0))\subset C\left( \R\times  (0,T) \right)$ equipped with the max norm. Define the map \(G\) by 
\[\left( {Gu} \right)\left( {x,t} \right) = g\left( x \right) + \frac{1}{2}\int\limits_0^t {\int\limits_\mathbb{R} {\left( {f\left( {y,\tau } \right) + \varepsilon {u^3}\left( {y,\tau } \right)} \right){e^{ - \left| {x - y} \right|}}dy} d\tau } .\]
We want to show that $G$ takes $X$ into $X$ and that $G$ is a contraction. For now, take $f, g \in L^{\infty}(\Omega)$. We have the estimate 
\begin{align*}
    {\left\| {Gu} \right\|_\infty } &= {\left\| {g\left( x \right) + \frac{1}{2}\int\limits_0^t {\int\limits_\mathbb{R} {\left( {f\left( {y,\tau } \right) + \varepsilon {u^3}\left( {y,\tau } \right)} \right){e^{ - \left| {x - y} \right|}}dy} d\tau } } \right\|_\infty } \hfill \\
     &\leqslant {\left\| g \right\|_\infty } + \left( {{{\left\| f \right\|}_\infty } + \varepsilon {r^3}} \right){\left\| {\frac{1}{2}\int\limits_0^t {\int\limits_\mathbb{R} {{e^{ - \left| {x - y} \right|}}dy} d\tau } } \right\|_\infty } \hfill \\
     &\leqslant {\left\| g \right\|_\infty } + \left( {{{\left\| f \right\|}_\infty } + \varepsilon {r^3}} \right)T
\end{align*}
For $\|Gu\| \leq r$, we have 
\[{\left\| g \right\|_\infty } + \left( {{{\left\| f \right\|}_\infty } + \varepsilon {r^3}} \right)T \leqslant r \Rightarrow \varepsilon  \leqslant {r^{ - 3}}\left\{ {\frac{{r - {{\left\| g \right\|}_\infty }}}{T} - {{\left\| f \right\|}_\infty }} \right\}\]
and implicitly, 
\[r > {\left\| g \right\|_\infty },\qquad T < \frac{{r - {{\left\| g \right\|}_\infty }}}{{{{\left\| f \right\|}_\infty }}}.\]
For $G$ to be a contraction, we estimate 
\begin{align*}
    \left\| {G{u_1} - G{u_2}} \right\| &= \left\| {\frac{\varepsilon }{2}\int\limits_0^t {\int\limits_\mathbb{R} {\left( {u_1^3\left( {y,\tau } \right) - u_2^3\left( {y,\tau } \right)} \right){e^{ - \left| {x - y} \right|}}dy} d\tau } } \right\| \hfill \\
     &= \left\| {\frac{\varepsilon }{2}\int\limits_0^t {\int\limits_\mathbb{R} {\left( {{u_1} - {u_2}} \right)\left( {u_1^2 + {u_1}{u_2} + u_2^2} \right)\left( {y,\tau } \right){e^{ - \left| {x - y} \right|}}dy} d\tau } } \right\| \hfill \\
     &\leqslant \varepsilon 3{r^2}T\left\| {{u_1} - {u_2}} \right\|
\end{align*}
So, $G$ is a contraction as long as $3\varepsilon {r^2}T < 1$.

In summary, if $f, g \in L^{\infty}(\Omega)$, $r > {\left\| g \right\|_\infty }$, $T < \frac{{r - {{\left\| g \right\|}_\infty }}}{{{{\left\| f \right\|}_\infty }}}$, $\varepsilon  < \min \left\{ {{r^{ - 3}}\left\{ {\frac{{r - {{\left\| g \right\|}_\infty }}}{T} - {{\left\| f \right\|}_\infty }} \right\},\frac{1}{{3{r^2}T}}} \right\}$, then the problem admits a unique solution $u \in B_r(0)$. 



\cbk 

\end{document}
 
