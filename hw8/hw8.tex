\documentclass[letterpaper,twoside,11pt]{article}
\usepackage{a4wide,graphicx,fancyhdr,clrscode,tabularx,amsmath,amssymb,amsfonts,color,enumitem, bm, array, textcomp, subcaption, color, listings, chemformula, tcolorbox, setspace, xcolor}
\usepackage{amsthm}		%theorem style 
%\usepackage{mathptmx}      %SET MATH TYPE FONT TO TIMES NEW ROMAN
%These lines make the theorem NAME BOLD
\newtheoremstyle{mystyle}%                % Name
  {}%                                     % Space above
  {}%                                     % Space below
  {\itshape}%                                     % Body font
  {}%                                     % Indent amount
  {\bfseries}%                            % Theorem head font
  {.}%                                    % Punctuation after theorem head
  { }%                                    % Space after theorem head, ' ', or \newline
  {\thmname{#1}\thmnumber{ #2}\thmnote{ (#3)}}%                                     % Theorem head spec (can be left empty, meaning `normal')
\theoremstyle{mystyle}
\newtheorem{theorem}{Theorem}[section]
%end of making theorem name bold. 
\newtheorem*{thm}{Theorem}		%Theorem no number. 
\newtheorem{definition}{Definition}[section]
\newtheorem{corollary}{Corollary}[theorem]
\newtheorem{lemma}[theorem]{Lemma}
\newtheorem{prop}{Proposition}[section]
\newtheorem*{ex}{Example}
\newtheorem{notee}{Note}[section]
\newtheorem*{exercise}{Exercise}
\newtheorem*{note}{Note}
\usepackage[super]{nth}
\usepackage[makeroom]{cancel}

%----------------------- Macros and Definitions --------------------------
\setlength{\fboxsep}{2.5\fboxsep}
%Sets boxlength size

\setlength\headheight{15pt}
\addtolength\topmargin{-25pt}
\addtolength\footskip{0pt}

\fancypagestyle{plain}{%
\fancyhf{}
\fancyhead[LO,RE]{\sffamily UT Austin}
\fancyhead[RO,LE]{\sffamily CSE 386D}
\fancyfoot[LO,RE]{\sffamily Oden Institute}
\fancyfoot[RO,LE]{\sffamily\bfseries\thepage} 
\renewcommand{\headrulewidth}{0pt}
\renewcommand{\footrulewidth}{0pt}
}

\pagestyle{fancy}
\fancyhf{}
\fancyhead[RO,LE]{\sffamily CSE 386D}
\fancyhead[LO,RE]{\sffamily UT Austin}
\fancyfoot[LO,RE]{\sffamily Oden Institute}
\fancyfoot[RO,LE]{\sffamily\bfseries\thepage}
\renewcommand{\headrulewidth}{1pt}
\renewcommand{\footrulewidth}{0pt}
\newcommand{\R}{{\mathbb R}}
\newcommand{\N}{{\mathbb N}}
\newcommand{\Z}{{\mathbb Z}}
\newcommand{\Q}{{\mathbb Q}}
\newcommand{\C}{{\mathbb C}}
\newcommand{\SL}{{\mathcal{L}}}
\usepackage{libertine}            %% For fancy font
\DeclareMathOperator*{\slim}{s-lim}
\newcommand{\cg}{\color{gray}}
\newcommand{\cbk}{\color{black}}
\newcommand{\cred}{\color{red}}
\newcommand{\cblu}{\color{blue}}
\newcommand{\inv}{^{-1}}
\newcommand{\sch}{\mathcal S} 




\begin{document}
%\fontfamily{ptm}\selectfont     %% TO SELECT THE FONT
\title{\vspace{-2\baselineskip} 
Homework 8
}
%\author{Jonathan Zhang \qquad EID: { jdz357} \qquad { jdz@utexas.edu}}
\author{Jonathan Zhang \qquad EID: { jdz357} }
\date{}
\maketitle

Unless otherwise stated, denote by $\left( \cdot, \cdot \right)$ as the $L^2$ inner product, and $\|\cdot\|$ as $\|\cdot \|_{L^2\left( \Omega \right)}$. 

\begin{exercise}[2]
  Suppose that the hypotheses of the Generalized Lax-Milgram theorem are satisfied. Suppose that $x_{0,1}$ and $x_{0,2}$ in $\mathcal X$ are such that the sets $X + x_{0,1} = X + x_{0,2}$. Prove that the solutions $u_1 \in X + x_{0,1}$ and $u_2 \in X + x_{0,2}$ of the abstract variational problem agree. What does this say about Dirichlet BVP? 
\end{exercise}

Assume that both $u_1, u_2$ satisfy the variational problem $B(u, v) = F(v)$ for every $v \in Y$. Then, $B(u_2 - u_1 , v) = 0$ for every $v \in Y$. At the same time, $B$ satisfies the inf-sup condition, 
\[\mathop {\sup }\limits_{\left\| v \right\| = 1} B\left( {{u_2} - {u_1},v} \right) \geqslant \gamma \left\| {{u_2} - {u_1}} \right\|.\]
Thus, $\|u_2-u_1\| = 0$ and we conclude that the solutions are the same. 
The result shows that the solution to Dirichlet BVP's is independent of how the boundary data is extended to the whole space. 

\begin{exercise}[4]
BVP for $u (x,y) : \R^2 \to \R$. Write as a variational problem and show there exists a unique solution. Carefully define function spaces and identify where $f$ must lie. 
\[\left\{ {\begin{array}{*{20}{c}}
  { - {u_{xx}} + {e^y}u = f}&{\left( {x,y} \right) \in {{\left( {0,1} \right)}^2}} \\[.2cm] 
  {u\left( {0,y} \right) = 0,\,\,u\left( {1,y} \right) = \cos y}&{y \in \left( {0,1} \right)} 
\end{array}} \right.\]
\end{exercise}

\cred Treat as a problem for fixed $y$? Where does $f$ come from? \cbk 


\begin{exercise}[9]
Let $\Omega\subset \R^d$ be a bounded domain with Lipschitz boundary, $f \in L^2 \left( \Omega \right)$, $\alpha > 0$. Consider the Robin problem
\[\left\{ {\begin{array}{*{20}{c}}
  { -\Delta u + u = f}&{x\in \Omega} \\[.2cm] 
  {\displaystyle \frac{\partial u}{\partial n }+ \alpha u = 0}&{x \in \partial \Omega} 
\end{array}} \right.\]
\begin{enumerate}
  \item Formulate a variational principle $B(u,v) = (f, v)$ for $v \in H^1 \left( \Omega \right)$. 
  \item Show that this problem has a unique weak solution. 
\end{enumerate}
\end{exercise}

\begin{proof}
  \begin{enumerate}
    \item Variational Problem. Multiply by a test function $v$ and integrate over $\Omega$:
    \[\left( { - {\nabla ^2}u,v} \right) + \left( {u,v} \right) = \left( {f,v} \right).\]
    Integrating the first term by parts, 
    \[ - \left( {{\nabla ^2}u,v} \right) = \left( {\nabla u,\nabla v} \right) - {\left\langle {\nabla u \cdot n,v} \right\rangle _{\partial \Omega }} = \left( {\nabla u,\nabla v} \right) + {\left\langle {\alpha u,v} \right\rangle _{\partial \Omega }}\]
    we arrive at 
    \[\left( {\nabla u,\nabla v} \right) + {\left\langle {\alpha u,v} \right\rangle _{\partial \Omega }} + \left( {u,v} \right) = \left( {f,v} \right).\]
    The VP reads: 
    \[\left\{ {\begin{array}{*{20}{l}}
      {u \in {H^1}\left( \Omega  \right)} \\[.2cm] 
      {\left( {\nabla u,\nabla v} \right) + {{\left\langle {\alpha u,v} \right\rangle }_{\partial \Omega }} + \left( {u,v} \right) = B\left( {u,v} \right) = F\left( v \right) = \left( {f,v} \right)} \\[.2cm]
      \text{for every } v\in H^1 (\Omega).  
    \end{array}} \right.\]
    \item Coercivity:  
    \[B\left( {v,v} \right) = {\left\| {\nabla v} \right\|^2} + {\left\langle {\alpha v,v} \right\rangle _{\partial \Omega }} + {\left\| v \right\|^2} \geqslant {\left\| {\nabla v} \right\|^2} + {\left\| v \right\|^2} = \left\| v \right\|_{{H^1}\left( \Omega  \right)}^2.\]


    Continuity: 
    \begin{align*}
        \left| {B\left( {u,v} \right)} \right| &= \left| {\left( {\nabla u,\nabla v} \right) + {{\left\langle {\alpha u,v} \right\rangle }_{\partial \Omega }} + \left( {u,v} \right)} \right| \hfill \\[.2cm]
         &\leqslant \left\| {\nabla u} \right\|\left\| {\nabla v} \right\| + \left\| u \right\|\left\| v \right\| + \alpha {\left\| {{\gamma _0}u} \right\|_{{L^2}\left( {\partial \Omega } \right)}}{\left\| {{\gamma _0}v} \right\|_{{L^2}\left( {\partial \Omega } \right)}} \hfill \\[.2cm]
         &\leqslant C{\left\| u \right\|_{{H^1}\left( \Omega  \right)}}{\left\| v \right\|_{{H^1}\left( \Omega  \right)}}  
    \end{align*}
    Therefore, by Lax-Milgram, the VP admits a unique solution.

  \end{enumerate}
\end{proof}



\begin{exercise}[13]
Suppose $\Omega\subset \R^d$ is a bounded Lipschitz domain. Consider the Stokes problem for vector $u$ and scalar $p$: 

\[\left\{ 
\begin{array}{*{20}{rl}}
- \Delta u + \nabla p = f  & x \in \Omega \\[.2cm] 
\nabla \cdot u = 0 & x \in \Omega \\[.2cm] 
u = 0 & x \in \partial \Omega  
\end{array}
\right. \]
where the first equation holds for each coordinate. This problem is a saddle-point problem in which we minimize some energy subject to the constraint $\nabla \cdot u = 0$. However, if we work over the constrained space, we can handle this problem. Let 
\[H = \left\{ v \in \left( H_0^1 (\Omega) \right)^d  : \nabla \cdot v = 0\right\}\]
\begin{enumerate}
  \item Show that $H$ is a Hilbert space. 
  \item Determine an appropriate Sobolev space for $f$ and formulate an appropriate VP for constrained Stokes. 
  \item Show that there is a unique solution to the VP. 
\end{enumerate}
\end{exercise}

\begin{proof}
  \begin{enumerate}
    \item $H$ is clearly a linear subspace of Hilbert Space $\left( H_0^1 (\Omega) \right)^d$. What remains is to show that $H$ is closed. Let $u_n \in H$ be a Cauchy sequence. (That is, $\nabla \cdot u_n = 0$.) Assume that $u_n \to u $ in the $H^1$ sense. Then $\nabla u_n \to \nabla u$ in $(L^2(\Omega))^{d\times d}$, and hence each element $\left( \nabla u_n \right)_{ij} \to \left( \nabla u \right)_{ij}$ in $L^2\left( \Omega \right)$. 
  \end{enumerate}
\end{proof}


\begin{exercise}[14]
Let $\mathcal H = H_0^1 \left( \Omega \right) \times H^1 \left( \Omega \right)$ and consider the solution $\left( u,v \right)\in \mathcal H$ to the differential problem  
\[\left\{ {\begin{array}{*{20}{c}}
  - \Delta u = f + a(x)\left( v-u \right) & x \in \Omega \\[.2cm] 
  v - \Delta v = g + a(x)(u-v) & x \in \Omega \\[.2cm] 
  u = 0, \nabla v \cdot n = \gamma & x \in \partial \Omega
\end{array}} \right.\]
where $a \in L^\infty \left( \Omega \right)$. 
\begin{enumerate}
  \item Develop an appropriate weak or variational form for the problem. In what Sobolev spaces should $f, g, \gamma$ lie? 
  \item Prove that there exists a unique solution provided that $a \geq 0$. 
\end{enumerate}
\end{exercise}



\end{document}
 