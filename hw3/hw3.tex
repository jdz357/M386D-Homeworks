\documentclass[letterpaper,twoside,11pt]{article}
\usepackage{a4wide,graphicx,fancyhdr,clrscode,tabularx,amsmath,amssymb,amsfonts,color,enumitem, bm, array, textcomp, subcaption, color, listings, chemformula, tcolorbox, setspace, xcolor}
\usepackage{amsthm}		%theorem style 
%\usepackage{mathptmx}      %SET MATH TYPE FONT TO TIMES NEW ROMAN
%These lines make the theorem NAME BOLD
\newtheoremstyle{mystyle}%                % Name
  {}%                                     % Space above
  {}%                                     % Space below
  {\itshape}%                                     % Body font
  {}%                                     % Indent amount
  {\bfseries}%                            % Theorem head font
  {.}%                                    % Punctuation after theorem head
  { }%                                    % Space after theorem head, ' ', or \newline
  {\thmname{#1}\thmnumber{ #2}\thmnote{ (#3)}}%                                     % Theorem head spec (can be left empty, meaning `normal')
\theoremstyle{mystyle}
\newtheorem{theorem}{Theorem}[section]
%end of making theorem name bold. 
\newtheorem*{thm}{Theorem}		%Theorem no number. 
\newtheorem{definition}{Definition}[section]
\newtheorem{corollary}{Corollary}[theorem]
\newtheorem{lemma}[theorem]{Lemma}
\newtheorem{prop}{Proposition}[section]
\newtheorem*{ex}{Example}
\newtheorem{notee}{Note}[section]
\newtheorem*{note}{Note}
\usepackage[super]{nth}
\usepackage[makeroom]{cancel}

%----------------------- Macros and Definitions --------------------------
\setlength{\fboxsep}{2.5\fboxsep}
%Sets boxlength size

\setlength\headheight{15pt}
\addtolength\topmargin{-25pt}
\addtolength\footskip{0pt}

\fancypagestyle{plain}{%
\fancyhf{}
\fancyhead[LO,RE]{\sffamily UT Austin}
\fancyhead[RO,LE]{\sffamily CSE 386D}
\fancyfoot[LO,RE]{\sffamily Oden Institute}
\fancyfoot[RO,LE]{\sffamily\bfseries\thepage} 
\renewcommand{\headrulewidth}{0pt}
\renewcommand{\footrulewidth}{0pt}
}

\pagestyle{fancy}
\fancyhf{}
\fancyhead[RO,LE]{\sffamily CSE 386D}
\fancyhead[LO,RE]{\sffamily UT Austin}
\fancyfoot[LO,RE]{\sffamily Oden Institute}
\fancyfoot[RO,LE]{\sffamily\bfseries\thepage}
\renewcommand{\headrulewidth}{1pt}
\renewcommand{\footrulewidth}{0pt}
\newcommand{\R}{{\mathbb R}}
\newcommand{\N}{{\mathbb N}}
\newcommand{\Z}{{\mathbb Z}}
\newcommand{\Q}{{\mathbb Q}}
\newcommand{\C}{{\mathbb C}}
\newcommand{\SL}{{\mathcal{L}}}
\usepackage{libertine}            %% For fancy font
\DeclareMathOperator*{\slim}{s-lim}
\newcommand{\cg}{\color{gray}}
\newcommand{\cbk}{\color{black}}
\newcommand{\cred}{\color{red}}
\newcommand{\cblu}{\color{blue}}
\newcommand{\inv}{^{-1}}
\newcommand{\sch}{\mathcal S} 




\begin{document}
%\fontfamily{ptm}\selectfont     %% TO SELECT THE FONT
\title{\vspace{-2\baselineskip} 
Homework 3
}
%\author{Jonathan Zhang \qquad EID: { jdz357} \qquad { jdz@utexas.edu}}
\author{Jonathan Zhang \qquad EID: { jdz357} }
\date{}
\maketitle

\cred 
\subsection*{Problem 6.18:}
 For $f\in \mathcal S (\R)$, define the Hilbert transform of $f$ by $Hf = PV\left( \frac{1}{\pi x} \right) \ast f$.
\begin{enumerate}
  \item Show that $PV(1/x)\in \sch '$.
  \item Show that $\mathcal F \left( PV\left( 1/x \right) \right) = -i\sqrt{\pi/2} sgn\left( \xi \right)$. 
  \item Show that $\|Hf\|_{L^2} = \|f\|_{L^2}$ and $HHf = -f$ for $f \in \sch \left( \R \right)$.  
  \item Extend $H$ to $L^2\left( \R \right)$. 
\end{enumerate}

\paragraph*{Solution}
\begin{enumerate}
  \item asdf
  \item Recall $xPV\left( 1/x \right)=1$. Taking Fourier transform on both sides yields 
  \[\mathcal F \left( xPV\left( 1/x \right) \right) = \]
\end{enumerate}










\cbk 


\subsection*{Problem 6.20:}
Compute the Fourier transforms of the following functions considered as tempered distributions. 
\begin{enumerate}
  \item $x^n$ for $x \in \R$ and integer $n\geq 0$
  \item $e^{-|x|}$ for $x\in \R$
  \item $e^{i|x|^2}$ for $x \in \R^d$
  \item $\sin x$ and $\cos x$ for $x\in \R$. 
\end{enumerate}
\paragraph*{Solution}
\begin{enumerate}
  \item \begin{align*}
    \left\langle {{{\hat x}^n},\phi } \right\rangle  &= \left\langle {{x^n},\hat \phi } \right\rangle  \hfill \\
     &= \left\langle {1,{x^n}\hat \phi } \right\rangle  \hfill \\
     &= {\left( i \right)^{ - n}}\left\langle {1,{{\left( {ix} \right)}^n}\hat \phi } \right\rangle  \hfill \\
     &= {\left( i \right)^{ - n}}\left\langle {1,{{\left( {{D^n}\phi } \right)}^ \wedge }} \right\rangle  \hfill \\
     &= {\left( i \right)^{ - n}}\left\langle {\hat 1,\left( {{D^n}\phi } \right)} \right\rangle  \hfill \\
     &= {\left( i \right)^{ - n}}\left\langle {{{\left( {2\pi } \right)}^{1/2}}{\delta _0},\left( {{D^n}\phi } \right)} \right\rangle  \hfill \\
     &= {\left( {2\pi } \right)^{1/2}}{\left( i \right)^{ - n}}\left\langle {{{\left( { - 1} \right)}^n}{D^n}{\delta _0},\phi } \right\rangle  \hfill \\
     &= \left\langle {{{\left( {2\pi } \right)}^{1/2}}{{\left( i \right)}^n}{D^n}{\delta _0},\phi } \right\rangle  \hfill \\ 
  \end{align*}
  \item In this case, $e^{-|x|}$ is actually an $L^1$ function. Therefore, we can equivalently just compute its Fourier transform in the usual way. 
  \begin{align*}
    \mathcal{F}\left( {{e^{ - |x|}}} \right) &= {\left( {2\pi } \right)^{ - 1/2}}\int\limits_\mathbb{R} {{e^{ - \left| x \right| - ix\xi }}dx}  \hfill \\
     &= {\left( {2\pi } \right)^{ - 1/2}}\left[ {\int\limits_{ - \infty }^0 {{e^{x - ix\xi }}dx}  + \int\limits_0^\infty  {{e^{ - x - ix\xi }}dx} } \right] \hfill \\
     &= {\left( {2\pi } \right)^{ - 1/2}}\left[ {\frac{1}{{1 - i\xi }} + \frac{1}{{1 + i\xi }}} \right] \hfill \\
     &= {\left( {2\pi } \right)^{ - 1/2}}\frac{2}{{1 + {\xi ^2}}} 
  \end{align*}
  so 
  \[\left\langle {{{\left( {{e^{ - |x|}}} \right)}^ \wedge },\phi } \right\rangle  = \left\langle {{{\left( {2\pi } \right)}^{ - 1/2}}\frac{2}{{1 + {\xi ^2}}},\phi } \right\rangle .\]
  \item \cred asdf asdfj weird pie slice integral thing \cbk 
  \item Follows from the Fourier transform of $x^n$ by linearity. 
  \[\cos x = \sum\limits_{n = 0}^\infty  {\frac{{{{\left( { - 1} \right)}^n}{x^{2n}}}}{{\left( {2n} \right)!}}}  \Rightarrow {\left( {\cos x} \right)^ \wedge } = \sum\limits_{n = 0}^\infty  {\frac{{{{\left( { - 1} \right)}^n}{{\left( {{x^{2n}}} \right)}^ \wedge }}}{{\left( {2n} \right)!}}}  = \sum\limits_{n = 0}^\infty  {\frac{{{{\left( { - 1} \right)}^n}}}{{\left( {2n} \right)!}}{{\left( {2\pi } \right)}^{1/2}}{i^{2n}}{D^{2n}}{\delta _0}} \]
  and
  \[\sin x = \sum\limits_{n = 0}^\infty  {\frac{{{{\left( { - 1} \right)}^n}{x^{2n + 1}}}}{{\left( {2n + 1} \right)!}}}  \Rightarrow {\left( {\sin x} \right)^ \wedge }\sum\limits_{n = 0}^\infty  {\frac{{{{\left( { - 1} \right)}^n}{{\left( {{x^{2n + 1}}} \right)}^ \wedge }}}{{\left( {2n + 1} \right)!}}}  = \sum\limits_{n = 0}^\infty  {\frac{{{{\left( { - 1} \right)}^n}}}{{\left( {2n + 1} \right)!}}{{\left( {2\pi } \right)}^{1/2}}{i^{2n + 1}}{D^{2n + 1}}{\delta _0}} \]
\end{enumerate}

























\cbk 
\subsection*{Problem 6.23:}
Define the space $H$, endowed with inner product $\left( f,g \right)_H$. 
\begin{enumerate}
  \item Show that $H$ is complete. 
  \item Prove that $H$ is continuously imbedded in $L^2 \left( \R \right)$. 
  \item If $m(\xi) \geq \alpha |\xi|^2$ for some $\alpha >0$, prove that for $f \in H$, the tempered distributional derivative $f'\in L^2(\R)$. 
\end{enumerate}

\paragraph*{Solution} Notice first that the norm on $H$ is merely a weighted $L^2$ norm on Fourier space. 
\begin{enumerate}
  \item Let $f_n \in H$ be a Cauchy sequence. Since the Fourier transform preserves norm, then $\hat f_n$ is also a Cauchy sequence in $L^2_m (\R)$. Since $m$ is a non-negative measure, $L^2_m$ is complete, so the sequence posesses a limit $\hat f_n \to g \in L^2_m$. Now, $g$, as the limit of measurable functions $\hat f_n$, is also measurable. Furthermore, $g \in L^2$ since $||g||_{L^2_m} \geq m^* \|g\|_{L^2}$. If we define now $f := \mathcal F \inv (g)$ using the $L^2$ Fourier inverse, we uncover the remaining properties we need. As the limit of measurable functions $f_n$, $f$ is measurable. Additionally, $f \in L^2$ and so it immediately generates a tempered distribution, so $f \in \sch ' (\R)$. Finally, $\|f\|_H = \|\mathcal F \inv (g) \|_H = \|g\|_{L^2} < \infty$. 
  \item The inclusion map 
  \[H \ni f \to i\left( f \right) = f \in {L^2}\]
  is bounded, since 
  \[{\left\| f \right\|_H} = {\left\| {\hat f} \right\|_{L_m^2}} \geqslant {m^*}{\left\| {\hat f} \right\|_{{L^2}}} = {m^*}{\left\| f \right\|_{{L^2}}}\]
  i.e. the continuity constant is $1/m^*$. 
  \item We have 
  \begin{align*}
    \left\| f \right\|_H^2 &= \int {{{\left| {\hat f\left( \xi  \right)} \right|}^2}m\left( \xi  \right)d\xi }  \hfill \\
     &\geqslant \alpha \int {{{\left| {\xi \hat f\left( \xi  \right)} \right|}^2}d\xi }  \hfill \\
     &= \alpha \int {{{\left| {{{\left( {Df} \right)}^ \wedge }\left( \xi  \right)} \right|}^2}d\xi }  \hfill \\
     &= \alpha {\left\| {{{\left( {Df} \right)}^ \wedge }} \right\|^2} \Rightarrow  f'\in L^2.
  \end{align*}
\end{enumerate}


















\subsection*{Problem 6.25:} 
Use the Fourier transform to find a solution to 
\[u - \frac{\partial^2 u }{\partial x_1^2 } - \frac{\partial^2 u }{\partial x_2^2 } = e^{-x_1^2 - x_2^2 } \] 
Can you find a fundamental solution? 

\paragraph*{Solution} If we can find a fundamental solution, then we have solved the given right hand side. Note $d=2$. Let us attempt to solve 
\[u_0 - \frac{{{\partial ^2}u_0}}{{\partial x_1^2}} - \frac{{{\partial ^2}u_0}}{{\partial x_2^2}} = {\delta _0}.\] 
Taking Fourier transform, 
\[\hat u_0 - {\left( {i{\xi _1}} \right)^2}\hat u_0 - {\left( {i{\xi _2}} \right)^2}\hat u_0 = \hat u_0\left( {1 + {{\left| {{\xi _1}} \right|}^2} + {{\left| {{\xi _2}} \right|}^2}} \right) = {\left( {2\pi } \right)^{ - 1}}\]
so, in the Fourier domain, the fundamental solution is 
\[\hat u_0 = {\left( {2\pi } \right)^{ - 1}}\frac{1}{{1 + {{\left| {{\xi _1}} \right|}^2} + {{\left| {{\xi _2}} \right|}^2}}} \Rightarrow {{ u}_0} = {\left[ {{{\left( {2\pi } \right)}^{ - 1}}\frac{1}{{1 + {{\left| {{\xi _1}} \right|}^2} + {{\left| {{\xi _2}} \right|}^2}}}} \right]^ \vee }.\]
Now for any right hand side $f$, the solution is given by $u = u_0 \ast f$. In this case, 
\[u = {\left[ {{{\left( {2\pi } \right)}^{ - 1}}\frac{1}{{1 + {{\left| {{\xi _1}} \right|}^2} + {{\left| {{\xi _2}} \right|}^2}}}} \right]^ \vee } * {e^{ - x_1^2 - x_2^2}}\]







\cred 
\subsection*{Problem 6.27:}
Telegrapher's equation 
\[u_{tt} + 2u_t + u = c^2 u_{xx} \]
for $x \in \R$ and $t>0$, and $u(x, 0) = f(x)$ and $u_t(x, 0) = g(x)$ are given in $L^2$. 
\begin{enumerate}
  \item Use Fourier transform in $x$ and its inverse to find an explicit representation of the solution. 
  \item Justify your representation is a solution. 
  \item Show that the solution can be viewed as the sum of two wave packets, one moving to the right with a constant speed, and one moving to the left with the same speed. 
\end{enumerate}
















\cred 
\subsection*{Problem 6.30:}
Consider 
\[\Delta^2 u + u = f(x)\] 
for $x\in \R^d$ and $\Delta^2 = \Delta\Delta$. 
\begin{enumerate}
  \item Suppose that $f \in \sch ' \left( \R^d  \right)$. Use the Fourier transform to find a solution $u \in \sch'$. Leave answer as a multiplier operator. 
  \item Is the solution unique? 
  \item Suppose that $f \in \sch \left( \R^d  \right)$. Write the solution as a convolution operator. 
  \item Show that $\check{m}\in L^2 \left( \R^d \right)$ for some range of $d$ and use this to extend the convolution solution to $f\in L^1\left( \R^d \right)$. In that case, in what $L^p$ space is $u$? 
\end{enumerate}


















\end{document}
 