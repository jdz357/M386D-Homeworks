\documentclass[letterpaper,twoside,11pt]{article}
\usepackage{a4wide,graphicx,fancyhdr,clrscode,tabularx,amsmath,amssymb,amsfonts,color,enumitem, bm, array, textcomp, subcaption, color, listings, chemformula, tcolorbox, setspace, xcolor}
\usepackage{amsthm}		%theorem style 
%\usepackage{mathptmx}      %SET MATH TYPE FONT TO TIMES NEW ROMAN
%These lines make the theorem NAME BOLD
\newtheoremstyle{mystyle}%                % Name
  {}%                                     % Space above
  {}%                                     % Space below
  {\itshape}%                                     % Body font
  {}%                                     % Indent amount
  {\bfseries}%                            % Theorem head font
  {.}%                                    % Punctuation after theorem head
  { }%                                    % Space after theorem head, ' ', or \newline
  {\thmname{#1}\thmnumber{ #2}\thmnote{ (#3)}}%                                     % Theorem head spec (can be left empty, meaning `normal')
\theoremstyle{mystyle}
\newtheorem{theorem}{Theorem}[section]
%end of making theorem name bold. 
\newtheorem*{thm}{Theorem}		%Theorem no number. 
\newtheorem{definition}{Definition}[section]
\newtheorem{corollary}{Corollary}[theorem]
\newtheorem{lemma}[theorem]{Lemma}
\newtheorem{prop}{Proposition}[section]
\newtheorem*{ex}{Example}
\newtheorem{notee}{Note}[section]
\newtheorem*{note}{Note}
\usepackage[super]{nth}
\usepackage[makeroom]{cancel}

%----------------------- Macros and Definitions --------------------------
\setlength{\fboxsep}{2.5\fboxsep}
%Sets boxlength size

\setlength\headheight{15pt}
\addtolength\topmargin{-25pt}
\addtolength\footskip{0pt}

\fancypagestyle{plain}{%
\fancyhf{}
\fancyhead[LO,RE]{\sffamily UT Austin}
\fancyhead[RO,LE]{\sffamily CSE 386D}
\fancyfoot[LO,RE]{\sffamily Oden Institute}
\fancyfoot[RO,LE]{\sffamily\bfseries\thepage} 
\renewcommand{\headrulewidth}{0pt}
\renewcommand{\footrulewidth}{0pt}
}

\pagestyle{fancy}
\fancyhf{}
\fancyhead[RO,LE]{\sffamily CSE 386D}
\fancyhead[LO,RE]{\sffamily UT Austin}
\fancyfoot[LO,RE]{\sffamily Oden Institute}
\fancyfoot[RO,LE]{\sffamily\bfseries\thepage}
\renewcommand{\headrulewidth}{1pt}
\renewcommand{\footrulewidth}{0pt}
\newcommand{\R}{{\mathbb R}}
\newcommand{\N}{{\mathbb N}}
\newcommand{\Z}{{\mathbb Z}}
\newcommand{\Q}{{\mathbb Q}}
\newcommand{\C}{{\mathbb C}}
\newcommand{\SL}{{\mathcal{L}}}
\usepackage{libertine}            %% For fancy font
\DeclareMathOperator*{\slim}{s-lim}
\newcommand{\cg}{\color{gray}}
\newcommand{\cbk}{\color{black}}
\newcommand{\cred}{\color{red}}
\newcommand{\cblu}{\color{blue}}
\newcommand{\inv}{^{-1}}
\newcommand{\sch}{\mathcal S} 




\begin{document}
%\fontfamily{ptm}\selectfont     %% TO SELECT THE FONT
\title{\vspace{-2\baselineskip} 
Homework 5
}
%\author{Jonathan Zhang \qquad EID: { jdz357} \qquad { jdz@utexas.edu}}
\author{Jonathan Zhang \qquad EID: { jdz357} }
\date{}
\maketitle


\subsection*{Problem 7.6:}
Suppose $\Omega \subset \R^d$ is a possibly unbounded domain and $\{U_\alpha\}_{\alpha \in I}$ is a collection of open sets in $\R^d$ that cover $\Omega$. Prove there exists a locally finite partition of unity in $\Omega$ subordinate to the cover. 

\paragraph*{Solution} We start with a lemma: 
\begin{lemma}
  If $S$ is a countable dense subset of $\R^d$ with the collection of balls 
  \[\mathcal B = \left\{ B_r(x) \subset \R^d | r \in \mathbb{Q}, x\in S, \text{ and } \exists \alpha \in I : B_r(x) \subset U_\alpha \right\},\]
  then $\cup B_{r_j/2}(x_j)$ is an open cover of $\Omega$. 

\end{lemma}
\begin{proof}
  Fix $r_j/2 >0$ for all $j$ and consider a point $x\in \Omega$. Suppose now to the contrary that there does not exists a $j$ such that $x$ belongs to $B_{r_j/2}(x_j)$. Then, since $\Q$ is dense in $\R$, for any $\varepsilon>0$ there exists a rational $q$ arbitrarily close to $x$. Since $q\in Q$, then also $q\in S$, and so it has a radius $r_{q/2}$ associated with it. If $\varepsilon \leq r_{q/2}$, then we are done. If not, select a new $\varepsilon_1 < \varepsilon$ and repeat the process, arriving at a new $q_1\in \Q$ such that $x\in B_{\varepsilon_1}(q_1)$. Repeating, we keep looking with smaller $\varepsilon_k$, $q_k$, $r_{q_k/2}$. If we repeat infinitely many times, though, we have found some $j$ such that $\varepsilon > r_j$ for every $\varepsilon >0$, otherwise we would have stopped. The only way to satisfy this would be for $r_j = 0$, a contradiction. 
\end{proof}

We now want to construct the functions that make up the partition of unity. Let $\varphi(x)$ denote Cauchy's infinitely differentiable function in 1D. We can easily translate and dilate this to form a bump function $\phi_j \in C_0^\infty \left( B_j \right)$ satisfying $0 \leq \phi_j \leq 1$ and $\phi_j = 1$ on $B_{r_j/2}$:
\[\phi_j(x) = \varphi\left( \frac{2}{r_j} \left( |x-x_j| +r_j\right)  \right) \varphi\left( -\frac{2}{r_j} \left( |x-x_j| -r_j\right)  \right).\]
Now let $\psi_1 = \phi_1$, $\psi_k = \prod_{j = 1}^{k-1}\psi_j \psi_k$. 

We can now demonstrate the desired properties. 
First of all, property (iii) is easy to see. Indeed, for every $j \in \mathcal J$ that indexes $\mathcal B$, there exists $\alpha \in I$ such that 
\[\text{supp }\psi_j \subset \text{supp } \phi_j \subset \text{supp } B_j \subset \text{supp }U_\alpha.\]

Now we show that for every compact $K\subset \Omega$, all but finitely many $\psi_j$ vanish. Observe that if $\psi_i = 0$, then all $\psi_{i+k}$ vanish for $k>0$. 
We have already shown that $B_{r_j/2}$ provide an open cover for $\Omega$. Since $K$ is compact, it admits a finite subcover, that is, $K\subset \cup_{j=1}^M B_{r_j/2}(x_j)$. Now for $x \in K$, there exists a $N\leq M$ such that $x\in B_{r_N/2}(x_N)$. We then have that $\phi_N(x) = 1$, and that $\psi_n(x) = 0$ for $n\geq N$. So, only finitely many $\psi$ support on $K$. It is also easy to see that $\psi_j$ range betewen $0$ and $1$. To show the summation property, go by induction. We need to show that 
\[\sum_{j = 1 }^{m}\psi_j = 1 - \prod_{j = 1}^{m}\left( 1-\phi_j \right).\]
This holds trivially for $m = 1$. Now, for the inductive step, 
\begin{align*}
  \sum_{j = 1}^{m+1}\psi_j &= \psi_{m+1} + \sum_{j = 1}^{m}\psi_j\\ 
  &= \psi_{m+1} + 1 - \prod_{j = 1 }^{m } \left( 1-\phi_j \right) \\
  &= \phi_{m+1}\prod_{j = 1}^{m}\left( 1- \phi_j \right) + 1 - \prod_{j = 1}^{m}\left( 1-\phi_j \right)\\
  &=1-\prod_{j = 1}^{m+1}\left( 1-\phi_j \right).
\end{align*} 
To show the summation property, we again use the fact that we have an open cover of half-radius balls. Using what we have just proved, 
\[\sum_{j \in \mathcal J} \psi_j (x) = \sum_{j = 1}^{m}\psi_j(x) = 1- \prod_{j = 1 }^{m }\left( 1-\phi_j(x) \right)= 1-\left( 1-\phi_m(x) \right)\prod_{j = 1}^{m-1}\left( 1-\phi_j(x) \right) = 1.\]











\subsection*{Problem 7.7:}
$u \in \mathcal D'\left( \R^d \right)$, $\phi \in \mathcal D$. 
\paragraph*{Solution}
\begin{enumerate}
  \item Define the functional 
  \[\psi:\R\ni t \to \psi(t) = \phi(x-ty) \in \R.\]
  Apply the fundamental theorem, 
  \begin{align*}
    \psi \left( 1 \right) - \psi \left( 0 \right) &= \int\limits_0^1 {\frac{{d\psi }}{{dt}}dt}  \hfill \\
    \phi \left( {x - y} \right) - \phi \left( x \right) &= \int\limits_0^1 {\frac{d}{{dt}}\phi \left( {x - ty} \right)dt}  \hfill \\
    \tau_y\phi - \phi \left( x \right) &= \int\limits_0^1 { - y \cdot \nabla \phi \left( {x - ty} \right)dt} 
  \end{align*}
  and apply $u \in \mathcal D'$. We have 
  \begin{align*}
    u\left( {{\tau _y}\phi  - \phi } \right) = u\left( {{\tau _y}\phi } \right) - u\left( \phi  \right) &= u\left( {\int\limits_0^1 { - y \cdot \nabla \left( {{\tau _{ty}}\phi } \right)dt} } \right) \hfill \\
     &=  - \int\limits_0^1 {u\left( {\sum\limits_{j = 1}^d {{y_j}\frac{\partial }{{\partial {x_j}}}\left( {{\tau _{ty}}\phi } \right)} } \right)dt}  \hfill \\
     &=  - \int\limits_0^1 {\sum\limits_{j = 1}^d {{y_j}\left\langle {u,\frac{\partial }{{\partial {x_j}}}\left( {{\tau _{ty}}\phi } \right)} \right\rangle } dt}  \hfill \\
     &= \int\limits_0^1 {\sum\limits_{j = 1}^d {{y_j}\left\langle {\frac{\partial }{{\partial {x_j}}}u,\left( {{\tau _{ty}}\phi } \right)} \right\rangle } dt}  \hfill \\
     &= \int\limits_0^1 {\sum\limits_{j = 1}^d {{y_j}\frac{{\partial u}}{{\partial {x_j}}}\left( {{\tau _{ty}}\phi } \right)} dt} 
    \end{align*}
    \item Let $f \in W^{1,1}_{loc} \left( \R^d \right)$. Then, 
    \[{\Lambda _f}\left( {{\tau _y}\phi } \right) - {\Lambda _f}\phi  = \int\limits_0^1 {\sum\limits_{j = 1}^d {{y_j}\frac{{\partial f}}{{\partial {x_j}}}\left( {{\tau _{ty}}\phi } \right)} dt} ,\]
    \[{\tau _{ - y}}{\Lambda _f}\phi  - {\Lambda _f}\phi  = \int\limits_{{\mathbb{R}^d}} {\left( {f\left( {x + y} \right) - f\left( x \right)} \right)\phi \left( x \right)dx}  = \int\limits_0^1 {\sum\limits_{j = 1}^d {{y_j}\int\limits_{{\mathbb{R}^d}} {\frac{{\partial f}}{{\partial {x_j}}}\left( {x + ty} \right)\phi \left( x \right)dx} } dt} .\]
    Exchanging order of integration, we have 
    \[\int\limits_{{\mathbb{R}^d}} {\left( {f\left( {x + y} \right) - f\left( x \right)} \right)\phi \left( x \right)dx}  = \int\limits_{{\mathbb{R}^d}} {\left\{ {\int\limits_0^1 {\sum\limits_{j = 1}^d {{y_j}\frac{{\partial f}}{{\partial {x_j}}}\left( {x + ty} \right)dt} } } \right\}\phi \left( x \right)dx} .\] 
    Now, we see equality of the prefactors within the integrand. Finally, we need just recognize that $y \cdot \nabla f \left( x+ty \right)$ is exactly the expssion represented by the sum. 
    \item For $p \in (1,\infty]$, $L_{loc}^p\left( \R^d \right)\subset L_{loc}^1$. Thus, for $f\in W^{1, \infty}_{loc}\left( \R^d \right)$, $f \in W^{1,1}_{loc}$ and we can apply the result from the previous part. The previous part tells us that 
    \[\left| {f\left( {x + \left( {y - x} \right)} \right) - f\left( x \right)} \right| = \int\limits_0^1 {\left( {y - x} \right) \cdot \nabla f\left( {x + t\left( {y - x} \right)} \right)dt} .\]
    So then, 
    \begin{align*}
      \left| {f\left( y \right) - f\left( x \right)} \right| &= \left| {\int\limits_0^1 {\left( {y - x} \right) \cdot \nabla f\left( {x + t\left( {y - x} \right)} \right)dt} } \right| \hfill \\
       &\leqslant \left| {y - x} \right|\int\limits_0^1 {dt} {\left\| {\nabla f} \right\|_{{L^\infty }\left( {{B_R}\left( 0 \right)} \right)}} \hfill \\
       &\leqslant {C_{R,f}}\left| {y - x} \right| 
      \end{align*}
\end{enumerate}




\subsection*{Problem 7.8:} Counterexamples. 
\begin{enumerate}
  \item We know that for $f(x) = |x|^r$, that $f \in W^{1,p}\left( \Omega \right)$ if and only if $r > -\left( d-p \right)/p$. For $q > dp/(d-p)$, $f \notin L^q \left( \Omega \right)$ if, for some $B_r(0) \subset \Omega \cap B_1(0)$, $\|f\|_q^q \geq \int_{B_r(0)} |x|^{rq}dx$, that is, if $rq \leq -1$. Thus, we can find a counterexample for $r \in ( -(d-p)/p, -(d-p)/dp ]$. 
  \item The previous example is bot bounded and shows there is no embedding $W^{1,p}\left( \Omega \right)\hookrightarrow C_B^0$. However, from the Sobolev embedding theorem, if $\Omega$ is bounded and $mp>d$, then $W^{m,p}_0\left( \Omega \right)\hookrightarrow C^0_B\left( \Omega \right)$. In the case of the $H^s$ spaces, though, we have $p=2$ and thus $m>1/2$, so $W^{s,2}_0 \hookrightarrow C^0_B$ for $s > 1/2$. This implies that $\delta \in H^s$ for $s < -1/2$. 
  \item Notice immediately that $f$ is not bounded at the origin. However, 
  \[{\left| {\log \log \left( {\frac{{4R}}{{\left| x \right|}}} \right)} \right|^p} \leqslant {\left( {4R{{\left| x \right|}^{ - 1/2p}}} \right)^p}\] for every $ x\in B_r(0)$. So then, $f \in L^p\left( \Omega \right)$. For the derivative, notice that 
  \[Df\left( x \right) = \frac{{ - x}}{{\left| {{x^2}} \right|\log \left( {4R/\left| x \right|} \right)}}.\]
  Define a similar sequence 
  \[D{f_n} = \left\{ {\begin{array}{*{20}{c}}
    {Df\left( x \right)}&{x \in \Omega  \setminus {B_n}} \\ 
    0&\text{otherwise} 
  \end{array}} \right.\]
  and observe that $Df_n$ converges monotonically to $Df$ in $L^p\left( \Omega \right)$. Computation of the norm is done in spherical coordinates, 
  \[\left\| {D{f_n}} \right\|_p^p = \int\limits_{\Omega  \setminus {B_n}} {{{\left| {\frac{{ - x}}{{\left| {{x^2}} \right|\log \left( {4R/\left| x \right|} \right)}}} \right|}^p}dx}  = \int\limits_{\Omega  \setminus {B_n}} {{{\left| {\frac{1}{{\left| x \right|\log \left( {4R/\left| x \right|} \right)}}} \right|}^p}dx} \sim \int\limits_{{\varepsilon _n}}^R {\frac{1}{{\rho \log \left( {4R/\rho } \right)}}d\rho } .\]
  As $\rho\to0$, the integrand is zero and is bounded across the rest of the domain. Thus, $f \in W^{1,p}$ as requested. 
  \item The distributional derivative of absolute value is 
  \[Du = \left\{ {\begin{array}{*{20}{c}}
    { - 1}&{x < 0} \\ 
    a&{x = 0} \\ 
    1&{x > 0} 
  \end{array}} \right.\] for any choice of $a\in \R$. Thus, $Du \in L^\infty\left( -1,1 \right)$ and $u \in W^{1,\infty}\left( -1, 1 \right)$. Assume now to the contrary that there exists a sequence of smooth functions $f_n \in C^\infty$, $f_n \to u$ in $W^{1,\infty}\left( -1, 1 \right)$. If this is the case, then for the derivative, we have $Df_n \to Du$ in $L^\infty$. Since $f_n$ are smooth, then so are $Df_n$. For $\varepsilon = 1/2$, this implies in particular that $Df_n(x) \in \left( -3/2, -1/2 \right)$ for $x<0$ and $Df_n \in (1/2, 3/2)$ for $x>0$. The contradiction is that the intermediate value theorem implies that there exists an interval such that $Df_n \in [-1/2, 1/2]$ in order to connect these two regions. So, $C^\infty \left( -1, 1 \right)\cap W^{1,\infty}\left( -1, 1 \right)$ is not dense in $W^{1,\infty}\left( -1, 1 \right)$. 
\end{enumerate}



\subsection*{Problem 7.11:}
We have 

\begin{align*}
  \left\| f \right\|_{{H^s}}^2 &= \int\limits_{{\mathbb{R}^d}} {{{\left( {1 + {{\left| \xi  \right|}^2}} \right)}^s}{{\left| {\hat f\left( \xi  \right)} \right|}^2}d\xi }  \hfill \\
   &= \int\limits_{{\mathbb{R}^d}} {{{\left( {1 + {{\left| \xi  \right|}^2}} \right)}^s}{{\left| {\hat f\left( \xi  \right)} \right|}^{2s}}{{\left| {\hat f\left( \xi  \right)} \right|}^{2\left( {s - 1} \right)}}d\xi }  \hfill \\
   &= \int\limits_{{\mathbb{R}^d}} {{{\left\{ {{{\left( {1 + {{\left| \xi  \right|}^2}} \right)}^{1/2}}\left| {\hat f\left( \xi  \right)} \right|} \right\}}^{2s}}{{\left| {\hat f\left( \xi  \right)} \right|}^{2\left( {s - 1} \right)}}d\xi }  \hfill \\
   &\leqslant {\left\| {{{\left( {1 + {{\left| \xi  \right|}^2}} \right)}^{1/2}}\left| {\hat f\left( \xi  \right)} \right|} \right\|^{2s}}_{{L^2}\left( {{\mathbb{R}^d}} \right)}{\left\| {{{\left| {\hat f} \right|}^{2\left( {s - 1} \right)}}} \right\|_{{L^2}\left( {{\mathbb{R}^d}} \right)}} \hfill \\
   &\leqslant {\left\| {{{\left( {1 + {{\left| \xi  \right|}^2}} \right)}^{1/2}}\left| {\hat f\left( \xi  \right)} \right|} \right\|^{2s}}_{{L^2}\left( {{\mathbb{R}^d}} \right)}{\left\| {\left| {\hat f} \right|} \right\|^{2\left( {1 - s} \right)}}_{{L^2}\left( {{\mathbb{R}^d}} \right)} 
\end{align*}
and hence, 
\[\left\| f \right\|_{{H^s}}^{} \leqslant {\left\| {{{\left( {1 + {{\left| \xi  \right|}^2}} \right)}^{1/2}}\left| {\hat f\left( \xi  \right)} \right|} \right\|^s}_{{L^2}\left( {{\mathbb{R}^d}} \right)}{\left\| {\left| {\hat f} \right|} \right\|^{\left( {1 - s} \right)}}_{{L^2}\left( {{\mathbb{R}^d}} \right)} \Rightarrow \left\| f \right\|_{{H^s}}^{} \leqslant \left\| f \right\|_{{H^1}}^s\left\| f \right\|_{{L^2}}^{1 - s}\]




\end{document}
 