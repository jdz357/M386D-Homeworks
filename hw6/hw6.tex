\documentclass[letterpaper,twoside,11pt]{article}
\usepackage{a4wide,graphicx,fancyhdr,clrscode,tabularx,amsmath,amssymb,amsfonts,color,enumitem, bm, array, textcomp, subcaption, color, listings, chemformula, tcolorbox, setspace, xcolor}
\usepackage{amsthm}		%theorem style 
%\usepackage{mathptmx}      %SET MATH TYPE FONT TO TIMES NEW ROMAN
%These lines make the theorem NAME BOLD
\newtheoremstyle{mystyle}%                % Name
  {}%                                     % Space above
  {}%                                     % Space below
  {\itshape}%                                     % Body font
  {}%                                     % Indent amount
  {\bfseries}%                            % Theorem head font
  {.}%                                    % Punctuation after theorem head
  { }%                                    % Space after theorem head, ' ', or \newline
  {\thmname{#1}\thmnumber{ #2}\thmnote{ (#3)}}%                                     % Theorem head spec (can be left empty, meaning `normal')
\theoremstyle{mystyle}
\newtheorem{theorem}{Theorem}[section]
%end of making theorem name bold. 
\newtheorem*{thm}{Theorem}		%Theorem no number. 
\newtheorem{definition}{Definition}[section]
\newtheorem{corollary}{Corollary}[theorem]
\newtheorem{lemma}[theorem]{Lemma}
\newtheorem{prop}{Proposition}[section]
\newtheorem*{ex}{Example}
\newtheorem{notee}{Note}[section]
\newtheorem*{exercise}{Exercise}
\newtheorem*{note}{Note}
\usepackage[super]{nth}
\usepackage[makeroom]{cancel}

%----------------------- Macros and Definitions --------------------------
\setlength{\fboxsep}{2.5\fboxsep}
%Sets boxlength size

\setlength\headheight{15pt}
\addtolength\topmargin{-25pt}
\addtolength\footskip{0pt}

\fancypagestyle{plain}{%
\fancyhf{}
\fancyhead[LO,RE]{\sffamily UT Austin}
\fancyhead[RO,LE]{\sffamily CSE 386D}
\fancyfoot[LO,RE]{\sffamily Oden Institute}
\fancyfoot[RO,LE]{\sffamily\bfseries\thepage} 
\renewcommand{\headrulewidth}{0pt}
\renewcommand{\footrulewidth}{0pt}
}

\pagestyle{fancy}
\fancyhf{}
\fancyhead[RO,LE]{\sffamily CSE 386D}
\fancyhead[LO,RE]{\sffamily UT Austin}
\fancyfoot[LO,RE]{\sffamily Oden Institute}
\fancyfoot[RO,LE]{\sffamily\bfseries\thepage}
\renewcommand{\headrulewidth}{1pt}
\renewcommand{\footrulewidth}{0pt}
\newcommand{\R}{{\mathbb R}}
\newcommand{\N}{{\mathbb N}}
\newcommand{\Z}{{\mathbb Z}}
\newcommand{\Q}{{\mathbb Q}}
\newcommand{\C}{{\mathbb C}}
\newcommand{\SL}{{\mathcal{L}}}
\usepackage{libertine}            %% For fancy font
\DeclareMathOperator*{\slim}{s-lim}
\newcommand{\cg}{\color{gray}}
\newcommand{\cbk}{\color{black}}
\newcommand{\cred}{\color{red}}
\newcommand{\cblu}{\color{blue}}
\newcommand{\inv}{^{-1}}
\newcommand{\sch}{\mathcal S} 




\begin{document}
%\fontfamily{ptm}\selectfont     %% TO SELECT THE FONT
\title{\vspace{-2\baselineskip} 
Homework 6
}
%\author{Jonathan Zhang \qquad EID: { jdz357} \qquad { jdz@utexas.edu}}
\author{Jonathan Zhang \qquad EID: { jdz357} }
\date{}
\maketitle



\begin{exercise}[7.10]
Suppose $\Omega\subset \R^d$ is bounded with Lipschitz boundary and $f_j \rightharpoonup f$ and $g_j\rightharpoonup g$ weakly in $H^1\left( \Omega \right)$. Show that, for a subsequence, $\nabla \left( f_jg_j \right) \to \nabla (fg)$ as a distribution. Find all $p$ in $[1,\infty]$ such that the convergence can be taken weakly in $L^p\left( \Omega \right)$. 
\end{exercise}







\begin{exercise}[7.15]
Suppose that $w \in L^1\left( \R \right)$, $w(x) > 0$, and $w$ is even. Moreover, for $x > 0$, $w \in C^2[0, \infty)$, $w'(x) < 0$, and $w''(x) >0$. Consider the equation 
\[w\ast u - u'' = f\in L^2\left( \R \right).\]
\begin{enumerate}
  \item Show that $\hat w \left( \xi \right) >0$. 
  \item Fund a fundamental solution. Leave answer in terms of inverse Fourier transform. 
  \item Find the solution operator as a convolution operator. 
  \item Show that $u \in H^2\left( \R \right)$. 
\end{enumerate}
\end{exercise}

\begin{proof}
  \begin{enumerate}
    \item First, use some properties of $w$: 
    \begin{align*}
      \hat w\left( \xi  \right) &= {\left( {2\pi } \right)^{ - 1/2}}\int\limits_\mathbb{R} {w\left( x \right){e^{ - ix\xi }}dx}  \hfill \\
       &= {\left( {2\pi } \right)^{ - 1/2}}\int\limits_\mathbb{R} {w\left( x \right)\left( {\cos \left( {x\xi } \right) - i\sin \left( {x\xi } \right)} \right)dx}  \hfill \\
       &= {\left( {2\pi } \right)^{ - 1/2}}\int\limits_\mathbb{R} {w\left( x \right)\cos \left( {x\xi } \right)dx}  \hfill \\
       &= 2{\left( {2\pi } \right)^{ - 1/2}}\int\limits_0^\infty  {w\left( x \right)\cos \left( {x\xi } \right)dx}  
    \end{align*}
    Since the product $w(x)\sin(x\xi)$ is odd, its integral over the real line is zero. IBP tells us 
    \begin{align*}
        \int\limits_0^\infty  {w\left( x \right)\cos \left( {x\xi } \right)dx}  = \left[ {w\left( x \right)\sin \left( {x\xi } \right)\frac{1}{\xi }} \right]_0^\infty  - \int\limits_0^\infty  {\frac{1}{\xi }w'\left( x \right)\sin \left( {x\xi } \right)dx}  \hfill \\
         = \mathop {\lim }\limits_{x \to \infty } w\left( x \right)\sin \left( {x\xi } \right)\frac{1}{\xi } - \int\limits_0^\infty  {\frac{1}{\xi }w'\left( x \right)\sin \left( {x\xi } \right)dx}  .
    \end{align*}
    Let us examine the integral 
    \[ - \int\limits_0^\infty  {\frac{1}{\xi }w'\left( x \right)\sin \left( {x\xi } \right)dx} .\]
    and consider it first on an interval $(0, 2\pi / |\xi|)$. For the moment, let us restrict ourselves to $\xi > 0$. We will demonstrate that over a single interval, the integral is positive. Then, we will extend the same reasoning to all the inervals, and conclude that the original integral is positive. 

    Let us decompose the integral as
    \[\int\limits_0^{2\pi /\left| \xi  \right|} {w'\left( x \right)\sin \left( {x\xi } \right)dx}  = \int\limits_0^{\pi /\left| \xi  \right|} {w'\left( x \right)\sin \left( {x\xi } \right)dx}  + \int\limits_{\pi /\left| \xi  \right|}^{2\pi /\left| \xi  \right|} {w'\left( x \right)\sin \left( {x\xi } \right)dx} .\]
    %The first integral is negative, since the product $w'(x) \sin(x\xi) < 0$ for $x \in (0, \pi/\xi)$. 
    For the second integral, change variables to $u = x- \pi/|\xi|$: 
    \begin{align*}
    \int\limits_0^{2\pi /\left| \xi  \right|} {w'\left( x \right)\sin \left( {x\xi } \right)dx}  &= \int\limits_0^{\pi /\left| \xi  \right|} {w'\left( x \right)\sin \left( {x\xi } \right)dx}  + \int\limits_{0}^{\pi /\left| \xi  \right|} {w'\left( u+\pi/|\xi|\right)\sin \left( {(u+\pi/|\xi|)\xi } \right)dx} \\
    &=\int\limits_0^{\pi /\left| \xi  \right|} {w'\left( x \right)\sin \left( {x\xi } \right)dx}  - \int\limits_{0}^{\pi /\left| \xi  \right|} {w'\left( u+\pi/|\xi|\right)\sin (u\xi)du} \\
    &=\int\limits_0^{\pi /\left| \xi  \right|} \left({w'\left( u \right)}  -  w'\left( u+\pi/|\xi|\right)\right)\sin (u\xi)du.
    \end{align*}
    Now by Taylor expansion, 
    \[w'\left( u + \pi/|\xi| \right) = w'(u) + w''(\tilde u) \pi/|\xi|\quad \text{for some } \tilde u \in [u, u+\pi/|\xi|]. \]
    So the integral simplifies to 
    \[\int\limits_0^{\pi /\left| \xi  \right|} -\pi/|\xi|w''\left( \tilde u \right)\sin (u\xi)du.\]
    The sign of this integral is negative by inspection. The same reasoning applies for the integral over the interval $(n\pi/|\xi|, (n+1)\pi/|\xi|)$. That means the original integral is positive since it is premultiplied by $-1/\xi$. 

  \end{enumerate}
\end{proof}












\begin{exercise}[7.16]
  Let $H$ be a Hilbert space and $Z$ a closed linear subspace. Prove that $H/Z$ has the inner product 
  \[\left( \hat x, \hat y \right)_{H/Z} = \left( P_Z^{\perp} x,P_Z^{\perp} y  \right)_H\]
  and that $H/Z$ is isomorphic to $Z^\perp$.
\end{exercise}

We want to show the following: 
\begin{enumerate}
  \item The proposed inner product is well defined. 
  \item The inner product in $H/Z$ is in fact an inner product. 
  \item There is a natural isomorphism between $H/Z$ and $Z^\perp$. 
\end{enumerate}
\begin{proof}
  \begin{enumerate}
    \item Let $x_1, x_2 \in \hat x$ be two elements in the equivalence class. Then, 
    \[{\left( {\hat x,\hat y} \right)_{H/Z}} = \left( {P_Z^ \bot {x_1},P_Z^ \bot y} \right) = \left( {P_Z^ \bot {x_2},P_Z^ \bot y} \right)\]
    if and only if 
    \[\left( {P_Z^ \bot \left( {{x_1} - {x_2}} \right),P_Z^ \bot y} \right) = 0.\]
    But, since $x_1$ and $x_2$ belong to the same equivalence class, their difference is a member of $Z$. Therefore, the projection ${P_Z^ \bot \left( {{x_1} - {x_2}} \right)}=0$, so the value of ${\left( {\hat x,\hat y} \right)_{H/Z}}$ is independent of the choice of representative. 

    \item Symmetry: 
    \[{\left( {\hat x,\hat y} \right)_{H/Z}} = {\left( {P_Z^ \bot {x_1},P_Z^ \bot y} \right)_H} = {\left( {P_Z^ \bot y,P_Z^ \bot {x_1}} \right)_H} = {\left( {\hat y,\hat x} \right)_{H/Z}}.\]
    Positivity: 
    \[{\left( {\hat x,\hat x} \right)_{H/Z}} = {\left( {P_Z^ \bot {x_1},P_Z^ \bot {x_1}} \right)_H} = {\left\| {P_Z^ \bot {x_1}} \right\|^2} \geqslant 0,\]
    and we see clearly that $(\hat x, \hat x)_{H/Z} = 0$ only when $x \in Z$, or equivalently, $x\in \hat 0$. 
    Linearity: Projections are linear, and the inner product in $H$ is linear. 

    \item The two spaces are connected by the natural isomorphism 
    \[i:H/Z \ni \hat x \to P_Z^ \bot {x_1} \in {Z^ \bot };\quad {i^{ - 1}}:{Z^ \bot } \ni x \to \hat x \in H/Z.\]
    Map $i$ is a well defined map, since we just showed the projection is independent of the choice of representative. 
  \end{enumerate}
\end{proof}




\begin{exercise}[7.17]
  Elliptic regularity theorey shows that if $\Omega \subset \R^d$ has a smooth boundary and $f \in H^s\left( \Omega \right)$, then $-\Delta u = f$ in $\Omega$, $u = 0$ on $\partial \Omega$ has a unique solution $u \in H^{s+2}$. For what values of $s$ will $u$ be continuous? Can you be sure that a fundamental solution is continuous? Answer depends on $d$. 

\end{exercise}




\begin{exercise}[7.19]
  Let $\Omega \subset \R^d$ be a domain with Lipschitz boundary. Let $w\in L^\infty \left( \Omega \right)$. Define 
  \[H_w\left( \Omega \right) = \left\{f \in L^2 (\Omega) : \nabla \left( wf \right) \in \left( L^2(\Omega) \right)^d\right\}\]
\begin{enumerate}
  \item Give reasonable conditions on $w$ so that $H_w\left( \Omega \right) = H^1 \left( \Omega \right)$. 
  \item Prove that $H_w\left( \Omega \right)$ is a Hilbert space. What is the inner product? 
  \item Can you define the trace of $f$ on $\partial \Omega$? First consider trace of $wf$. 
  \item Characterize $H_w\left( \R \right)$ if $w(x)$ is the Heaviside function. 
\end{enumerate}
\end{exercise}



\end{document}
 