\documentclass[letterpaper,twoside,11pt]{article}
\usepackage{a4wide,graphicx,fancyhdr,clrscode,tabularx,amsmath,amssymb,amsfonts,color,enumitem, bm, array, textcomp, subcaption, color, listings, chemformula, tcolorbox, setspace, xcolor}
\usepackage{amsthm}		%theorem style 
%\usepackage{mathptmx}      %SET MATH TYPE FONT TO TIMES NEW ROMAN
%These lines make the theorem NAME BOLD
\newtheoremstyle{mystyle}%                % Name
  {}%                                     % Space above
  {}%                                     % Space below
  {\itshape}%                                     % Body font
  {}%                                     % Indent amount
  {\bfseries}%                            % Theorem head font
  {.}%                                    % Punctuation after theorem head
  { }%                                    % Space after theorem head, ' ', or \newline
  {\thmname{#1}\thmnumber{ #2}\thmnote{ (#3)}}%                                     % Theorem head spec (can be left empty, meaning `normal')
\theoremstyle{mystyle}
\newtheorem{theorem}{Theorem}[section]
%end of making theorem name bold. 
\newtheorem*{thm}{Theorem}		%Theorem no number. 
\newtheorem{definition}{Definition}[section]
\newtheorem{corollary}{Corollary}[theorem]
\newtheorem{lemma}[theorem]{Lemma}
\newtheorem{prop}{Proposition}[section]
\newtheorem*{ex}{Example}
\newtheorem{notee}{Note}[section]
\newtheorem*{exercise}{Exercise}
\newtheorem*{note}{Note}
\usepackage[super]{nth}
\usepackage[makeroom]{cancel}

%----------------------- Macros and Definitions --------------------------
\setlength{\fboxsep}{2.5\fboxsep}
%Sets boxlength size

\setlength\headheight{15pt}
\addtolength\topmargin{-25pt}
\addtolength\footskip{0pt}

\fancypagestyle{plain}{%
\fancyhf{}
\fancyhead[LO,RE]{\sffamily UT Austin}
\fancyhead[RO,LE]{\sffamily CSE 386D}
\fancyfoot[LO,RE]{\sffamily Oden Institute}
\fancyfoot[RO,LE]{\sffamily\bfseries\thepage} 
\renewcommand{\headrulewidth}{0pt}
\renewcommand{\footrulewidth}{0pt}
}

\pagestyle{fancy}
\fancyhf{}
\fancyhead[RO,LE]{\sffamily CSE 386D}
\fancyhead[LO,RE]{\sffamily UT Austin}
\fancyfoot[LO,RE]{\sffamily Oden Institute}
\fancyfoot[RO,LE]{\sffamily\bfseries\thepage}
\renewcommand{\headrulewidth}{1pt}
\renewcommand{\footrulewidth}{0pt}
\newcommand{\R}{{\mathbb R}}
\newcommand{\N}{{\mathbb N}}
\newcommand{\Z}{{\mathbb Z}}
\newcommand{\Q}{{\mathbb Q}}
\newcommand{\C}{{\mathbb C}}
\newcommand{\SL}{{\mathcal{L}}}
\usepackage{libertine}            %% For fancy font
\DeclareMathOperator*{\slim}{s-lim}
\newcommand{\cg}{\color{gray}}
\newcommand{\cbk}{\color{black}}
\newcommand{\cred}{\color{red}}
\newcommand{\cblu}{\color{blue}}
\newcommand{\inv}{^{-1}}
\newcommand{\sch}{\mathcal S} 




\begin{document}
%\fontfamily{ptm}\selectfont     %% TO SELECT THE FONT
\title{\vspace{-2\baselineskip} 
Homework 7
}
%\author{Jonathan Zhang \qquad EID: { jdz357} \qquad { jdz@utexas.edu}}
\author{Jonathan Zhang \qquad EID: { jdz357} }
\date{}
\maketitle



\begin{exercise}[8.1]
  If $A$ is positive definite, show that its eigenvalues are positive. If $A$ is symmetric and has positive eigenvalues, then $A$ is positive definite. 
\end{exercise}

\begin{proof}
  Suppose that $A$ is positive definite. Let $\left( \lambda, v \right)$ be an eigenpair for $A$. Then by the positive definiteness, 
    \[0<\left( v, Av \right) = \left( v, \lambda v \right) = \lambda \left( v, v \right) = \lambda \|v\|^2.\]
    Since this quantity is strictly positive, we must have $\lambda >0$. 

  On the other side, 
  assume now that $A$ has all positive eigenvalues. Then, $a$ admits a diagonalization $A= U \Lambda U^*$ for some $U^* = U\inv$ unitary matrix. Pick $x \neq 0$, and set $y = U^* x \neq 0$. Then 
    \[\left( x, Ax \right) = \left( x, U \Lambda U^* x \right) = \left( y, \Lambda y \right).\] 
    Expand $y = \sum y_i e_i$, then 
    \[\left( y, \Lambda y \right) = \sum_{i} \overline{y_i} y_i \lambda _i >0\] since $\lambda_i >0$ and $\overline{y_i}y_i = |y_i|^2 \geq 0 $ since $y \neq 0$ by assumption. This shows that $\left( x, Ax \right) = \left( y, \Lambda y \right) >0$ and $a$ is positive definite. 

\end{proof}






\newpage 
\begin{exercise}[8.3]

\end{exercise}





\newpage
\begin{exercise}[8.5]
\end{exercise}










\newpage 
\begin{exercise}[8.8]
  $\Omega \subset \R^d$ bounded, connected, Lipschitz domain. Let $V \subset \Omega$ have positive measure. Let $H = \left\{ u \in H^1 (\Omega) : u|_V = 0 \right\}$. 
  \begin{enumerate}
    \item Why is $H$ a Hilbert space? 
    \item Prove that there exists $C>0$ such that 
    \[\|u\| \leq C \|\nabla u\|\]
    for every $u \in H$. 
  \end{enumerate}
\end{exercise}

\begin{enumerate}
  \item $H$ is the null space of the restriction to $V$ operator. Since restriction is continuous, then $H$ is closed. As a closed subspace of Hilbert space $H^1\left( \Omega \right)$, then $H$ is itself a Hilbert space. 
  \item Go by contradiction. Assume to the contrary that there exists a sequence $u_n \in H$ such that 
  \[\|u_n\|= 1 \qquad \text{and} \qquad \|\nabla u_n\| \to 0.\]
  Now, from every bounded sequence in a Hilbert space, we can extract a weakly convergent subsequence $u_{n_k} \rightharpoonup u \in H$. Weak convergence of $u_{n_k}$ implies weak convergence of $\nabla u_{n_k} \rightharpoonup \nabla u$ in $L^2$. Since the norm is continuous we pass to the limit and we have that $\nabla u = 0$, i.e. $u$ is a constant. But if $u$ vanishes on $V$, then the only possibility is that $u \equiv 0$. 

  On the other side, we know from the Rellich-Kondrachov Theorem that we have $u_{n_k} \to u$ in $L^2$. Convergence in $L^2$ also implies convergence in the norm, but since $\|u_n\| = 1$ that means $\|u\| = 1$, a contradiction. 

\end{enumerate}



\end{document}
 