\documentclass[letterpaper,twoside,11pt]{article}
\usepackage{a4wide,graphicx,fancyhdr,clrscode,tabularx,amsmath,amssymb,amsfonts,color,enumitem, bm, array, textcomp, subcaption, color, listings, chemformula, tcolorbox, setspace, xcolor}
\usepackage{amsthm}		%theorem style 
\usepackage{mathptmx}      %SET MATH TYPE FONT TO TIMES NEW ROMAN
%These lines make the theorem NAME BOLD
\newtheoremstyle{mystyle}%                % Name
  {}%                                     % Space above
  {}%                                     % Space below
  {\itshape}%                                     % Body font
  {}%                                     % Indent amount
  {\bfseries}%                            % Theorem head font
  {.}%                                    % Punctuation after theorem head
  { }%                                    % Space after theorem head, ' ', or \newline
  {\thmname{#1}\thmnumber{ #2}\thmnote{ (#3)}}%                                     % Theorem head spec (can be left empty, meaning `normal')
\theoremstyle{mystyle}
\newtheorem{theorem}{Theorem}[section]
%end of making theorem name bold. 
\newtheorem*{thm}{Theorem}		%Theorem no number. 
\newtheorem{definition}{Definition}[section]
\newtheorem{corollary}{Corollary}[theorem]
\newtheorem{lemma}[theorem]{Lemma}
\newtheorem{prop}{Proposition}[section]
\newtheorem*{ex}{Example}
\newtheorem{notee}{Note}[section]
\newtheorem*{exercise}{Exercise}
\newtheorem*{note}{Note}
\usepackage[super]{nth}
\usepackage[makeroom]{cancel}

%----------------------- Macros and Definitions --------------------------
\setlength{\fboxsep}{2.5\fboxsep}
%Sets boxlength size

\setlength\headheight{15pt}
\addtolength\topmargin{-25pt}
\addtolength\footskip{0pt}

\fancypagestyle{plain}{%
\fancyhf{}
\fancyhead[LO,RE]{\sffamily UT Austin}
\fancyhead[RO,LE]{\sffamily CSE 386D}
\fancyfoot[LO,RE]{\sffamily Oden Institute}
\fancyfoot[RO,LE]{\sffamily\bfseries\thepage} 
\renewcommand{\headrulewidth}{0pt}
\renewcommand{\footrulewidth}{0pt}
}

\pagestyle{fancy}
\fancyhf{}
\fancyhead[RO,LE]{\sffamily CSE 386D}
\fancyhead[LO,RE]{\sffamily UT Austin}
\fancyfoot[LO,RE]{\sffamily Oden Institute}
\fancyfoot[RO,LE]{\sffamily\bfseries\thepage}
\renewcommand{\headrulewidth}{1pt}
\renewcommand{\footrulewidth}{0pt}
\newcommand{\R}{{\mathbb R}}
\newcommand{\N}{{\mathbb N}}
\newcommand{\Z}{{\mathbb Z}}
\newcommand{\Q}{{\mathbb Q}}
\newcommand{\C}{{\mathbb C}}
\newcommand{\SL}{{\mathcal{L}}}
%\usepackage{libertine}            %% For fancy font
\DeclareMathOperator*{\slim}{s-lim}
\newcommand{\cg}{\color{gray}}
\newcommand{\cbk}{\color{black}}
\newcommand{\cred}{\color{red}}
\newcommand{\cblu}{\color{blue}}
\newcommand{\inv}{^{-1}}
\newcommand{\sch}{\mathcal S} 




\begin{document}
%\fontfamily{ptm}\selectfont     %% TO SELECT THE FONT
\title{\vspace{-2\baselineskip} 
Homework 7
}
%\author{Jonathan Zhang \qquad EID: { jdz357} \qquad { jdz@utexas.edu}}
\author{Jonathan Zhang \qquad EID: { jdz357} }
\date{}
\maketitle



\begin{exercise}[8.1]
  If $A$ is positive definite, show that its eigenvalues are positive. If $A$ is symmetric and has positive eigenvalues, then $A$ is positive definite. 
\end{exercise}

\begin{proof}
  Suppose that $A$ is positive definite. Let $\left( \lambda, v \right)$ be an eigenpair for $A$. Then by the positive definiteness, 
    \[0<\left( v, Av \right) = \left( v, \lambda v \right) = \lambda \left( v, v \right) = \lambda \|v\|^2.\]
    Since this quantity is strictly positive, we must have $\lambda >0$. 

  On the other side, 
  assume now that $A$ has all positive eigenvalues. Then, $A$ admits a diagonalization $A= U \Lambda U^*$ for some $U^* = U\inv$ unitary matrix. Pick $x \neq 0$, and set $y = U^* x \neq 0$. Then 
    \[\left( x, Ax \right) = \left( x, U \Lambda U^* x \right) = \left( y, \Lambda y \right).\] 
    Expand $y = \sum y_i e_i$, then 
    \[\left( y, \Lambda y \right) = \sum_{i} \overline{y_i} y_i \lambda _i >0\] since $\lambda_i >0$ and $\overline{y_i}y_i = |y_i|^2 \geq 0 $ since $y \neq 0$ by assumption. This shows that $\left( x, Ax \right) = \left( y, \Lambda y \right) >0$ and $A$ is positive definite. 

\end{proof}






\newpage 
\begin{exercise}[8.3]
Suppose we wish to find $u \in H^2 \left( \Omega \right)$ satisfying $-\Delta u + cu^2 = f \in L^2 \left( \Omega \right)$ for $\Omega \subset \R^d$ is a bounded Lipschitz domain. For consistency, we would require that $cu^2 \in L^2 \left( \Omega \right)$. Determine the smallest $p$ such that if $c \in L^p (\Omega)$, then this holds. Depends on $d$. 
\end{exercise}

\begin{proof}
  We want $cu^2 \in L^2 \left( \Omega \right)$. Suppose that $u \in L^r \left( \Omega \right)$ and $c \in L^p (\Omega)$. Then by Generalized H\"older's, we have the estimate
  \[\|cu^2 \|_2^2 = \int\limits_\Omega c^2 u^4 \leq \|c\|_p^2 \|u\|_r^4 \]
  provided that $2/p+4/r=1$. That is, given $r$, we determine $p = \frac{2}{1-4/r}$, and if $c \in L^p$, then we have $c u ^2 \in L^2 \left( \Omega \right)$.  

  We make use now of the Sobolev embedding theorem. We know $ u \in H^2 \left( \Omega \right) = W^{2, 2} \left( \Omega \right)$. Let $j = 0$, $m = 2$, $p = 2$. Then we have 
  \begin{enumerate}
    \item If $4 < d$, then $W^{2,2} (\Omega) \hookrightarrow W^{0,r}\left( \Omega \right)$ for every $r\leq \frac{2d}{d-4}$. Let us first investigate when $r = \frac{2d}{d-4}$. Substituting in, we find $p = \frac{2d}{8-d}$. We immediately find that we have problems when $ d > 8$, which is admissible in this case. Enforcing that $p \geq 1$, we obtain $ r \geq 4$, so we must have $ r \in [4, \frac{2d}{d-4}]$. However, we run into an issue when $d > 8$. Then, there are no values for $r$ that work, and hence the argument breaks down. 
    
    \item If $d < 4$, then $W^{2,2} (\Omega) \hookrightarrow C_B^0 (\Omega) = C^0 (\Omega) \cap L^\infty (\Omega)$. In this case, we obtain $ r = \infty$, and calculate $ p = 2$, so $c$ can be in $L^2$. 
  \end{enumerate}

  Therefore, we have the following result: 

  \begin{table}[htbp]
	\centering
	\begin{tabular}{c|c}
		$d$ & $p$ \\ \hline
		$<4$ & 2 \\
		$4$ & $2^+$ \\
		$5$ & 10/3 \\
		$6$ & 6 \\
		$7$ & 14 \\
    $8$ & $\infty$ \\\hline
	\end{tabular}
\end{table}


\end{proof}





\newpage
\begin{exercise}[8.5]
  Suppose that $\Omega \subset \R^d$ is a smooth, bounded, connected domain. Let 
  \[H = \left\{ u \in H^2 \left( \Omega \right)  : \int_\Omega u = 0, \nabla u \cdot n = 0 \text{ on } \partial \Omega  \right\}.\]
  Show that $H$ is a Hilbert space, and prove there exists $C > 0$ such that 
  \[\|u\|_{H^1 \left( \Omega \right) } \leq C \sum_{|\alpha| = 2} \|D^\alpha u \|_{L^2 (\Omega)}\]
  for every $u \in H$. 
\end{exercise}

\begin{proof}
  \begin{enumerate}
    \item $H$ is the finite intersection of two null spaces. The null space of the averaging operator and the null space of the $\gamma_1$ operator. Therefore, $H$ is closed. $H$ also inherits the $H^2 \left( \Omega \right)$ inner product. 
    \item Assume to the contrary that the inequality fails. That is, there exists a seqeunce $u_n \in H$ such that 
    \[\|u_n\|_{H^1} = 1 \qquad \text{and} \qquad \sum_{|\alpha| = 2} \|D^\alpha u_n \|_{L^2 (\Omega)} \to 0 . \]

    We remark that the second assumption implies that $D^\alpha u_n \to 0$ in $L^2$ for every $|\alpha| = 2$ multiindex.

    We claim now that $u_n$ is a bounded sequence in $H$. This follows since 
    \[\|u_n\|_{H^2}^2 = \|u_n\|_{H^1}^2 + \sum_{|\alpha| = 2} \|D^\alpha u_n \|_{L^2 (\Omega)}^2 = 1 + C\]
    where $C<\infty$ is the bound for the sequence $\sum_{|\alpha| = 2} \|D^\alpha u_n \|_{L^2 (\Omega)}^2$. We know it is bounded since it converges. Thus, we conclude there is a subsequence $u_{n_k} \rightharpoonup u$ in $H$. 

    On one hand, we have a bounded sequence in $H^1$ from which we can extract a weakly converging subsequence, denoted with the same symbol, $u_n \rightharpoonup u$ in $H^1$. 
    This implies weak convergence $D^\alpha u_{n_k} \rightharpoonup D^\alpha u$ in $L^2$ for every $|\alpha| = 2$. However, we know already that $D^\alpha u_n \to 0$ for every such multiindex, so we conclude that $D^\alpha u = 0$ almost everyhwere. Condition $\nabla u \cdot n = 0$ implies then that $\nabla u = 0$. 

    On the other hand, the R-K theorem implies that there exists a subsequence $u_{n_{k_\ell}} \to u$ in $H^1$. Strong convergence implies strong convergence of $ \nabla u_{n_{k_\ell}} \to \nabla u $ in $L^2$. Finally, we need the Poincare estimate to show the result. Notice that 
    \[1=\|u\|^2_{H^1} \leftarrow \|u_{n_{k_\ell}}\|^2_{H^1}= \|u_{n_{k_\ell}}\|^2 + \|\nabla u_{n_{k_\ell}} \|^2 \leq \left( C_P^2 + 1 \right)\|\nabla u_{n_{k_\ell}} \|^2 \to \left( C_P^2 + 1 \right) \|\nabla u \|^2 \to 0.\]
    


  \end{enumerate}
\end{proof}










\newpage 
\begin{exercise}[8.8]
  $\Omega \subset \R^d$ bounded, connected, Lipschitz domain. Let $V \subset \Omega$ have positive measure. Let $H = \left\{ u \in H^1 (\Omega) : u|_V = 0 \right\}$. 
  \begin{enumerate}
    \item Why is $H$ a Hilbert space? 
    \item Prove that there exists $C>0$ such that 
    \[\|u\| \leq C \|\nabla u\|\]
    for every $u \in H$. 
  \end{enumerate}
\end{exercise}

\begin{enumerate}
  \item $H$ is the null space of the restriction to $V$ operator. Since restriction is continuous, then $H$ is closed. As a closed subspace of Hilbert space $H^1\left( \Omega \right)$, then $H$ is itself a Hilbert space. 
  \item Go by contradiction. Assume to the contrary that there exists a sequence $u_n \in H$ such that 
  \[\|u_n\|= 1 \qquad \text{and} \qquad \|\nabla u_n\| \to 0.\]
  Now, from every bounded sequence in a Hilbert space, we can extract a weakly convergent subsequence $u_{n_k} \rightharpoonup u \in H$. Weak convergence of $u_{n_k}$ implies weak convergence of $\nabla u_{n_k} \rightharpoonup \nabla u$ in $L^2$. Since the norm is continuous we pass to the limit and we have that $\nabla u = 0$, i.e. $u$ is a constant. But if $u$ vanishes on $V$, then the only possibility is that $u \equiv 0$. 

  On the other side, we know from the Rellich-Kondrachov Theorem that we have $u_{n_k} \to u$ in $L^2$. Convergence in $L^2$ also implies convergence in the norm, butt since $\|u_n\| = 1$ that means $\|u\| = 1$, a contradiction. 

\end{enumerate}



\end{document}
 