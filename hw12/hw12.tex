\documentclass[letterpaper,twoside,11pt]{article}
\usepackage{a4wide,graphicx,fancyhdr,clrscode,tabularx,amsmath,amssymb,amsfonts,color,enumitem, bm, array, textcomp, subcaption, color, listings, chemformula, tcolorbox, setspace, xcolor}
\usepackage{amsthm}		%theorem style 
\usepackage{mathptmx}      %SET MATH TYPE FONT TO TIMES NEW ROMAN
%These lines make the theorem NAME BOLD
\newtheoremstyle{mystyle}%                % Name
  {}%                                     % Space above
  {}%                                     % Space below
  {\itshape}%                                     % Body font
  {}%                                     % Indent amount
  {\bfseries}%                            % Theorem head font
  {.}%                                    % Punctuation after theorem head
  { }%                                    % Space after theorem head, ' ', or \newline
  {\thmname{#1}\thmnumber{ #2}\thmnote{ (#3)}}%                                     % Theorem head spec (can be left empty, meaning `normal')
\theoremstyle{mystyle}
\newtheorem{theorem}{Theorem}[section]
%end of making theorem name bold. 
\newtheorem*{thm}{Theorem}		%Theorem no number. 
\newtheorem{definition}{Definition}[section]
\newtheorem{corollary}{Corollary}[theorem]
\newtheorem{lemma}[theorem]{Lemma}
\newtheorem{prop}{Proposition}[section]
\newtheorem*{ex}{Example}
\newtheorem{notee}{Note}[section]
\newtheorem*{exercise}{Exercise}
\newtheorem*{note}{Note}
\usepackage[super]{nth}
\usepackage[makeroom]{cancel}

%----------------------- Macros and Definitions --------------------------
\setlength{\fboxsep}{2.5\fboxsep}
%Sets boxlength size

\setlength\headheight{15pt}
\addtolength\topmargin{-25pt}
\addtolength\footskip{0pt}

\fancypagestyle{plain}{%
\fancyhf{}
\fancyhead[LO,RE]{\sffamily UT Austin}
\fancyhead[RO,LE]{\sffamily CSE 386D}
\fancyfoot[LO,RE]{\sffamily Oden Institute}
\fancyfoot[RO,LE]{\sffamily\bfseries\thepage} 
\renewcommand{\headrulewidth}{0pt}
\renewcommand{\footrulewidth}{0pt}
}

\pagestyle{fancy}
\fancyhf{}
\fancyhead[RO,LE]{\sffamily CSE 386D}
\fancyhead[LO,RE]{\sffamily UT Austin}
\fancyfoot[LO,RE]{\sffamily Oden Institute}
\fancyfoot[RO,LE]{\sffamily\bfseries\thepage}
\renewcommand{\headrulewidth}{1pt}
\renewcommand{\footrulewidth}{0pt}
\newcommand{\R}{{\mathbb R}}
\newcommand{\N}{{\mathbb N}}
\newcommand{\Z}{{\mathbb Z}}
\newcommand{\Q}{{\mathbb Q}}
\newcommand{\C}{{\mathbb C}}
\newcommand{\SL}{{\mathcal{L}}}
%\usepackage{libertine}            %% For fancy font
\DeclareMathOperator*{\slim}{s-lim}
\newcommand{\cg}{\color{gray}}
\newcommand{\cbk}{\color{black}}
\newcommand{\cred}{\color{red}}
\newcommand{\cblu}{\color{blue}}
\newcommand{\inv}{^{-1}}
\newcommand{\ve}{\varepsilon}
\newcommand{\sch}{\mathcal S} 

\newcommand{\Hone}{H^1\left( \Omega \right)}
\newcommand{\Hdiv}{H\left( \text{div}, \Omega \right)}
\newcommand{\Hcur}{H\left( \text{curl}, \Omega \right)}
\newcommand{\Ltwo}{L^2 \left( \Omega \right)}




\begin{document}
\fontfamily{ptm}\selectfont     %% TO SELECT THE FONT
\title{\vspace{-2\baselineskip} 
Homework 12
}
%\author{Jonathan Zhang \qquad EID: { jdz357} \qquad { jdz@utexas.edu}}
\author{Jonathan Zhang \qquad EID: { jdz357} }
\date{}
\maketitle


\begin{exercise}[10.2]
  \[F(y) = \int\limits_0^1 \left\{ y^2 - yy' \right\}dx, \qquad y \in C^1 \left( [0,1] \right).\]
  \begin{enumerate}
    \item Find all extremals. 
    \item If we require $y(0) = 0$, show by example that there is no minimum. 
    \item If we require that $y(0) = y(1) = 0$, show that the extremal is a minimum. (Hint: $yy' = \left( 1/2 y^2  \right)'$).
  \end{enumerate}
\end{exercise}

\cblu 

\begin{enumerate}
  \item Let \(f(x, y, y') = y^2 - yy'\). EL equation:
  \[2y-y'=\left( -y \right)'\]
  gives the extremals $y = 0$. 
  \item For $y = 0$, $F(y) = 0$. However, the function $e^x - 1$ satisfies the boundary condition, and $F(e^x - 1)=2-e < 0$, so, $y = 0$ is not the minimizer. Since $y = 0$ is the only solution satisfying the EL equations, then we conclude that there is no minimum.  
  \item Now, we have 
  \[F(y) = \int\limits_0^1 \left\{ y^2 - (y^2/2)' \right\}dx = \int\limits_0^1 y^2 dx \geq 0. \]
  And, the only function that attains zero is exactly $y = 0$, so it is the minimizer. 
\end{enumerate}

\cbk 

\begin{exercise}[10.4]
  Minimize 
  \[F(y) = \int\limits_0^1 f(x, y(x), y'(x), y''(x)) dx\]
  over the set of $y \in C^2([0,1])$ such that $y(0) = \alpha$, $y'(0) = \beta$, $y(1) = \gamma$, and $y'(1) = \delta$. That is, with $C_0^2([0,1]) = \left\{ u \in C^2 ([0,1]) : u(0) = u'(0) = u(1) = u'(1) = 0 \right\}$, and $y \in C_0^2 + p(x)$ where $p$ is the cubic polynomial that matches the BC. 
  \begin{enumerate}
    \item Find a differential equation, similar to the EL equation, that must be satisfied by the minimum (if it exists). 
    \item Apply your equation to find the extremal(s) of 
    \[F(y) = \int\limits_0^1 \left( y''(x) \right)^2 dx,\]
    where $y(0) = y'(0) = y'(1) = 0$, but $y(1) = 1$, and justify that each extremal is a (possibly non-strict) minimum. 
  \end{enumerate}
\end{exercise}

\cblu 
  We go through the exact same steps as in the text to derive the ordinary EL equations. Let $A$ be defined as 
  \[Ah = \int\limits_0^1 D_2f h + D_3f h' + D_4f h''.\]
  which is a bounded linear functional on $C^2$ with the norm $\|h\| = \max(\|h\|_{\infty}, \|h'\|_{\infty}, \|h''\|_{\infty})$. 
  We show now that $A$ is indeed the Frechet derivative. We have 
  \[\begin{gathered}
    \left| {F\left( {y + h} \right) - F\left( y \right) - Ah} \right| \leqslant \int\limits_0^1 {\int\limits_0^1 {\left| {{D_2}f\left( {x,y + th,y' + th',y'' + th''} \right) - {D_2}f\left( {x,y,y',y''} \right)} \right|dtdx} }  \hfill \\
     \quad\quad\quad \qquad \qquad + \int\limits_0^1 {\int\limits_0^1 {\left| {{D_3}f\left( {x,y + th,y' + th',y'' + th''} \right) - {D_3}f\left( {x,y,y',y''} \right)} \right|dtdx} }  \hfill \\
     \,\,\qquad \qquad + \int\limits_0^1 {\int\limits_0^1 {\left| {{D_4}f\left( {x,y + th,y' + th',y'' + th''} \right) - {D_4}f\left( {x,y,y',y''} \right)} \right|dtdx} } 
  \end{gathered} \]
  Since $D_2f$, $D_3f$, and $D_4f$ are uniformly continuous on compact sets, the right hand side is $o(\|h\|)$ and so $DF(y) = A$. 

  To show that $A$ is continuous, we use the uniform continuity of $D_2f, D_3f, D_4f$ and for $\|k\| \leq 1$, 
  \[\begin{gathered}
    \left| {DF\left( {y + h} \right)k - DF\left( y \right)k} \right| \leqslant \int\limits_0^1 {\left| {\left\{ {{D_2}f\left( {x,y + h,y' + h',y'' + h''} \right) - {D_2}f\left( {x,y,y',y''} \right)} \right\}k} \right|dx}   \\
     + \int\limits_0^1 {\left| {\left\{ {{D_3}f\left( {x,y + h,y' + h',y'' + h''} \right) - {D_3}f\left( {x,y,y',y''} \right)} \right\}k'} \right|dx} \\
     + \int\limits_0^1 {\left| {\left\{ {{D_4}f\left( {x,y + h,y' + h',y'' + h''} \right) - {D_4}f\left( {x,y,y',y''} \right)} \right\}k''} \right|dx}  
  \end{gathered} \]
  which tends to zero as $\|h\| \to 0$. 

  Now we develop the ODE: for $h \in C_0^2$, 
  \begin{align*}
      Ah &= \int\limits_0^1 {{D_2}fh + {D_3}fh' + {D_4}fh''}  \hfill \\
       &= \int\limits_0^1 {\left( {{D_2}f - {{\left( {{D_3}f} \right)}^\prime } + {{\left( {{D_4}f} \right)}^{\prime \prime }}} \right)hdx}  + \left[ {\left( {{D_3}f - {{\left( {{D_4}f} \right)}^\prime }} \right)h} \right]_0^1 + \left[ {{D_3}fh'} \right]_0^1 
  \end{align*}
  so, the ODE is 
  \[{{D_2}f - {{\left( {{D_3}f} \right)}^\prime } + {{\left( {{D_4}f} \right)}^{\prime \prime }}}=0.\]

  If we apply this result to
  \[F(y) = \int\limits_0^1 \left( y''(x) \right)^2 dx,\]
  we see that the ODE reduces to 
  \[y^{(iv) = 0}\]
  Thus, $y$ is the cubic polynomial that matches the boundary conditions. (That is, $y = p$.) Elementary calculation gives 
  \[p = y = -2x^3 + 3x^2.\]
  Moreover, we know that the extremal is a possibly non-strict minimum since $F$ is convex (can be easily seen since $x^2$ is convex),
  \[F\left( {\lambda y + \left( {1 - \lambda } \right)z} \right) = \int\limits_0^1 {{{\left( {\lambda y'' + \left( {1 - \lambda } \right)z''} \right)}^2}}  \leqslant \lambda F\left( y \right) + \left( {1 - \lambda } \right)F\left( z \right)\]
\cbk 



\begin{exercise}[10.6]
  Consider the functional 
  \[\Phi(x, y, y') = \int\limits_a^b F(x, y(x), y'(x))dx.\]
  \begin{enumerate}
    \item If $F \in C^2$ and $F = F(y, y')$ only, and if we assume that $y \in C^2$, prove that in this case the EL equations reduce to 
    \[\frac{d}{dx}\left( F - y'F_{y'} \right) = 0.\]
    \item Among all $C^2$ curves $y(x)$ joining the points $(0,1)$ and $(1, \cosh(1))$, find the one which generates the minimum area when rotated about the $x$ axis. This area is 
    \[A = 2\pi \int\limits_0^1 y \sqrt{1 + (y')^2} dx.\]
  \end{enumerate}
\end{exercise}

\cblu 

\begin{enumerate}
  \item Direct computation. 
  \begin{align*}
      \left( {{D_3}Fy' - f} \right)' &= {D_3}Fy'' + \left( {{D_3}F} \right)'y' - f' \hfill \\
       &= {D_3}Fy'' + {D_2}Fy' - \left( {{D_2}Fy' + {D_3}fy''} \right) \hfill \\
       &= 0 
  \end{align*}
  \item Seek to minimize 
  \[A\left( y \right) = \int\limits_0^1 {2\pi y\sqrt {1 + {{\left( {y'} \right)}^2}} dx} .\]
  Note that $D_3^2f \neq 0$ unless $y = 0$, and clearly $y(x) > 0$, so we have regular extremals. Now, we can use the Theorem when the integrand depends only on the second and third variables, 
  \[2\pi y\frac{{{{y'}^2}}}{{\sqrt {1 + {{\left( {y'} \right)}^2}} }} - 2\pi y\sqrt {1 + {{\left( {y'} \right)}^2}}  = {D_3}fy' - f = 2\pi C\]
  which implies 
  \[y' = \frac{1}{C}\sqrt{y^2 - C^2}.\]
  Separate variables and integrate, (following the problem in the text) gives the solution 
  \[y\left( x \right) = C\cosh \left( {x/C + \lambda } \right).\]
  The boundary conditons give the values for the two constants: 
  \[\begin{gathered}
    y\left( 0 \right) = C\cosh \left( \lambda  \right) = 1 \hfill \\
    y\left( 1 \right) = C\cosh \left( {1/C + \lambda } \right) = \cosh 1 
  \end{gathered} \]
  for which $C = 1$, $\lambda = 0$ is a solution. 
\end{enumerate}

\cbk 



\begin{exercise}[10.8]
  Consider the problem of finding a $C^1$ curve that minimizes 
  \[\int\limits_0^1 (y'(t))^2 dt\]
  subject to the conditions $y(0) = y(1) = 0$ and 
  \[\int\limits_0^1 y^2 = 1.\]
  \begin{enumerate}
    \item Remove the integral constraint by incorporating a Lagrange multiplier, and find the EL equations. 
    \item Find all extremals. 
    \item Find the solution. 
    \item Use your result to find the best constant $C$ in the inequality 
    \[\|y\|\leq C \|y'\|\]
    for functions that satisfy $y(0) = y(1) = 0$. 
  \end{enumerate}
\end{exercise}

\cblu 


\begin{enumerate}
  \item We form the Lagrangian, 
  \[\mathcal{L}\left( {y,\lambda } \right) = \int\limits_0^1 {{{\left( {y'\left( t \right)} \right)}^2}dt}  - \lambda \left[ {\int\limits_0^1 {\left\{ {y{{\left( t \right)}^2} - 1} \right\}dt} } \right] = \int\limits_0^1 {\left\{ {{{\left( {y'\left( t \right)} \right)}^2} - \lambda y{{\left( t \right)}^2} + \lambda } \right\}dt} \]
  which we seek to minimize. To that end, define 
  \[f(t, y, y') = {{{\left( {y'\left( t \right)} \right)}^2} - \lambda y{{\left( t \right)}^2} + \lambda }.\]
  The EL equations are
  \[ - 2\lambda y = {D_2}f = \frac{d}{{dx}}{D_3}f = \frac{d}{{dx}}\left( {2y'} \right) = 2y''.\]
  Hence, the resulting ODE is given simply by $-y'' = \lambda y$, accompanied with the BC $y(0) = y(1) = 0$. 
  \item Extremals can be found by solving the eigenvalue problem. To that end, we note that the operator $-\frac{d^2}{dx^2}$ accompanied with homogeneous boundary conditions is self-adjoint. Therefore, its eigenvalues are non-negative and real. We immediately see that zero cannot be an eigenvalue, for if $\lambda = 0$, then $y'' = 0$, so $y$ is linear, and after incorporating BC, we see that $y \equiv 0$, which does not satisfy the original integral constraint. The general solution is  
  \[y\left( t \right) = A\cos \left( {\sqrt \lambda  t} \right) + B\sin \left( {\sqrt \lambda  t} \right).\]
  BC imply $A = 0$ and 
  \[y\left( 1 \right) = B\sin \left( {\sqrt \lambda  } \right) = 0 \implies \lambda  = {n^2}{\pi ^2},\qquad n =1, 2, \dots .\]
  The coefficient $B$ is determined by satisfying the integral condition.
  \[1 = \int\limits_0^1 {{y^2}}  = {B^2}\int\limits_0^1 {{{\sin }^2}\left( {n\pi t} \right)dt}  \implies  B =  \pm \sqrt 2 .\]
  \item Make the observation 
  \[y\left( t \right) = \sin \left( {n\pi t} \right) \Rightarrow y'\left( t \right) = n\pi \cos \left( {n\pi t} \right).\]
  So, no matter which $n$ we pick, we will always be integrating $\cos^2(n\pi t)$, so the minimizer is the extremal with the smallest admissible $n$, that is, $n = 1$. So, the solution is $y(t) = \sqrt{2} \sin(\pi t)$. 
  \item The best constant is the one attaining the bound, that is, the smallest eigenvalue. This means $C = \pi^{-1}$. 
\end{enumerate}



\cbk 





\begin{exercise}[10.9]
  Find the $C^2$ curve that minimizes the functional 
  \[\int\limits_0^1 \left\{ y(t)^2 + y'(t)^2 \right\}dt \]
  subject to $y(0) = 0$, $y(1) = 1$ and the constraint 
  \[\int\limits_0^1 y = 0\]
\end{exercise}

\cblu 
Lagrangian: 
\[L\left( {y,\lambda } \right) = \int\limits_0^1 {{y^2} + {{\left( {y'} \right)}^2} - \lambda y} \]
EL equations: 
\[2y - \lambda  = {D_2}f = {\left( {{D_3}f} \right)^\prime } = {\left( {2y'} \right)^\prime } = 2y''\]
whose solution is the function 
\[y =  - \frac{{{{{e}}^{ - x}}\left( { - 1 + {{{e}}^x}} \right)\left( { - 2{{e}} + \lambda {{e}} - \lambda {{{e}}^2} - \lambda {{{e}}^x} - 2{{{e}}^{1 + x}} + \lambda {{{e}}^{1 + x}}} \right)}}{{2\left( { - 1 + {{{e}}^2}} \right)}}.\]
The integral constraint gives the value for the Lagrange multiplier, 
\[\lambda = \frac{2(e-1)}{e-3}.\]

We check also when the derivatives of the constraints are zero, but this gives a trivial solution which does not satisfy the original problem. 

\cbk 





\end{document}
 
