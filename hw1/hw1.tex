\documentclass[letterpaper,twoside,11pt]{article}
\usepackage{a4wide,graphicx,fancyhdr,clrscode,tabularx,amsmath,amssymb,amsfonts,color,enumitem, bm, array, textcomp, subcaption, color, listings, chemformula, tcolorbox, setspace, xcolor}
\usepackage{amsthm}		%theorem style 
%\usepackage{mathptmx}      %SET MATH TYPE FONT TO TIMES NEW ROMAN
%These lines make the theorem NAME BOLD
\newtheoremstyle{mystyle}%                % Name
  {}%                                     % Space above
  {}%                                     % Space below
  {\itshape}%                                     % Body font
  {}%                                     % Indent amount
  {\bfseries}%                            % Theorem head font
  {.}%                                    % Punctuation after theorem head
  { }%                                    % Space after theorem head, ' ', or \newline
  {\thmname{#1}\thmnumber{ #2}\thmnote{ (#3)}}%                                     % Theorem head spec (can be left empty, meaning `normal')
\theoremstyle{mystyle}
\newtheorem{theorem}{Theorem}[section]
%end of making theorem name bold. 
\newtheorem*{thm}{Theorem}		%Theorem no number. 
\newtheorem{definition}{Definition}[section]
\newtheorem{corollary}{Corollary}[theorem]
\newtheorem{lemma}[theorem]{Lemma}
\newtheorem{prop}{Proposition}[section]
\newtheorem*{ex}{Example}
\newtheorem{notee}{Note}[section]
\newtheorem*{note}{Note}
\usepackage[super]{nth}
\usepackage[makeroom]{cancel}

%----------------------- Macros and Definitions --------------------------
\setlength{\fboxsep}{2.5\fboxsep}
%Sets boxlength size

\setlength\headheight{15pt}
\addtolength\topmargin{-25pt}
\addtolength\footskip{0pt}

\fancypagestyle{plain}{%
\fancyhf{}
\fancyhead[LO,RE]{\sffamily UT Austin}
\fancyhead[RO,LE]{\sffamily CSE 386C}
\fancyfoot[LO,RE]{\sffamily Oden Institute}
\fancyfoot[RO,LE]{\sffamily\bfseries\thepage} 
\renewcommand{\headrulewidth}{0pt}
\renewcommand{\footrulewidth}{0pt}
}

\pagestyle{fancy}
\fancyhf{}
\fancyhead[RO,LE]{\sffamily CSE 386C}
\fancyhead[LO,RE]{\sffamily UT Austin}
\fancyfoot[LO,RE]{\sffamily Oden Institute}
\fancyfoot[RO,LE]{\sffamily\bfseries\thepage}
\renewcommand{\headrulewidth}{1pt}
\renewcommand{\footrulewidth}{0pt}
\newcommand{\R}{{\mathbb R}}
\newcommand{\N}{{\mathbb N}}
\newcommand{\Z}{{\mathbb Z}}
\newcommand{\Q}{{\mathbb Q}}
\newcommand{\C}{{\mathbb C}}
\newcommand{\SL}{{\mathcal{L}}}
\usepackage{libertine}            %% For fancy font
\DeclareMathOperator*{\slim}{s-lim}
\newcommand{\cg}{\color{gray}}
\newcommand{\cbk}{\color{black}}
\newcommand{\cred}{\color{red}}
\newcommand{\cblu}{\color{blue}}
\newcommand{\inv}{^{-1}}




\begin{document}
%\fontfamily{ptm}\selectfont     %% TO SELECT THE FONT
\title{\vspace{-2\baselineskip} 
Homework 13
}
%\author{Jonathan Zhang \qquad EID: { jdz357} \qquad { jdz@utexas.edu}}
\author{Jonathan Zhang \qquad EID: { jdz357} }
\date{}
\maketitle


\subsection*{Problem 1:}
Let $X$ be a Hilbert space, $\mathcal D \subset X$ dense, and $A: \mathcal D \to X$ a closed Hermitian linear operator. Let $\lambda, \mu$ be two real numbers that are not eigenvalues of $A$, and suppose that $A-\mu$ has a compact inverse. Prove that 
\begin{enumerate}[label=(\alph*)]
  \item $(A-\mu)^{-1}$ is self-adjoint, 
  \item $A-\lambda$ has compact inverse,
  \item $(A-\mu)(A-\lambda)^{-1}$ maps $\mathcal D$ to itself and extends to a self-adjoint isomorphism of $X$. 
\end{enumerate}

\paragraph*{Solution} 
\begin{enumerate}[label=(\alph*)]
  \item We check simply that the operator is self adjoint: 
  \[\left\langle y, (A-\mu)^{-1} x  \right\rangle = \langle (A-\mu)^{-*}y, x \rangle = \langle ((A-\mu)^*)^{-1} y, x\rangle  = \langle(A-\mu)^{-1}y, x\rangle \]
  since $A = A^*$ and $\mu = \mu^*$ since $\mu \in \R$. 
  \item Since $(A-\mu)\inv$ is compact and symmetric, it admits a representation 
  \[(A-\mu)\inv x = \sum_n \alpha_n\langle e_n, x\rangle e_n.\] Since we can write $A-\lambda = A-\mu + (\mu-\lambda)$, then the corresponding representation for $(A-\lambda)\inv$ will be \[(A-\lambda)\inv = \sum_n (\alpha_n\inv + \mu - \lambda)\inv \langle e_n, x\rangle e_n. \]
  Notice that the coefficients decay by the behavior of $\alpha_n$'s. Therefore, as the limit of a sequence of finite rank operators, we can conclude that $(A-\lambda)\inv$ is compact. 
\item Consider the map $(A-\mu)(A-\lambda)^{-1}$. We will show that it takes $\mathcal D$ to itself. Indeed, for any $x\in \mathcal D$, 
\[(A-\mu)(A-\lambda)^{-1} x = (A-\lambda)(A-\lambda)\inv x + (\lambda-\mu)(A-\lambda)\inv x= x + (\lambda-\mu)(A-\lambda)\inv x.\]
Comparing the LHS and RHS, we see that the RHS lives in $\mathcal D$ only, so the LHS must also live in $\mathcal D$. Now, extend via limits to the whole space $X$. Notice that the inverse map is easy to determine: just switch the roles of $\lambda$ and $\mu$. 

\end{enumerate}





\newpage \subsection*{Problem 2:}
Let $A$ and $B$ be self-adjoint bounded linear operators on a Hilbert space. If $A\geq B \geq 0$ and $B$ is invertible, prove that $A$ is invertible and that $B^{-1}\geq A^{-1} \geq 0$ 

Hint show that $I\geq A^{-1/2}BA^{-1/2}\implies A^{1/2}B^{-1}A^{1/2}\geq I$.

\paragraph*{Solution} First we argue that $A$ is invertible. Indeed if $B$ is invertible, then zero is not in the spectrum of $B$. Since $A\geq B$, then we have that $\lambda_A \geq \lambda_B$ for every $\lambda_B\in\sigma(B)$. Therefore zero is also not in the spectrum of $A$ and thus $A$ is invertible. 

Then to show the inequality, we have a lemma: 
\begin{lemma}
  Let $A$ be a self adjoint, positive, invertible operator. Then its inverse is also positive. 
\end{lemma}
\begin{proof}
  Define $y = Ax$. Then $\langle y, A^{-1} y\rangle = \langle Ax, x\rangle \geq 0$. 
\end{proof}
So, if $I \geq A^{-1/2}BA^{-1/2}$, then multiplication by (positive and symmetric) operators $A^{1/2}$ and $B^{-1}$ imply that $A^{1/2}B^{-1}A^{1/2}\geq I$. (We use here the fact that the square root is positive.) To arrive at the conclusion, notice that the hint really is just saying $A\geq B \implies B^{-1}\geq A^{-1}$. Since we showed that $A^{-1}$ is positive, we have the result. 

\subsection*{Problem 3:}
On $C^k(\R^n)$ consider the norm $\|f\|_k = \sup\left\{ |D^\alpha f(x) | : x\in \R^n, |\alpha|\leq k \right\}$. For $x, y\in \R^n$ and $f: \R^n \to \C$ define $(\mathcal T_y f)(x)= f(x-y)$. Show that if $f\in C^k(\R^n)$ with $k\geq 2$, then 
\[\|\mathcal T_y f - f + y\cdot \nabla f\|_{k-2} \leq 2|y|^2 \|f\|_k\] with $|y| = \sum_{j=1}^n |y_j|$.

\paragraph*{Solution} Proof goes by induction. In the base case, let $k = 2$. Then, we apply the mean value theorem and conclude that for any $x, y \in \R^n$, there exists $t, t_i \in [0,1]$, $i = 1,...n$ such that
\[f(x) - f(x-y) = y\cdot \nabla f(x-ty)\]
and 
\[\partial_i f(x) - \partial_i f(x-ty) = ty\cdot \nabla\partial_i f(x-t_i y).\] This means that 
\begin{align*}
  |\mathcal T_y f(x) - f(x) + y\cdot\nabla f(x)| &= |f(x-y) - f(x) + y\cdot \nabla f(x)| \\
  &= |-y\cdot \nabla f(x-ty) + y\cdot \nabla f(x) | \\
  &= y\cdot (\nabla f(x) - \nabla f(x-ty))| \\
  &= \left\vert \sum_i y_i (\partial_i f(x) - \partial_i f(x-ty))\right\vert \\
  &=  \left\vert \sum_i y_i (ty\cdot \nabla \partial _i f(x-t_i y))\right\vert \\
  &= \left\vert \sum_i y_i \left( \sum_j ty_j\partial_j\partial_i f(x-t_iy)\right)\right\vert \\
  &\leq t |y|^2 \left\vert \sum_{i,j}\partial_j\partial_i f(x-t_iy) \right\vert \\
  &\leq |y|^2 \|f\|_2 
\end{align*} 
Now we recover the norm by taking sup on both sides, so that $\|\mathcal T_yf - f + y\cdot \nabla f\|_0 \leq |y|^2 \|f\|_2$. Now, we can easily extend this to the general case. For $f \in C^k$, what we have just shown holds now with $f$ replaced by $D^\alpha f$ where $|\alpha|\leq k-2$. Taking the supremum again now produces the desired result. 






\subsection*{Problem 4:} Prove Poisson sum formula, $\sum_k e^{ikx} = 2\pi \delta(x)$ for $|x|< \pi$, i.e. 
\[\sum_{k=-\infty}^\infty \int\limits_{-\pi}^\pi e^{ikx} f(x) dx = 2\pi f(0), \qquad f\in \mathcal D((-\pi, \pi)).\]

\paragraph*{Solution} The identity is essentially just two Fourier transforms. Define the Fourier transform $\mathcal F : L^2(-\pi, \pi) \to \ell^2(\Z) $ given by \[(\mathcal F f)(k) = \frac{1}{\sqrt{2\pi}}\int\limits_{-\pi}^\pi f(\xi) e^{ik\xi }d\xi\]
Its inverse is given by 
\[(\mathcal F^{-1} \hat f)(x) = \frac{1}{\sqrt{2\pi}}\sum_k \hat f(k) e^{-ikx}\]
Combine these together, 
\[f(x) = \frac{1}{2\pi}\sum_k \int\limits_{-\pi}^\pi f(\xi) e^{ik\xi}d\xi e^{-ikx}\]
from which one evaluates at $x=0$ to get the result:
\[f(0) = \frac{1}{2\pi}\sum_k \int\limits_{-\pi}^\pi f(\xi) e^{ik\xi}d\xi .\]

\end{document}
 