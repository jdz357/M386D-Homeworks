\documentclass[letterpaper,twoside,11pt]{article}
\usepackage{a4wide,graphicx,fancyhdr,clrscode,tabularx,amsmath,amssymb,amsfonts,color,enumitem, bm, array, textcomp, subcaption, color, listings, chemformula, tcolorbox, setspace, xcolor}
\usepackage{amsthm}		%theorem style 
%\usepackage{mathptmx}      %SET MATH TYPE FONT TO TIMES NEW ROMAN
%These lines make the theorem NAME BOLD
\newtheoremstyle{mystyle}%                % Name
  {}%                                     % Space above
  {}%                                     % Space below
  {\itshape}%                                     % Body font
  {}%                                     % Indent amount
  {\bfseries}%                            % Theorem head font
  {.}%                                    % Punctuation after theorem head
  { }%                                    % Space after theorem head, ' ', or \newline
  {\thmname{#1}\thmnumber{ #2}\thmnote{ (#3)}}%                                     % Theorem head spec (can be left empty, meaning `normal')
\theoremstyle{mystyle}
\newtheorem{theorem}{Theorem}[section]
%end of making theorem name bold. 
\newtheorem*{thm}{Theorem}		%Theorem no number. 
\newtheorem{definition}{Definition}[section]
\newtheorem{corollary}{Corollary}[theorem]
\newtheorem{lemma}[theorem]{Lemma}
\newtheorem{prop}{Proposition}[section]
\newtheorem*{ex}{Example}
\newtheorem{notee}{Note}[section]
\newtheorem*{note}{Note}
\usepackage[super]{nth}
\usepackage[makeroom]{cancel}

%----------------------- Macros and Definitions --------------------------
\setlength{\fboxsep}{2.5\fboxsep}
%Sets boxlength size

\setlength\headheight{15pt}
\addtolength\topmargin{-25pt}
\addtolength\footskip{0pt}

\fancypagestyle{plain}{%
\fancyhf{}
\fancyhead[LO,RE]{\sffamily UT Austin}
\fancyhead[RO,LE]{\sffamily CSE 386D}
\fancyfoot[LO,RE]{\sffamily Oden Institute}
\fancyfoot[RO,LE]{\sffamily\bfseries\thepage} 
\renewcommand{\headrulewidth}{0pt}
\renewcommand{\footrulewidth}{0pt}
}

\pagestyle{fancy}
\fancyhf{}
\fancyhead[RO,LE]{\sffamily CSE 386D}
\fancyhead[LO,RE]{\sffamily UT Austin}
\fancyfoot[LO,RE]{\sffamily Oden Institute}
\fancyfoot[RO,LE]{\sffamily\bfseries\thepage}
\renewcommand{\headrulewidth}{1pt}
\renewcommand{\footrulewidth}{0pt}
\newcommand{\R}{{\mathbb R}}
\newcommand{\N}{{\mathbb N}}
\newcommand{\Z}{{\mathbb Z}}
\newcommand{\Q}{{\mathbb Q}}
\newcommand{\C}{{\mathbb C}}
\newcommand{\SL}{{\mathcal{L}}}
\usepackage{libertine}            %% For fancy font
\DeclareMathOperator*{\slim}{s-lim}
\newcommand{\cg}{\color{gray}}
\newcommand{\cbk}{\color{black}}
\newcommand{\cred}{\color{red}}
\newcommand{\cblu}{\color{blue}}
\newcommand{\inv}{^{-1}}




\begin{document}
%\fontfamily{ptm}\selectfont     %% TO SELECT THE FONT
\title{\vspace{-2\baselineskip} 
Homework 1
}
%\author{Jonathan Zhang \qquad EID: { jdz357} \qquad { jdz@utexas.edu}}
\author{Jonathan Zhang \qquad EID: { jdz357} }
\date{}
\maketitle


\subsection*{Problem 6.2:}
Compute the Fourier transform of $\exp\left( -a| x| ^2 \right)$, $a>0$, directly for $x\in \R$. 

\paragraph*{Solution} 
First of all, notice that the Gaussian is an $L^1$ function. 
Computing directly,
\begin{align*}
    \hat f\left( \xi  \right) &= {\left( {2\pi } \right)^{ - d/2}}\int\limits_{{\mathbb{R}^d}} {{e^{ - a{{\left| x \right|}^2} - ix \cdot \xi }}dx}  \hfill \\
     &= {\left( {2\pi } \right)^{ - d/2}}\int\limits_{{\mathbb{R}^d}} {{e^{ - a{{\left( {x + \frac{{i\xi }}{{2a}}} \right)}^2} - \frac{{{\xi ^2}}}{{4a}}}}dx}  \hfill \\
     &= {\left( {2\pi } \right)^{ - d/2}}{e^{ - \frac{{{\xi ^2}}}{{4a}}}}\int\limits_{{\mathbb{R}^d}} {{e^{ - a{{\left( {x + \frac{{i\xi }}{{2a}}} \right)}^2}}}dx}.  
\end{align*}
To compute the remaining integral, we first observe that the integrand is a holomorphic function, so Cauchy's Theorem implies that the integral over any closed contour is zero. We choose the rectangle prescribed by vertices $-R, R, -R-\frac{i\xi}{2a},R-\frac{i\xi}{2a}$.  



\subsection*{Problem 6.3:}
If $f\in L^1\left( \R^d \right)$ and $f>0$, show that for every $\xi\neq 0$, $\left\vert \hat f \left( \xi \right) \right\vert< \hat f\left( 0 \right)$. 

\paragraph*{Solution} By definition, 
\[\hat f\left( 0 \right) = {\left( {2\pi } \right)^{ - d/2}}\int\limits_{{\mathbb{R}^d}} {f\left( x \right)dx} .\]
The weak inequality follows by direct computation, 
\begin{align*}
  \left| {\hat f\left( \xi  \right)} \right| &= \left| {{{\left( {2\pi } \right)}^{ - d/2}}\int\limits_{{\mathbb{R}^d}} {f\left( x \right){e^{ - ix \cdot \xi }}dx} } \right| \hfill \\
   &\leqslant {\left( {2\pi } \right)^{ - d/2}}\int\limits_{{\mathbb{R}^d}} {\left| {f\left( x \right)} \right|dx}  \hfill \\
   &\leqslant {\left( {2\pi } \right)^{ - d/2}}\int\limits_{{\mathbb{R}^d}} {f\left( x \right)dx}  \hfill \\
   &= \hat f\left( 0 \right) 
  \end{align*}



\subsection*{Problem 6.4:}
If $f\in L^1\left( \R^d \right)$ and $f(x) = g\left( \left\vert x\right\vert  \right)$ for some $g$, show that $\hat f\left( \xi \right) = h\left( \left\vert \xi\right\vert \right) $ for some $h$. Can you relate $g$ and $h$? 

\paragraph*{Solution} 



\subsection*{Problem 6.6:}
Show that the Fourier transform $\mathcal F : L^1\left( \R^d \right)\to C_v\left( \R^d \right)$ is not onto. Show that $\mathcal F\left( L^1\left( \R^d \right) \right)$ is dense in $C_v\left( \R^d \right)$. 

\paragraph*{Solution} 
Some like Fourier Inverse Theorem, Open Mapping Theorem, conclude bounded inverse. 


\end{document}
 