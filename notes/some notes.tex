\documentclass[letterpaper,twoside,12pt]{article}
\usepackage{a4wide,graphicx,fancyhdr,clrscode,tabularx,amsmath,amssymb,amsfonts,color,enumitem, bm, bbm, array, textcomp, subcaption, color, listings, chemformula, tcolorbox, setspace, xcolor, hyperref}
\usepackage{amsthm}		%theorem style 
%\usepackage{geometry}  %Adjust margins if needed. 
%\geometry{margin=1.25in} 
%\usepackage{mathptmx}      %SET MATH TYPE FONT TO TIMES NEW ROMAN
%These lines make the theorem NAME BOLD
\newtheoremstyle{mystyle}%                % Name
  {}%                                     % Space above
  {}%                                     % Space below
  {\itshape}%                                     % Body font
  {}%                                     % Indent amount
  {\bfseries}%                            % Theorem head font
  {.}%                                    % Punctuation after theorem head
  { }%                                    % Space after theorem head, ' ', or \newline
  {\thmname{#1}\thmnumber{ #2}\thmnote{ (#3)}}%                                     % Theorem head spec (can be left empty, meaning `normal')
\theoremstyle{mystyle}
\newtheorem{theorem}{Theorem}[section]
%end of making theorem name bold. 
\newtheorem*{thm}{Theorem}		%Theorem no number. 
\newtheorem{definition}{Definition}[section]
\newtheorem{corollary}{Corollary}[theorem]
\newtheorem{lemma}[theorem]{Lemma}
\newtheorem*{lemm}{Lemma}
\newtheorem{prop}{Proposition}[section]
\newtheorem*{propp}{Proposition}
\newtheorem*{ex}{Example}
\newtheorem*{example}{Example}
\newtheorem{notee}{Note}[section]
\newtheorem*{note}{Note}
\usepackage[super]{nth}
\usepackage[makeroom]{cancel}

%----------------------- Macros and Definitions --------------------------
\setlength{\fboxsep}{2.5\fboxsep}
%Sets boxlength size

\setlength\headheight{15pt}
\addtolength\topmargin{-25pt}
\addtolength\footskip{0pt}

\fancypagestyle{plain}{%
\fancyhf{}
\fancyhead[LO,RE]{\sffamily UT Austin}
\fancyhead[RO,LE]{\sffamily CSE 386D}
\fancyfoot[LO,RE]{\sffamily Oden Institute}
\fancyfoot[RO,LE]{\sffamily\bfseries\thepage} 
\renewcommand{\headrulewidth}{0pt}
\renewcommand{\footrulewidth}{0pt}
}

\pagestyle{fancy}
\fancyhf{}
\fancyhead[RO,LE]{\sffamily CSE 386D}
\fancyhead[LO,RE]{\sffamily UT Austin}
\fancyfoot[LO,RE]{\sffamily Oden Institute}
\fancyfoot[RO,LE]{\sffamily\bfseries\thepage}
\renewcommand{\headrulewidth}{1pt}
\renewcommand{\footrulewidth}{0pt}
\newcommand{\R}{{\mathbb R}}
\newcommand{\N}{{\mathbb N}}
\newcommand{\Z}{{\mathbb Z}}
\newcommand{\Q}{{\mathbb Q}}
\newcommand{\C}{{\mathbb C}}
\newcommand{\SL}{{\mathcal{L}}}
\newcommand{\DD}{\mathcal D}
\newcommand{\ScS}{\mathcal S}
\newcommand{\Om}{\Omega}
\DeclareMathOperator*{\slim}{s-lim}
\newcommand{\cg}{\color{gray}}
\newcommand{\cbk}{\color{black}}
\newcommand{\cred}{\color{red}}
\newcommand{\cblu}{\color{blue}}
\usepackage{libertine}            %% For fancy font
\begin{document}
%\fontfamily{ptm}\selectfont     %% TO SELECT THE FONT
\title{\vspace{-2\baselineskip} 
Some Notes and Theorems, Part 2
}
\author{Jonathan Zhang}
\date{}
\maketitle


\section{Preliminaries}
\section{Normed Linear Spaces and Banach Spaces}
\section{Hilbert Spaces}
\section{Spectral Theory and Compact Operators}
\section{Distributions}
\begin{tcolorbox}[colback=red!5!white,colframe=red!75!black]
  \begin{theorem}
    Here, we use the following notation: $\left( {{M^\alpha }f} \right)\left( x \right) = {x^\alpha }f\left( x \right)$.
    
    If $T\in \ScS'$ and $f \in \ScS$, then $T\ast f$ is smooth and polynomially bounded. 
  \end{theorem}
  \end{tcolorbox}
  
  


\newpage\section{The Fourier Transform}

We motivate our discission on the Fourier Transform by observing that it converts the task of solving a PDE into the task of solving an ODE, and that we can represent functions in terms of harmonic functions. 

The Fourier Transform can be defined naturally for functions in $L^1\left( \R^d \right)$, where we will start. Then, we can extend it to functions in $L^2\left( \R^d \right)$, and we shall see that it maps $L^2$ onto itself. Then, we will discuss the Fourier transform in context of distributions. 

\subsection{$L^1\left( \R^d \right)$ Theory}

For \cg (a point) \cbk $\xi \in \R^d$, the function 
\[\varphi_\xi : x\mapsto  e^{-ix\cdot \xi} = \cos\left( x\cdot \xi  \right) - i \sin \left( x\cdot \xi \right)\] for $x \in \R^d$ defines a wave in the direction $\xi$. Moreover, the period of $\varphi_\xi$ in the $j$-th direction is $2\pi / \xi_j$. 
\cg Consider the simplest nontrivial example, say when we are in $\R^2$ and we let $\xi = \left( 1, 1 \right)$. Then, the function $\varphi_\xi (x) = e^{-i(x + y)} = \cos\left( x + y \right) - i \sin \left( x + y \right)$ defines a wave that propagates in the direction $\left( 1, 1 \right)$. Look at the real part by itself, for example. It is clear that the period in the $x$ direction is $2\pi$, indeed as it should be. The same can be easily seen with the imaginary part. \cbk These harmonic functions have certain nice properties: 

\begin{prop}
  \begin{enumerate}
    \item $\left| {{\varphi _\xi }} \right| = 1$, and $\overline{\varphi_\xi} = \varphi_{-\xi}$ for every $\xi \in \R^d$, 
    \item $\varphi_\xi \left( x + y \right) = \varphi_\xi\left( x \right)\varphi_\xi\left( y \right)$ for every $x, y, \xi\in \R^d$, 
    \item $-\nabla^2 \varphi_\xi = |\xi|^2 \varphi_\xi$ for every $\xi \in \R^d$. 
  \end{enumerate}
\end{prop}
\begin{proof}
  \begin{enumerate}
    \item We clearly have 
    \[{\left| {{\varphi _\xi }} \right|^2} = {\cos ^2}\left( {x \cdot \xi } \right) + {\sin ^2}\left( {x \cdot \xi } \right) = 1,\]
    and 
    \[\overline{\varphi_\xi}  = \cos\left( x\cdot \xi  \right) + i \sin\left( x \cdot \xi \right) = \cos\left( -x \cdot \xi \right) - i\sin\left( -x\cdot \xi \right) = \varphi_{-\xi}.\]
    \item \[\varphi_\xi \left( x + y \right) = e^{-i\left( x + y \right)\cdot \xi} = e^{-ix\cdot \xi -iy\cdot \xi} = e^{-ix\cdot \xi} e^{-iy\cdot \xi} = \varphi_\xi \left( x \right) \varphi_\xi \left( y \right).\]
    \item \[-\nabla^2 \varphi_\xi = - \left( -i\xi \right)^2 e^{-ix\cdot \xi} = |\xi|^2 \varphi_\xi.\] We remark that this property demonstrates that functions $\varphi_\xi$ are eigenfunctions of the Laplacian operator with eigenvalues $-|\xi|^2$. 
  \end{enumerate}
\end{proof}

We now define the Fourier Transform: 
\begin{definition}
  For $f \in L^1\left( \R^d \right)$, the Fourier Transform of $f$ is 
  \[\left( \mathcal F f \right)\left( \xi \right) =\hat f \left( \xi  \right)= \frac{1}{\left( 2\pi  \right)^{d/2}} \int\limits_{\R^d } f(x) e^{-ix\cdot \xi} dx.\]
\end{definition}
At this moment, it is not immediately clear what function spaces the Fourier Transform connects. However, for $f \in L^1$, the Fourier transform is well defined since 
\[\left\vert f(x) e^{- ix\cdot \xi} \right\vert = |f(x)|.\]
\cg What we mean when we say well defined is that the integral defining $\hat f$ converges whenever $f \in L^1$. Indeed, $f \in L^1$ iff $\int |f(x)| dx < \infty$, but as we just showed, $\int |f(x) | dx = \int \left\vert f(x) e^{- ix\cdot \xi} \right\vert dx$, so the integral converges. \cbk 

Additionally, we remark that it is possible (and equivalent) to define the Fourier transform in different ways, but that the end results are very similar and it is merely a matter of preference. 

We preface some additional discussion with a motivating example to show that we need to take care when talking about function spaces:
\begin{example}
  On $\R$, let $f : x \mapsto \chi_{[-1/2, 1/2]}(x)$, the characteristic function on the interval $[-1/2, 1/2]$. Obviously, $f$ is an $L^1\left( \R \right)$ function. Now for every $\xi\neq 0$, 
  \begin{align*}
    \hat f (\xi) &= \left( 2\pi \right)^{-1/2} \int\limits_\R f(x) e^{-ix\cdot \xi} dx \\
    &= \left( 2\pi \right)^{-1/2} \int\limits_{-1/2}^{1/2} e^{-ix\cdot \xi} dx \\
    &=\left( 2\pi \right)^{-1/2} \frac{2}{k}\sin\left( \frac{k}{2} \right).
  \end{align*}
  This is fine, but notice that $\hat f$ is no longer integrable! So, just because $f \in L^1$, we have no reason to believe that $\hat f$ is also in $L^1$. 
\end{example}
We have now the following proposition: 
\begin{prop}
  The Fourier Transform viewed in the spaces
  \[\mathcal F : L^1\left( \R^d \right) \to L^\infty \left( \R^d \right)\]
  is a bounded linear map, and 
  \[\left\| \hat f \right\|_{L^\infty \left( \R^d \right)} \leq \left( 2\pi \right)^{-d/2} \left\| f \right\|_{L^1\left( \R^d \right)}. \]
\end{prop}
\begin{proof}
  Clearly, $\mathcal F$ is linear since the integral is a linear operator. 
\end{proof}




\subsection{$\mathcal S \left( \R^d \right)$ Theory}

\subsection{$L^2\left( \R^d \right)$ Theory}

\subsection{$\ScS'$ Theory}

\subsection{Applications}











































\newpage\section{Sobolev Spaces}
\subsection{Definitions, Basic Properties}
\subsection{Extensions from $\Omega$ to $\R^d$}
\subsection{Sobolev Imbedding Theorem}
\subsection{Compactness}
\subsection{The Spaces $H^s$}
\subsection{Trace Theorem}
\subsection{The Spaces $W^{s, p}\left( \Omega \right)$}

\newpage\section{Boundary Value Problems}
\subsection{Second Order Elliptic PDEs}
\subsection{A Variational Problem and Minimization of Energy}
\subsection{Closed Range Theorem, Operators Bounded Below}
\subsection{The Lax-Milgram Theorem}
\subsection{Applications to Second Order Elliptic Equations}
\subsection{Galerkin Approximations}
\subsection{Green's Functions}

\newpage\section{Differential Calculus in Banach Spaces}
\newpage\section{Calculus of Variations }


\end{document}
 